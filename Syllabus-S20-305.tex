\documentclass[10pt]{article}
\usepackage{mdwlist}

\usepackage{geometry} 
\geometry{letterpaper, textwidth=5.5in, textheight=8.5in, marginparsep=7pt, marginparwidth=.6in}
\setlength\parindent{0in}
\usepackage[no-math]{fontspec}
\setmainfont[Ligatures=TeX]{Adobe Garamond Pro}

\usepackage[usenames,dvipsnames]{color}
\usepackage{marginnote}
\newcommand{\years}[1]{\vspace{2pt}\marginnote{\scriptsize #1}}
\renewcommand*{\raggedleftmarginnote}{}
\setlength{\marginparsep}{7pt}
\reversemarginpar

\usepackage{sectsty} 
\usepackage[normalem]{ulem} 
\sectionfont{\mdseries\upshape\Large}
\subsectionfont{\mdseries\scshape\normalsize} 
\subsubsectionfont{\mdseries\upshape\large} 
\usepackage{marvosym}

\usepackage[bookmarks, colorlinks, breaklinks, 
 pdftitle={Syllabus - Philosophy 305 - Summer 2020},
 pdfauthor={Brian Weatherson},
 pdfproducer={http://brian.weatherson.org/}
]{hyperref}  
\hypersetup{linkcolor=blue,citecolor=blue,filecolor=black,urlcolor=MidnightBlue} 

% DOCUMENT
\begin{document}
{\LARGE Philosophy 305 - Introduction to Formal Methods}\\[0.5cm]
{\large Summer 2020}

\section*{Instructor}

Brian Weatherson \\
Department of Philosophy \\
2207 Angell Hall \\

\Letter: \href{mailto:weath@umich.edu}{weath@umich.edu}

\ComputerMouse: \href{http://canvas.umich.edu}{http://canvas.umich.edu} \\

Office Hours: Tuesday/Thursday 10-11, via Zoom (links to be announced), and by appointment \\

\section*{Course Description}

This course will introduce some important formal tools that are used elsewhere in philosophy. We will look at propositional logic, probability theory and the logic of modals and conditionals. Obviously that's a lot to cover in a short time - the aim here is to make sure you understand the basics, and the symbolism, so you can follow simple applications of these tools, and you have the foundations to understand more complicated applications.

\section*{Canvas}

There is a Canvas site for this course, which can be accessed from \url{https://canvas.umich.edu}. Course documents (syllabus, lecture notes, assignments) will be available from this site. Please make sure that you can access this site. Consult the site regularly for announcements, including changes to the course schedule. And there are many tools on the site to communicate with each other, and with me. 

\section*{Required Materials}

Three books, which can be downloaded. The third is available on Canvas, the other two are from external sites.

\begin{itemize*}
\item \textit{The Carnap Book} by Graham Leach-Krause, Available at \url{http://carnap.io}.
\item \textit{Odds and Ends} by Jonathan Weisberg, Available at \url{https://jonathanweisberg.org/vip/}
\item \textit{Boxes and Diamonds: An Open Introduction to Modal Logic, Ann Arbor remix}.
\end{itemize*}

\newpage

\section*{Course Requirements}

\begin{enumerate}
\item \emph{Do five-six weekly quizzes}. During the term six weekly assignments will be posted. Your best five scores will count towards the final grade. If you are away or sick one week, you should just skip that quiz. We don't really have a way of giving extensions on quizzes, because once the answers are out there isn't much point, so allowing some missed quizzes allows for some absences. \smallskip

Some of the quizzes will be through Canvas, and others will be through the Carnap site that Professor Leach-Krause built to accompany his book. Each quiz will be due at \textbf{5pm} on Friday. We will spend some time the following Tuesday discussing the quiz at office hours. \smallskip

Each quiz counts for 15\% of the grade, so collectively they count for 75\% of the grade. 

\item \emph{Final exam}. There will be a relatively short test in the exam period, held on Canvas. This will cover all the material from the course. The intent is that it will not be nearly as hard as the quizzes; anyone who has been doing their own work during the term should do fine on it.

\end{enumerate}

Note that you do \textbf{not} have to attend any classes. This is an entirely virtual course. I will be posting lectures that you can download and watch whenever, though you are under no obligation to do so. And I will be holding online office hours, to which you can come along if you have questions. But mostly this will be taught asynchronously. 

\section*{Grade Breakdown}

\begin{itemize*}
\item Assignments: 75\%
\item Exam: 25\% 
\end{itemize*}


\newpage
\section*{Plagiarism}

\noindent  You are responsible for making sure that none of your work is plagiarized. If you do need citations, and it's rare in this course, be sure to cite work that you use, both direct quotations and paraphrased ideas. Any citation method that is tolerably clear is permitted, but if you'd like a good description of a citation scheme that works well in philosophy, look at \url{http://bit.ly/VDhRJ4}.\smallskip

You are encouraged to discuss the course material, including assignments, with your classmates, but all written work that you hand in under your own name must be your own. If work is handed is as the work of multiple people, you are affirming that each person did a fair share of the work. (Note that when you're submitting work on Canvas, you have to each submit the paper, even if it is co-authored. That way Canvas knows that everyone has turned in work.)\smallskip

You should also be familiar with the academic integrity policies of the College of Literature, Science \& the Arts at the University of Michigan, which are available here: \url{http://www.lsa.umich.edu/academicintegrity/}. Violations of these policies will be reported to the Office of the Assistant Dean for Student Academic Affairs, and sanctioned with a course grade of F.

\section*{Disability}

\noindent  The University of Michigan abides by the Americans with Disabilities Act of 1990, Section 504 of the Rehabilitation Act of 1973, and other applicable federal and state laws that prohibit discrimination on the basis of disability, which mandate that reasonable accommodations be provided for qualified students with disabilities. \smallskip

If you have a disability, and may require some type of instructional and\slash or examination accommodation, please contact me early in the semester. If you have not already done so, you will also need to register with the Office of Services for Students with Disabilities. The office is located at G664 Haven Hall. \smallskip

For more information on disability services at the University of Michigan, go to \url{http://ssd.umich.edu}. 

\newpage
\section*{Course Outline \& Readings}

Each week you should do the readings \textbf{before} the start of the week. I would love each class to consist primarily of discussing the assignments and discussing the readings. to start each class by just going over questions people had about the reading; that way we'll figure out in real time what needs most attention. 

\subsection*{Week 0 - July 1-3}

Introduction to propositional logic\
\textit{The Carnap Book}, Chapter 1

\subsection*{Week 1 - July 6-10}

Direct, Conditional, Nested and Indirect Derivations \\
\textit{The Carnap Book}, Chapters 1, 3, 4, 5, 6, 8

\subsection*{Week 2 - July 11-15}

Truth tables and Truth Trees \\
\textit{The Carnap Book}, Chapter 10 \\
\textit{Odds and Ends}, Chapters 2-3 \\
\textit{Boxes and Diamonds}, Chapters 1 and 2

\subsection*{Week 3 - July 18-22}

Probability and Conditional Probability\\
\textit{Odds and Ends}, Chapters 1, 4-9
 
\subsection*{Week 4 - July 25-29}

Decision Theory and Theories of Probability\\
\textit{Odds and Ends}, Chapters 11-13, 15, 19

\subsection*{Week 5 - August 1-5}

Modal Logics and Frames \\
\textit{Boxes and Diamonds}, Chapters 3-5

\subsection*{Week 6 - August 8-12}

Conditionals \\
\textbf{Reading}: \textit{Boxes and Diamonds}, Chapter 6-7

\subsection*{Week 7 - August 15-16}

Revision

\bigskip

There will be assignments due every Friday from July 10 to August 12.


\end{document}\documentclass[]{article}
