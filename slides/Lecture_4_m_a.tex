% Options for packages loaded elsewhere
\PassOptionsToPackage{unicode}{hyperref}
\PassOptionsToPackage{hyphens}{url}
%
\documentclass[
  ignorenonframetext,
]{beamer}
\usepackage{pgfpages}
\setbeamertemplate{caption}[numbered]
\setbeamertemplate{caption label separator}{: }
\setbeamercolor{caption name}{fg=normal text.fg}
\beamertemplatenavigationsymbolsempty
% Prevent slide breaks in the middle of a paragraph
\widowpenalties 1 10000
\raggedbottom
\setbeamertemplate{part page}{
  \centering
  \begin{beamercolorbox}[sep=16pt,center]{part title}
    \usebeamerfont{part title}\insertpart\par
  \end{beamercolorbox}
}
\setbeamertemplate{section page}{
  \centering
  \begin{beamercolorbox}[sep=12pt,center]{part title}
    \usebeamerfont{section title}\insertsection\par
  \end{beamercolorbox}
}
\setbeamertemplate{subsection page}{
  \centering
  \begin{beamercolorbox}[sep=8pt,center]{part title}
    \usebeamerfont{subsection title}\insertsubsection\par
  \end{beamercolorbox}
}
\AtBeginPart{
  \frame{\partpage}
}
\AtBeginSection{
  \ifbibliography
  \else
    \frame{\sectionpage}
  \fi
}
\AtBeginSubsection{
  \frame{\subsectionpage}
}
\usepackage{lmodern}
\usepackage{amssymb,amsmath}
\usepackage{ifxetex,ifluatex}
\ifnum 0\ifxetex 1\fi\ifluatex 1\fi=0 % if pdftex
  \usepackage[T1]{fontenc}
  \usepackage[utf8]{inputenc}
  \usepackage{textcomp} % provide euro and other symbols
\else % if luatex or xetex
  \usepackage{unicode-math}
  \defaultfontfeatures{Scale=MatchLowercase}
  \defaultfontfeatures[\rmfamily]{Ligatures=TeX,Scale=1}
\fi
% Use upquote if available, for straight quotes in verbatim environments
\IfFileExists{upquote.sty}{\usepackage{upquote}}{}
\IfFileExists{microtype.sty}{% use microtype if available
  \usepackage[]{microtype}
  \UseMicrotypeSet[protrusion]{basicmath} % disable protrusion for tt fonts
}{}
\makeatletter
\@ifundefined{KOMAClassName}{% if non-KOMA class
  \IfFileExists{parskip.sty}{%
    \usepackage{parskip}
  }{% else
    \setlength{\parindent}{0pt}
    \setlength{\parskip}{6pt plus 2pt minus 1pt}}
}{% if KOMA class
  \KOMAoptions{parskip=half}}
\makeatother
\usepackage{xcolor}
\IfFileExists{xurl.sty}{\usepackage{xurl}}{} % add URL line breaks if available
\IfFileExists{bookmark.sty}{\usepackage{bookmark}}{\usepackage{hyperref}}
\hypersetup{
  pdftitle={305 Lecture 36 - Expected Utility},
  pdfauthor={Brian Weatherson},
  hidelinks,
  pdfcreator={LaTeX via pandoc}}
\urlstyle{same} % disable monospaced font for URLs
\newif\ifbibliography
\setlength{\emergencystretch}{3em} % prevent overfull lines
\providecommand{\tightlist}{%
  \setlength{\itemsep}{0pt}\setlength{\parskip}{0pt}}
\setcounter{secnumdepth}{-\maxdimen} % remove section numbering
\let\Tiny=\tiny

 \setbeamertemplate{navigation symbols}{} 

% \usetheme{Madrid}
 \usetheme[numbering=none, progressbar=foot]{metropolis}
 \usecolortheme{wolverine}
 \usepackage{color}
 \usepackage{MnSymbol}
% \usepackage{movie15}

\usepackage{amssymb}% http://ctan.org/pkg/amssymb
\usepackage{pifont}% http://ctan.org/pkg/pifont
\newcommand{\cmark}{\ding{51}}%
\newcommand{\xmark}{\ding{55}}%

\DeclareSymbolFont{symbolsC}{U}{txsyc}{m}{n}
\DeclareMathSymbol{\boxright}{\mathrel}{symbolsC}{128}
\DeclareMathAlphabet{\mathpzc}{OT1}{pzc}{m}{it}


% \usepackage{tikz-qtree}
% \usepackage{markdown}
% \usepackage{prooftrees}
% \forestset{not line numbering, close with = x}
% Allow for easy commas inside trees
\renewcommand{\,}{\text{, }}


\usepackage{tabulary}

\usepackage{open-logic-config}

\setlength{\parskip}{1ex plus 0.5ex minus 0.2ex}

\AtBeginSection[]
{
\begin{frame}
	\Huge{\color{darkblue} \insertsection}
\end{frame}
}

\renewenvironment*{quote}	
	{\list{}{\rightmargin   \leftmargin} \item } 	
	{\endlist }

\definecolor{darkgreen}{rgb}{0,0.7,0}
\definecolor{darkblue}{rgb}{0,0,0.8}

\newcommand{\starttab}{\begin{center}
\vspace{6pt}
\begin{tabular}}

\newcommand{\stoptab}{\end{tabular}
\vspace{6pt}
\end{center}
\noindent}


\newcommand{\sif}{\rightarrow}
\newcommand{\siff}{\leftrightarrow}
\newcommand{\EF}{\end{frame}}


\newcommand{\TreeStart}[1]{
%\end{frame}
\begin{frame}
\begin{center}
\begin{tikzpicture}[scale=#1]
\tikzset{every tree node/.style={align=center,anchor=north}}
%\Tree
}

\newcommand{\TreeEnd}{
\end{tikzpicture}
%\end{center}
}

\newcommand{\DisplayArg}[2]{
\begin{enumerate}
{#1}
\end{enumerate}
\vspace{-6pt}
\hrulefill

%\hspace{14pt} #2
%{\addtolength{\leftskip}{14pt} #2}
\begin{quote}
{\normalfont #2}
\end{quote}
\vspace{12pt}
}

\newenvironment{ProofTree}[1][1]{
\begin{center}
\begin{tikzpicture}[scale=#1]
\tikzset{every tree node/.style={align=center,anchor=south}}
}
{
\end{tikzpicture}
\end{center}
}

\newcommand{\TreeFrame}[2]{
\begin{columns}[c]
\column{0.5\textwidth}
\begin{center}
\begin{prooftree}{}
#1
\end{prooftree}
\end{center}
\column{0.45\textwidth}
%\begin{markdown}
#2
%\end{markdown}
\end{columns}
}

\newcommand{\ScaledTreeFrame}[3]{
\begin{columns}[c]
\column{0.5\textwidth}
\begin{center}
\scalebox{#1}{
\begin{prooftree}{}
#2
\end{prooftree}
}
\end{center}
\column{0.45\textwidth}
%\begin{markdown}
#3
%\end{markdown}
\end{columns}
}

\usepackage[bb=boondox]{mathalfa}
\DeclareMathAlphabet{\mathbx}{U}{BOONDOX-ds}{m}{n}
\SetMathAlphabet{\mathbx}{bold}{U}{BOONDOX-ds}{b}{n}
\DeclareMathAlphabet{\mathbbx} {U}{BOONDOX-ds}{b}{n}

\RequirePackage{bussproofs}
\RequirePackage[tableaux]{prooftrees}

\newenvironment{oltableau}{\center\tableau{}} %wff format={anchor = base west}}}
       {\endtableau\endcenter}
       
\newcommand{\formula}[1]{$#1$}

\usepackage{tabulary}
\usepackage{booktabs}

\def\begincols{\begin{columns}}
\def\begincol{\begin{column}}
\def\endcol{\end{column}}
\def\endcols{\end{columns}}

\usepackage[italic]{mathastext}
\usepackage{nicefrac}

\definecolor{mygreen}{RGB}{0, 100, 0}
\definecolor{mypink2}{RGB}{219, 48, 122}
\definecolor{dodgerblue}{RGB}{30,144,255}

\def\True{\textcolor{dodgerblue}{\text{T}}}
\def\False{\textcolor{red}{\text{F}}}

\title{305 Lecture 36 - Expected Utility}
\author{Brian Weatherson}
\date{July 27, 2020}

\begin{document}
\frame{\titlepage}

\begin{frame}{Plan}
\protect\hypertarget{plan}{}

\begin{itemize}
\tightlist
\item
  Today we're going to talk about the role of probability in decision
  making.
\item
  And to do this, we need to introduce a new concept, \textbf{Expected
  Value}.
\end{itemize}

\end{frame}

\begin{frame}{Associated Reading}
\protect\hypertarget{associated-reading}{}

Odds and Ends, chapter 11; though like with chapter 8 I'm going to take
a different path through the material to the book.

\end{frame}

\begin{frame}{Random Variables}
\protect\hypertarget{random-variables}{}

\begin{itemize}
\tightlist
\item
  A \textbf{random variable} is simply a variable that takes different
  numerical values in different states.
\item
  In other words, it is a function from possibilities to numbers.
\item
  It need not be `random' in any familiar sense.
\item
  The function from possible situations to the value of 2 + 2 in that
  situation is a random variable, albeit a constant one.
\item
  It's just a slightly confusing term for any variable that takes
  different, numerical, values in different situations.
\end{itemize}

\end{frame}

\begin{frame}{Labels}
\protect\hypertarget{labels}{}

\begin{itemize}
\tightlist
\item
  Typically, random variables are denoted by capital letters.
\item
  So we might have a random variable \(X\) whose value is the age of the
  next President of the United States, and his or her inauguration.
\item
  Or we might have a random variable \(Y\) that is the number of
  children you will have in your lifetime.
\item
  Basically any mapping from possibilities to numbers can be a random
  variable.
\end{itemize}

\end{frame}

\begin{frame}{An Example}
\protect\hypertarget{an-example}{}

\begin{itemize}
\tightlist
\item
  You've asked each of your friends who will win the Lakers v Clippers
  game.
\item
  12 said the Lakers will win.
\item
  7 said the Clippers will win. \pause 
\item
  Then we can let \(X\) be a random variable measuring the number of
  your friends who correctly predicted the result of the game.
\end{itemize}

\begin{equation*}
X = 
    \begin{cases}
        12,& \text{if Lakers win} ,\\ 
        7,& \text{if Clippers win} .
    \end{cases}
\end{equation*}

\end{frame}

\begin{frame}{Expected Value}
\protect\hypertarget{expected-value}{}

\begin{itemize}
\tightlist
\item
  Given a random variable \(X\) and a probability function \(Pr\), we
  can work out the \textbf{expected value} of that random variable with
  respect to that probability function.
\item
  Intuitively, the expected value of \(X\) is a weighted average of the
  possible values of \(X\), where the weights are given by the
  probability (according to \(Pr\)) of each value coming about.
\end{itemize}

\end{frame}

\begin{frame}{Calculating Expected Value}
\protect\hypertarget{calculating-expected-value}{}

\begin{itemize}
\tightlist
\item
  More formally, we work out the expected value of \(X\) this way.
\item
  For each possibility, we multiply the value of \(X\) in that case by
  the probability of the possibility obtaining.
\item
  Then we sum the numbers we've got, and the result is the expected
  value of \(X\).
\item
  We'll write the expected value of \(X\) as \(Exp(X)\).
\end{itemize}

\end{frame}

\begin{frame}{Back to the Example}
\protect\hypertarget{back-to-the-example}{}

\begin{itemize}
\tightlist
\item
  So if the probability that the Lakers win is 0.7, and the probability
  that the Clippers win is 0.3, then
\end{itemize}

\begin{align*}
Exp(X) &= 12 \times 0.7 + 7 \times 0.3 \\
 &= 8.4 + 2.1 \\
 &= 10.5
\end{align*}

\end{frame}

\begin{frame}{Notes}
\protect\hypertarget{notes}{}

\begin{enumerate}
\tightlist
\item
  The expected value of \(X\) isn't in any sense the value that we
  expect \(X\) to take. It's more like an average.
\item
  If this kind of situation recurs a lot, you would expect the long run
  average value \(X\) takes to be roundabout the expected value.
\item
  That's a better way of conceptualising what expected values are.
\end{enumerate}

\end{frame}

\begin{frame}{For Next Time}
\protect\hypertarget{for-next-time}{}

\begin{itemize}
\tightlist
\item
  We will look at how to formally model a decision problem.
\end{itemize}

\end{frame}

\end{document}
