% Options for packages loaded elsewhere
\PassOptionsToPackage{unicode}{hyperref}
\PassOptionsToPackage{hyphens}{url}
%
\documentclass[
  ignorenonframetext,
]{beamer}
\usepackage{pgfpages}
\setbeamertemplate{caption}[numbered]
\setbeamertemplate{caption label separator}{: }
\setbeamercolor{caption name}{fg=normal text.fg}
\beamertemplatenavigationsymbolsempty
% Prevent slide breaks in the middle of a paragraph
\widowpenalties 1 10000
\raggedbottom
\setbeamertemplate{part page}{
  \centering
  \begin{beamercolorbox}[sep=16pt,center]{part title}
    \usebeamerfont{part title}\insertpart\par
  \end{beamercolorbox}
}
\setbeamertemplate{section page}{
  \centering
  \begin{beamercolorbox}[sep=12pt,center]{part title}
    \usebeamerfont{section title}\insertsection\par
  \end{beamercolorbox}
}
\setbeamertemplate{subsection page}{
  \centering
  \begin{beamercolorbox}[sep=8pt,center]{part title}
    \usebeamerfont{subsection title}\insertsubsection\par
  \end{beamercolorbox}
}
\AtBeginPart{
  \frame{\partpage}
}
\AtBeginSection{
  \ifbibliography
  \else
    \frame{\sectionpage}
  \fi
}
\AtBeginSubsection{
  \frame{\subsectionpage}
}
\usepackage{lmodern}
\usepackage{amssymb,amsmath}
\usepackage{ifxetex,ifluatex}
\ifnum 0\ifxetex 1\fi\ifluatex 1\fi=0 % if pdftex
  \usepackage[T1]{fontenc}
  \usepackage[utf8]{inputenc}
  \usepackage{textcomp} % provide euro and other symbols
\else % if luatex or xetex
  \usepackage{unicode-math}
  \defaultfontfeatures{Scale=MatchLowercase}
  \defaultfontfeatures[\rmfamily]{Ligatures=TeX,Scale=1}
\fi
% Use upquote if available, for straight quotes in verbatim environments
\IfFileExists{upquote.sty}{\usepackage{upquote}}{}
\IfFileExists{microtype.sty}{% use microtype if available
  \usepackage[]{microtype}
  \UseMicrotypeSet[protrusion]{basicmath} % disable protrusion for tt fonts
}{}
\makeatletter
\@ifundefined{KOMAClassName}{% if non-KOMA class
  \IfFileExists{parskip.sty}{%
    \usepackage{parskip}
  }{% else
    \setlength{\parindent}{0pt}
    \setlength{\parskip}{6pt plus 2pt minus 1pt}}
}{% if KOMA class
  \KOMAoptions{parskip=half}}
\makeatother
\usepackage{xcolor}
\IfFileExists{xurl.sty}{\usepackage{xurl}}{} % add URL line breaks if available
\IfFileExists{bookmark.sty}{\usepackage{bookmark}}{\usepackage{hyperref}}
\hypersetup{
  pdftitle={305 Lecture 21 - Rules for Truth Trees},
  pdfauthor={Brian Weatherson},
  hidelinks,
  pdfcreator={LaTeX via pandoc}}
\urlstyle{same} % disable monospaced font for URLs
\newif\ifbibliography
\setlength{\emergencystretch}{3em} % prevent overfull lines
\providecommand{\tightlist}{%
  \setlength{\itemsep}{0pt}\setlength{\parskip}{0pt}}
\setcounter{secnumdepth}{-\maxdimen} % remove section numbering
\let\Tiny=\tiny

 \setbeamertemplate{navigation symbols}{} 

% \usetheme{Madrid}
 \usetheme[numbering=none, progressbar=foot]{metropolis}
 \usecolortheme{wolverine}
 \usepackage{color}
 \usepackage{MnSymbol}
% \usepackage{movie15}

\usepackage{amssymb}% http://ctan.org/pkg/amssymb
\usepackage{pifont}% http://ctan.org/pkg/pifont
\newcommand{\cmark}{\ding{51}}%
\newcommand{\xmark}{\ding{55}}%

\DeclareSymbolFont{symbolsC}{U}{txsyc}{m}{n}
\DeclareMathSymbol{\boxright}{\mathrel}{symbolsC}{128}
\DeclareMathAlphabet{\mathpzc}{OT1}{pzc}{m}{it}


% \usepackage{tikz-qtree}
% \usepackage{markdown}
% \usepackage{prooftrees}
% \forestset{not line numbering, close with = x}
% Allow for easy commas inside trees
\renewcommand{\,}{\text{, }}


\usepackage{tabulary}

\usepackage{open-logic-config}

\setlength{\parskip}{1ex plus 0.5ex minus 0.2ex}

\AtBeginSection[]
{
\begin{frame}
	\Huge{\color{darkblue} \insertsection}
\end{frame}
}

\renewenvironment*{quote}	
	{\list{}{\rightmargin   \leftmargin} \item } 	
	{\endlist }

\definecolor{darkgreen}{rgb}{0,0.7,0}
\definecolor{darkblue}{rgb}{0,0,0.8}

\newcommand{\starttab}{\begin{center}
\vspace{6pt}
\begin{tabular}}

\newcommand{\stoptab}{\end{tabular}
\vspace{6pt}
\end{center}
\noindent}


\newcommand{\sif}{\rightarrow}
\newcommand{\siff}{\leftrightarrow}
\newcommand{\EF}{\end{frame}}


\newcommand{\TreeStart}[1]{
%\end{frame}
\begin{frame}
\begin{center}
\begin{tikzpicture}[scale=#1]
\tikzset{every tree node/.style={align=center,anchor=north}}
%\Tree
}

\newcommand{\TreeEnd}{
\end{tikzpicture}
%\end{center}
}

\newcommand{\DisplayArg}[2]{
\begin{enumerate}
{#1}
\end{enumerate}
\vspace{-6pt}
\hrulefill

%\hspace{14pt} #2
%{\addtolength{\leftskip}{14pt} #2}
\begin{quote}
{\normalfont #2}
\end{quote}
\vspace{12pt}
}

\newenvironment{ProofTree}[1][1]{
\begin{center}
\begin{tikzpicture}[scale=#1]
\tikzset{every tree node/.style={align=center,anchor=south}}
}
{
\end{tikzpicture}
\end{center}
}

\newcommand{\TreeFrame}[2]{
\begin{columns}[c]
\column{0.5\textwidth}
\begin{center}
\begin{prooftree}{}
#1
\end{prooftree}
\end{center}
\column{0.45\textwidth}
%\begin{markdown}
#2
%\end{markdown}
\end{columns}
}

\newcommand{\ScaledTreeFrame}[3]{
\begin{columns}[c]
\column{0.5\textwidth}
\begin{center}
\scalebox{#1}{
\begin{prooftree}{}
#2
\end{prooftree}
}
\end{center}
\column{0.45\textwidth}
%\begin{markdown}
#3
%\end{markdown}
\end{columns}
}

\usepackage[bb=boondox]{mathalfa}
\DeclareMathAlphabet{\mathbx}{U}{BOONDOX-ds}{m}{n}
\SetMathAlphabet{\mathbx}{bold}{U}{BOONDOX-ds}{b}{n}
\DeclareMathAlphabet{\mathbbx} {U}{BOONDOX-ds}{b}{n}

\RequirePackage{bussproofs}
\RequirePackage[tableaux]{prooftrees}

\newenvironment{oltableau}{\center\tableau{}} %wff format={anchor = base west}}}
       {\endtableau\endcenter}
       
\newcommand{\formula}[1]{$#1$}

\usepackage{tabulary}
\usepackage{booktabs}

\def\begincols{\begin{columns}}
\def\begincol{\begin{column}}
\def\endcol{\end{column}}
\def\endcols{\end{columns}}

\usepackage[italic]{mathastext}
\usepackage{nicefrac}

\definecolor{mygreen}{RGB}{0, 100, 0}
\definecolor{mypink2}{RGB}{219, 48, 122}
\definecolor{dodgerblue}{RGB}{30,144,255}

\def\True{\textcolor{dodgerblue}{\text{T}}}
\def\False{\textcolor{red}{\text{F}}}

\title{305 Lecture 21 - Rules for Truth Trees}
\author{Brian Weatherson}
\date{July 15, 2020}

\begin{document}
\frame{\titlepage}

\begin{frame}{Plan}
\protect\hypertarget{plan}{}

This lecture introduces the rules we use for building up truth trees.

\end{frame}

\begin{frame}{Associated Reading}
\protect\hypertarget{associated-reading}{}

Boxes and Diamonds, sections 2.2-2.3.

\end{frame}

\begin{frame}{What Rules Do}
\protect\hypertarget{what-rules-do}{}

The rules tell you what new lines to write down given the lines you've
already got.

\begin{itemize}
\tightlist
\item
  To some extent they simply have to be memorised.
\item
  But hopefully they are all (except for the rules about
  \(\rightarrow\)) fairly intuitive.
\end{itemize}

\end{frame}

\begin{frame}{Rules for \(\neg\)}
\protect\hypertarget{rules-for-neg}{}

\begin{center}
\AxiomC{\sFmla{\True}{\lnot A}}
\RightLabel{\TRule{\True}{\lnot}}
\UnaryInfC{\sFmla{\False}{A}}
\DisplayProof
\pause
\hspace{10pt}
\AxiomC{\sFmla{\False}{\lnot A}}
\RightLabel{\TRule{\False}{\lnot}}
\UnaryInfC{\sFmla{\True}{A}}
\DisplayProof
\end{center}
\bigskip \pause

That is, if you have a negated sentence, write down the unnegated
sentence with an opposite truth value.

\end{frame}

\begin{frame}{Rules for \(\wedge\)}
\protect\hypertarget{rules-for-wedge}{}

\begin{center}\noindent
\AxiomC{\sFmla{\True}{A \wedge B}}
\RightLabel{\TRule{\True}{\wedge}}
\UnaryInfC{\sFmla{\True}{A}}
\noLine
\UnaryInfC{\sFmla{\True}{B}}
\DisplayProof
\hspace{10pt}
\pause
\AxiomC{\sFmla{\False}{A \wedge B}}
\RightLabel{\TRule{\False}{\wedge}}
\UnaryInfC{$\sFmla{\False}{A} \quad \mid \quad \sFmla{\False}{B}$}
\DisplayProof
\end{center}
\bigskip \pause

\begin{itemize}
\tightlist
\item
  If you have a true conjunction, write down each conjunct.
\item
  If you have a false conjunction, create two branches, one for each way
  it can be false.
\end{itemize}

\end{frame}

\begin{frame}{Rules for \(\vee\)}
\protect\hypertarget{rules-for-vee}{}

\begin{center}
\AxiomC{\sFmla{\True}{A \vee B}}
\RightLabel{\TRule{\True}{\vee}}
\UnaryInfC{$\sFmla{\True}{A} \quad \mid \quad \sFmla{\True}{B}$}
\DisplayProof
\hspace{10pt}
\AxiomC{\sFmla{\False}{A \vee B}}
\RightLabel{\TRule{\False}{\vee}}
\UnaryInfC{\sFmla{\False}{A}}
\noLine
\UnaryInfC{\sFmla{\False}{B}}
\DisplayProof
\end{center}
\bigskip

\begin{itemize}
\tightlist
\item
  If you have a true disjunction, create two branches for the two
  disjuncts.
\item
  If you have a false disjunction, write down that each disjunct is
  false.
\end{itemize}

\end{frame}

\begin{frame}{Rules for \(\rightarrow\)}
\protect\hypertarget{rules-for-rightarrow}{}

\begin{center}
\AxiomC{\sFmla{\True}{A \lif B}}
\RightLabel{\TRule{\True}{\lif}}
\UnaryInfC{$\sFmla{\False}{A} \quad \mid \quad \sFmla{\True}{B}$}
\DisplayProof
\pause
\hspace{10pt}
\AxiomC{\sFmla{\False}{A \lif B}}
\RightLabel{\TRule{\False}{\lif}}
\UnaryInfC{\sFmla{\True}{A}}
\noLine
\UnaryInfC{\sFmla{\False}{B}}
\DisplayProof \pause
\end{center}
\bigskip

\begin{itemize}
\tightlist
\item
  And this is the hard rule.
\item
  When a conditional is true, we create two branches - one for the
  antecedent being false, the other for the consequent being true.
\item
  When a conditional is false, we infer that the antecedent is true and
  the conclusion false.
\end{itemize}

\end{frame}

\begin{frame}{For Next Time}
\protect\hypertarget{for-next-time}{}

We will look at some examples of truth trees.

\end{frame}

\end{document}
