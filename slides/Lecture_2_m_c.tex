% Options for packages loaded elsewhere
\PassOptionsToPackage{unicode}{hyperref}
\PassOptionsToPackage{hyphens}{url}
%
\documentclass[
  ignorenonframetext,
]{beamer}
\usepackage{pgfpages}
\setbeamertemplate{caption}[numbered]
\setbeamertemplate{caption label separator}{: }
\setbeamercolor{caption name}{fg=normal text.fg}
\beamertemplatenavigationsymbolsempty
% Prevent slide breaks in the middle of a paragraph
\widowpenalties 1 10000
\raggedbottom
\setbeamertemplate{part page}{
  \centering
  \begin{beamercolorbox}[sep=16pt,center]{part title}
    \usebeamerfont{part title}\insertpart\par
  \end{beamercolorbox}
}
\setbeamertemplate{section page}{
  \centering
  \begin{beamercolorbox}[sep=12pt,center]{part title}
    \usebeamerfont{section title}\insertsection\par
  \end{beamercolorbox}
}
\setbeamertemplate{subsection page}{
  \centering
  \begin{beamercolorbox}[sep=8pt,center]{part title}
    \usebeamerfont{subsection title}\insertsubsection\par
  \end{beamercolorbox}
}
\AtBeginPart{
  \frame{\partpage}
}
\AtBeginSection{
  \ifbibliography
  \else
    \frame{\sectionpage}
  \fi
}
\AtBeginSubsection{
  \frame{\subsectionpage}
}
\usepackage{lmodern}
\usepackage{amssymb,amsmath}
\usepackage{ifxetex,ifluatex}
\ifnum 0\ifxetex 1\fi\ifluatex 1\fi=0 % if pdftex
  \usepackage[T1]{fontenc}
  \usepackage[utf8]{inputenc}
  \usepackage{textcomp} % provide euro and other symbols
\else % if luatex or xetex
  \usepackage{unicode-math}
  \defaultfontfeatures{Scale=MatchLowercase}
  \defaultfontfeatures[\rmfamily]{Ligatures=TeX,Scale=1}
\fi
% Use upquote if available, for straight quotes in verbatim environments
\IfFileExists{upquote.sty}{\usepackage{upquote}}{}
\IfFileExists{microtype.sty}{% use microtype if available
  \usepackage[]{microtype}
  \UseMicrotypeSet[protrusion]{basicmath} % disable protrusion for tt fonts
}{}
\makeatletter
\@ifundefined{KOMAClassName}{% if non-KOMA class
  \IfFileExists{parskip.sty}{%
    \usepackage{parskip}
  }{% else
    \setlength{\parindent}{0pt}
    \setlength{\parskip}{6pt plus 2pt minus 1pt}}
}{% if KOMA class
  \KOMAoptions{parskip=half}}
\makeatother
\usepackage{xcolor}
\IfFileExists{xurl.sty}{\usepackage{xurl}}{} % add URL line breaks if available
\IfFileExists{bookmark.sty}{\usepackage{bookmark}}{\usepackage{hyperref}}
\hypersetup{
  pdftitle={305 Lecture 17 - Tautologies},
  pdfauthor={Brian Weatherson},
  hidelinks,
  pdfcreator={LaTeX via pandoc}}
\urlstyle{same} % disable monospaced font for URLs
\newif\ifbibliography
\setlength{\emergencystretch}{3em} % prevent overfull lines
\providecommand{\tightlist}{%
  \setlength{\itemsep}{0pt}\setlength{\parskip}{0pt}}
\setcounter{secnumdepth}{-\maxdimen} % remove section numbering
\let\Tiny=\tiny

 \setbeamertemplate{navigation symbols}{} 

% \usetheme{Madrid}
 \usetheme[numbering=none, progressbar=foot]{metropolis}
 \usecolortheme{wolverine}
 \usepackage{color}
 \usepackage{MnSymbol}
% \usepackage{movie15}

\usepackage{amssymb}% http://ctan.org/pkg/amssymb
\usepackage{pifont}% http://ctan.org/pkg/pifont
\newcommand{\cmark}{\ding{51}}%
\newcommand{\xmark}{\ding{55}}%

\DeclareSymbolFont{symbolsC}{U}{txsyc}{m}{n}
\DeclareMathSymbol{\boxright}{\mathrel}{symbolsC}{128}
\DeclareMathAlphabet{\mathpzc}{OT1}{pzc}{m}{it}


% \usepackage{tikz-qtree}
% \usepackage{markdown}
% \usepackage{prooftrees}
% \forestset{not line numbering, close with = x}
% Allow for easy commas inside trees
\renewcommand{\,}{\text{, }}


\usepackage{tabulary}

\usepackage{open-logic-config}

\setlength{\parskip}{1ex plus 0.5ex minus 0.2ex}

\AtBeginSection[]
{
\begin{frame}
	\Huge{\color{darkblue} \insertsection}
\end{frame}
}

\renewenvironment*{quote}	
	{\list{}{\rightmargin   \leftmargin} \item } 	
	{\endlist }

\definecolor{darkgreen}{rgb}{0,0.7,0}
\definecolor{darkblue}{rgb}{0,0,0.8}

\newcommand{\starttab}{\begin{center}
\vspace{6pt}
\begin{tabular}}

\newcommand{\stoptab}{\end{tabular}
\vspace{6pt}
\end{center}
\noindent}


\newcommand{\sif}{\rightarrow}
\newcommand{\siff}{\leftrightarrow}
\newcommand{\EF}{\end{frame}}


\newcommand{\TreeStart}[1]{
%\end{frame}
\begin{frame}
\begin{center}
\begin{tikzpicture}[scale=#1]
\tikzset{every tree node/.style={align=center,anchor=north}}
%\Tree
}

\newcommand{\TreeEnd}{
\end{tikzpicture}
%\end{center}
}

\newcommand{\DisplayArg}[2]{
\begin{enumerate}
{#1}
\end{enumerate}
\vspace{-6pt}
\hrulefill

%\hspace{14pt} #2
%{\addtolength{\leftskip}{14pt} #2}
\begin{quote}
{\normalfont #2}
\end{quote}
\vspace{12pt}
}

\newenvironment{ProofTree}[1][1]{
\begin{center}
\begin{tikzpicture}[scale=#1]
\tikzset{every tree node/.style={align=center,anchor=south}}
}
{
\end{tikzpicture}
\end{center}
}

\newcommand{\TreeFrame}[2]{
\begin{columns}[c]
\column{0.5\textwidth}
\begin{center}
\begin{prooftree}{}
#1
\end{prooftree}
\end{center}
\column{0.45\textwidth}
%\begin{markdown}
#2
%\end{markdown}
\end{columns}
}

\newcommand{\ScaledTreeFrame}[3]{
\begin{columns}[c]
\column{0.5\textwidth}
\begin{center}
\scalebox{#1}{
\begin{prooftree}{}
#2
\end{prooftree}
}
\end{center}
\column{0.45\textwidth}
%\begin{markdown}
#3
%\end{markdown}
\end{columns}
}

\usepackage[bb=boondox]{mathalfa}
\DeclareMathAlphabet{\mathbx}{U}{BOONDOX-ds}{m}{n}
\SetMathAlphabet{\mathbx}{bold}{U}{BOONDOX-ds}{b}{n}
\DeclareMathAlphabet{\mathbbx} {U}{BOONDOX-ds}{b}{n}

\RequirePackage{bussproofs}
\RequirePackage[tableaux]{prooftrees}

\newenvironment{oltableau}{\center\tableau{}} %wff format={anchor = base west}}}
       {\endtableau\endcenter}
       
\newcommand{\formula}[1]{$#1$}

\usepackage{tabulary}
\usepackage{booktabs}

\def\begincols{\begin{columns}}
\def\begincol{\begin{column}}
\def\endcol{\end{column}}
\def\endcols{\end{columns}}

\usepackage[italic]{mathastext}
\usepackage{nicefrac}

\definecolor{mygreen}{RGB}{0, 100, 0}
\definecolor{mypink2}{RGB}{219, 48, 122}
\definecolor{dodgerblue}{RGB}{30,144,255}

\def\True{\textcolor{dodgerblue}{\text{T}}}
\def\False{\textcolor{red}{\text{F}}}

\title{305 Lecture 17 - Tautologies}
\author{Brian Weatherson}
\date{July 13, 2020}

\begin{document}
\frame{\titlepage}

\begin{frame}{Plan}
\protect\hypertarget{plan}{}

This lecture is about how we use truth tables to check for logical
properties.

\end{frame}

\begin{frame}{Associated Reading}
\protect\hypertarget{associated-reading}{}

Carnap Book, Chapter 10, section ``Truth Values, Truth Tables,
Tautologies''.

\end{frame}

\begin{frame}{Tautologies}
\protect\hypertarget{tautologies}{}

We are going to start with a particular kind of sentence, a
\textbf{tautology}.

\end{frame}

\begin{frame}{Definition}
\protect\hypertarget{definition}{}

A tautology is a sentence that gets the value \(\True\) in every row of
its truth table.

\end{frame}

\begin{frame}{Examples of Tautologies}
\protect\hypertarget{examples-of-tautologies}{}

What are some sentences that might fit the bill?

\end{frame}

\begin{frame}{The Law of Excluded Middle}
\protect\hypertarget{the-law-of-excluded-middle}{}

\begin{center}
\begin{tabular}{@{ }c | c@{ }@{ }c@{ }@{ }c@{ }@{ }c@{ }@{ }c@{ }@{ }c}
A &  & A & $\lor$ & $\neg$ & A & \\
\hline 
$\True$ &  & $\True$ & \textcolor{red}{$\True$} & $\False$ & $\True$ & \\
$\False$ &  & $\False$ & \textcolor{red}{$\True$} & $\True$ & $\False$ & \\
\end{tabular}
\end{center}

\end{frame}

\begin{frame}{The Law of Non-Contradiction}
\protect\hypertarget{the-law-of-non-contradiction}{}

\begin{center}
\begin{tabular}{@{ }c | c@{ }@{}c@{}@{ }c@{ }@{ }c@{ }@{ }c@{ }@{ }c@{ }@{}c@{ }}
A & $\neg$ & ( & A & $\wedge$ & $\neg$ & A & )\\
\hline 
$\True$ & \textcolor{red}{$\True$} &  & $\True$ & $\False$ & $\False$ & $\True$ & \\
$\False$ & \textcolor{red}{$\True$} &  & $\False$ & $\False$ & $\True$ & $\False$ & \\
\end{tabular}
\end{center}

\end{frame}

\begin{frame}{Reflexive Conditionals}
\protect\hypertarget{reflexive-conditionals}{}

\begin{center}
\begin{tabular}{@{ }c | c@{ }@{ }c@{ }@{ }c@{ }@{ }c@{ }@{ }c}
A &  & A & $\rightarrow$ & A & \\
\hline 
$\True$ &  & $\True$ & \textcolor{red}{$\True$} & $\True$ & \\
$\False$ &  & $\False$ & \textcolor{red}{$\True$} & $\False$ & \\
\end{tabular}
\end{center}

\end{frame}

\begin{frame}{A Surprising One}
\protect\hypertarget{a-surprising-one}{}

\begin{center}
\begin{tabular}{@{ }c@{ }@{ }c | c@{ }@{}c@{}@{ }c@{ }@{ }c@{ }@{ }c@{ }@{}c@{}@{ }c@{ }@{}c@{}@{ }c@{ }@{ }c@{ }@{ }c@{ }@{}c@{}@{ }c}
A & B &  & ( & A & $\rightarrow$ & B & ) & $\lor$ & ( & B & $\rightarrow$ & A & ) & \\
\hline 
$\True$ & $\True$ &  &  & $\True$ & $\True$ & $\True$ &  & \textcolor{red}{$\True$} &  & $\True$ & $\True$ & $\True$ &  & \\
$\True$ & $\False$ &  &  & $\True$ & $\False$ & $\False$ &  & \textcolor{red}{$\True$} &  & $\False$ & $\True$ & $\True$ &  & \\
$\False$ & $\True$ &  &  & $\False$ & $\True$ & $\True$ &  & \textcolor{red}{$\True$} &  & $\True$ & $\False$ & $\False$ &  & \\
$\False$ & $\False$ &  &  & $\False$ & $\True$ & $\False$ &  & \textcolor{red}{$\True$} &  & $\False$ & $\True$ & $\False$ &  & \\
\end{tabular}
\end{center}

\end{frame}

\begin{frame}{Tautologies and Logical Truth}
\protect\hypertarget{tautologies-and-logical-truth}{}

\begin{itemize}
\tightlist
\item
  All tautologies are logical truths.
\item
  But the converse isn't true - some logical truths are not tautologies.
\item
  E.g., If Brian is necessarily a human, then Brian is a human.
\end{itemize}

\end{frame}

\begin{frame}{Validity}
\protect\hypertarget{validity}{}

We can also use truth tables to check for properties of arguments, and
in particular to check for validity.

\end{frame}

\begin{frame}{Truth Tables and Validity}
\protect\hypertarget{truth-tables-and-validity}{}

\begin{itemize}[<+->]
\tightlist
\item
  An argument is (truth-functionally) valid if (and only if) every line
  on the truth table where all the premises are \(\True\), the
  conclusion is \(\True\) as well.
\item
  Equivalently, an argument is invalid if there is a line where the
  premises are \(\True\) and the conclusion \(\False\), and valid
  otherwise.
\end{itemize}

\end{frame}

\begin{frame}{Example of Invalidity}
\protect\hypertarget{example-of-invalidity}{}

The argument \(A\), therefore \(A \wedge B\) is invalid because of the
second line.

\begin{center}
\bigskip
\begin{tabular}{@{ }c@{ }@{ }c | c | c@{ }@{ }c@{ }@{ }c@{ }@{ }c@{ }@{ }c}
A & B & A &  & A & $\wedge$ & B & \\
\hline 
$\True$ & $\True$ & \textcolor{red}{$\True$} &  & $\True$ & \textcolor{red}{$\True$} & $\True$ & \\
$\True$ & $\False$ & \textcolor{red}{$\True$} &  & $\True$ & \textcolor{red}{$\False$} & $\False$ & \\
$\False$ & $\True$ & \textcolor{red}{$\False$} &  & $\False$ & \textcolor{red}{$\False$} & $\True$ & \\
$\False$ & $\False$ & \textcolor{red}{$\False$} &  & $\False$ & \textcolor{red}{$\False$} & $\False$ & \\
\end{tabular}
\end{center}

\end{frame}

\begin{frame}{Another Invalidity Example}
\protect\hypertarget{another-invalidity-example}{}

Note that there are several lines with \(\True\) premises and
conclusion. But the argument \(A \lif B\), so \(A \lif C\) is invalid
because of line 2.

\begin{center}
\bigskip
\begin{tabular}{@{ }c@{ }@{ }c@{ }@{ }c | c@{ }@{ }c@{ }@{ }c@{ }@{ }c@{ }@{ }c | c@{ }@{ }c@{ }@{ }c@{ }@{ }c@{ }@{ }c}
A & B & C &  & A & $\rightarrow$ & B &  &  & A & $\rightarrow$ & C & \\
\hline 
$\True$ & $\True$ & $\True$ &  & $\True$ & \textcolor{red}{$\True$} & $\True$ &  &  & $\True$ & \textcolor{red}{$\True$} & $\True$ & \\
$\True$ & $\True$ & $\False$ &  & $\True$ & \textcolor{red}{$\True$} & $\True$ &  &  & $\True$ & \textcolor{red}{$\False$} & $\False$ & \\
$\True$ & $\False$ & $\True$ &  & $\True$ & \textcolor{red}{$\False$} & $\False$ &  &  & $\True$ & \textcolor{red}{$\True$} & $\True$ & \\
$\True$ & $\False$ & $\False$ &  & $\True$ & \textcolor{red}{$\False$} & $\False$ &  &  & $\True$ & \textcolor{red}{$\False$} & $\False$ & \\
$\False$ & $\True$ & $\True$ &  & $\False$ & \textcolor{red}{$\True$} & $\True$ &  &  & $\False$ & \textcolor{red}{$\True$} & $\True$ & \\
$\False$ & $\True$ & $\False$ &  & $\False$ & \textcolor{red}{$\True$} & $\True$ &  &  & $\False$ & \textcolor{red}{$\True$} & $\False$ & \\
$\False$ & $\False$ & $\True$ &  & $\False$ & \textcolor{red}{$\True$} & $\False$ &  &  & $\False$ & \textcolor{red}{$\True$} & $\True$ & \\
$\False$ & $\False$ & $\False$ &  & $\False$ & \textcolor{red}{$\True$} & $\False$ &  &  & $\False$ & \textcolor{red}{$\True$} & $\False$ & \\
\end{tabular}
\end{center}

\end{frame}

\begin{frame}{Hypothetical Syllogism}
\protect\hypertarget{hypothetical-syllogism}{}

On the other hand the argument from \(A \lif B\) and \(B \lif C\) to
\(A \lif C\) is valid.

\begin{center}
\bigskip
\begin{tabular}{@{ }c@{ }@{ }c@{ }@{ }c | c@{ }@{ }c@{ }@{ }c@{ }@{ }c@{ }@{ }c | c@{ }@{ }c@{ }@{ }c@{ }@{ }c@{ }@{ }c | c@{ }@{ }c@{ }@{ }c@{ }@{ }c@{ }@{ }c}
A & B & C &  & A & $\rightarrow$ & B &  &  & B & $\rightarrow$ & C &  &  & A & $\rightarrow$ & C & \\
\hline 
$\True$ & $\True$ & $\True$ &  & $\True$ & \textcolor{red}{$\True$} & $\True$ &  &  & $\True$ & \textcolor{red}{$\True$} & $\True$ &  &  & $\True$ & \textcolor{red}{$\True$} & $\True$ & \\
$\True$ & $\True$ & $\False$ &  & $\True$ & \textcolor{red}{$\True$} & $\True$ &  &  & $\True$ & \textcolor{red}{$\False$} & $\False$ &  &  & $\True$ & \textcolor{red}{$\False$} & $\False$ & \\
$\True$ & $\False$ & $\True$ &  & $\True$ & \textcolor{red}{$\False$} & $\False$ &  &  & $\False$ & \textcolor{red}{$\True$} & $\True$ &  &  & $\True$ & \textcolor{red}{$\True$} & $\True$ & \\
$\True$ & $\False$ & $\False$ &  & $\True$ & \textcolor{red}{$\False$} & $\False$ &  &  & $\False$ & \textcolor{red}{$\True$} & $\False$ &  &  & $\True$ & \textcolor{red}{$\False$} & $\False$ & \\
$\False$ & $\True$ & $\True$ &  & $\False$ & \textcolor{red}{$\True$} & $\True$ &  &  & $\True$ & \textcolor{red}{$\True$} & $\True$ &  &  & $\False$ & \textcolor{red}{$\True$} & $\True$ & \\
$\False$ & $\True$ & $\False$ &  & $\False$ & \textcolor{red}{$\True$} & $\True$ &  &  & $\True$ & \textcolor{red}{$\False$} & $\False$ &  &  & $\False$ & \textcolor{red}{$\True$} & $\False$ & \\
$\False$ & $\False$ & $\True$ &  & $\False$ & \textcolor{red}{$\True$} & $\False$ &  &  & $\False$ & \textcolor{red}{$\True$} & $\True$ &  &  & $\False$ & \textcolor{red}{$\True$} & $\True$ & \\
$\False$ & $\False$ & $\False$ &  & $\False$ & \textcolor{red}{$\True$} & $\False$ &  &  & $\False$ & \textcolor{red}{$\True$} & $\False$ &  &  & $\False$ & \textcolor{red}{$\True$} & $\False$ & \\
\end{tabular}
\end{center}

\end{frame}

\begin{frame}{For Next Time}
\protect\hypertarget{for-next-time}{}

We'll talk about how to build more complicated truth tables.

\end{frame}

\end{document}
