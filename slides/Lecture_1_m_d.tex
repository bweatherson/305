% Options for packages loaded elsewhere
\PassOptionsToPackage{unicode}{hyperref}
\PassOptionsToPackage{hyphens}{url}
%
\documentclass[
  ignorenonframetext,
]{beamer}
\usepackage{pgfpages}
\setbeamertemplate{caption}[numbered]
\setbeamertemplate{caption label separator}{: }
\setbeamercolor{caption name}{fg=normal text.fg}
\beamertemplatenavigationsymbolsempty
% Prevent slide breaks in the middle of a paragraph
\widowpenalties 1 10000
\raggedbottom
\setbeamertemplate{part page}{
  \centering
  \begin{beamercolorbox}[sep=16pt,center]{part title}
    \usebeamerfont{part title}\insertpart\par
  \end{beamercolorbox}
}
\setbeamertemplate{section page}{
  \centering
  \begin{beamercolorbox}[sep=12pt,center]{part title}
    \usebeamerfont{section title}\insertsection\par
  \end{beamercolorbox}
}
\setbeamertemplate{subsection page}{
  \centering
  \begin{beamercolorbox}[sep=8pt,center]{part title}
    \usebeamerfont{subsection title}\insertsubsection\par
  \end{beamercolorbox}
}
\AtBeginPart{
  \frame{\partpage}
}
\AtBeginSection{
  \ifbibliography
  \else
    \frame{\sectionpage}
  \fi
}
\AtBeginSubsection{
  \frame{\subsectionpage}
}
\usepackage{lmodern}
\usepackage{amssymb,amsmath}
\usepackage{ifxetex,ifluatex}
\ifnum 0\ifxetex 1\fi\ifluatex 1\fi=0 % if pdftex
  \usepackage[T1]{fontenc}
  \usepackage[utf8]{inputenc}
  \usepackage{textcomp} % provide euro and other symbols
\else % if luatex or xetex
  \usepackage{unicode-math}
  \defaultfontfeatures{Scale=MatchLowercase}
  \defaultfontfeatures[\rmfamily]{Ligatures=TeX,Scale=1}
\fi
% Use upquote if available, for straight quotes in verbatim environments
\IfFileExists{upquote.sty}{\usepackage{upquote}}{}
\IfFileExists{microtype.sty}{% use microtype if available
  \usepackage[]{microtype}
  \UseMicrotypeSet[protrusion]{basicmath} % disable protrusion for tt fonts
}{}
\makeatletter
\@ifundefined{KOMAClassName}{% if non-KOMA class
  \IfFileExists{parskip.sty}{%
    \usepackage{parskip}
  }{% else
    \setlength{\parindent}{0pt}
    \setlength{\parskip}{6pt plus 2pt minus 1pt}}
}{% if KOMA class
  \KOMAoptions{parskip=half}}
\makeatother
\usepackage{xcolor}
\IfFileExists{xurl.sty}{\usepackage{xurl}}{} % add URL line breaks if available
\IfFileExists{bookmark.sty}{\usepackage{bookmark}}{\usepackage{hyperref}}
\hypersetup{
  pdftitle={305 Lecture 07 - Direct Derivations},
  pdfauthor={Brian Weatherson},
  hidelinks,
  pdfcreator={LaTeX via pandoc}}
\urlstyle{same} % disable monospaced font for URLs
\newif\ifbibliography
\setlength{\emergencystretch}{3em} % prevent overfull lines
\providecommand{\tightlist}{%
  \setlength{\itemsep}{0pt}\setlength{\parskip}{0pt}}
\setcounter{secnumdepth}{-\maxdimen} % remove section numbering
\let\Tiny=\tiny

 \setbeamertemplate{navigation symbols}{} 

% \usetheme{Madrid}
 \usetheme[numbering=none, progressbar=foot]{metropolis}
 \usecolortheme{wolverine}
 \usepackage{color}
 \usepackage{MnSymbol}
% \usepackage{movie15}

\usepackage{amssymb}% http://ctan.org/pkg/amssymb
\usepackage{pifont}% http://ctan.org/pkg/pifont
\newcommand{\cmark}{\ding{51}}%
\newcommand{\xmark}{\ding{55}}%

\DeclareSymbolFont{symbolsC}{U}{txsyc}{m}{n}
\DeclareMathSymbol{\boxright}{\mathrel}{symbolsC}{128}
\DeclareMathAlphabet{\mathpzc}{OT1}{pzc}{m}{it}


% \usepackage{tikz-qtree}
% \usepackage{markdown}
% \usepackage{prooftrees}
% \forestset{not line numbering, close with = x}
% Allow for easy commas inside trees
\renewcommand{\,}{\text{, }}


\usepackage{tabulary}

\usepackage{open-logic-config}

\setlength{\parskip}{1ex plus 0.5ex minus 0.2ex}

\AtBeginSection[]
{
\begin{frame}
	\Huge{\color{darkblue} \insertsection}
\end{frame}
}

\renewenvironment*{quote}	
	{\list{}{\rightmargin   \leftmargin} \item } 	
	{\endlist }

\definecolor{darkgreen}{rgb}{0,0.7,0}
\definecolor{darkblue}{rgb}{0,0,0.8}

\newcommand{\starttab}{\begin{center}
\vspace{6pt}
\begin{tabular}}

\newcommand{\stoptab}{\end{tabular}
\vspace{6pt}
\end{center}
\noindent}


\newcommand{\sif}{\rightarrow}
\newcommand{\siff}{\leftrightarrow}
\newcommand{\EF}{\end{frame}}


\newcommand{\TreeStart}[1]{
%\end{frame}
\begin{frame}
\begin{center}
\begin{tikzpicture}[scale=#1]
\tikzset{every tree node/.style={align=center,anchor=north}}
%\Tree
}

\newcommand{\TreeEnd}{
\end{tikzpicture}
%\end{center}
}

\newcommand{\DisplayArg}[2]{
\begin{enumerate}
{#1}
\end{enumerate}
\vspace{-6pt}
\hrulefill

%\hspace{14pt} #2
%{\addtolength{\leftskip}{14pt} #2}
\begin{quote}
{\normalfont #2}
\end{quote}
\vspace{12pt}
}

\newenvironment{ProofTree}[1][1]{
\begin{center}
\begin{tikzpicture}[scale=#1]
\tikzset{every tree node/.style={align=center,anchor=south}}
}
{
\end{tikzpicture}
\end{center}
}

\newcommand{\TreeFrame}[2]{
\begin{columns}[c]
\column{0.5\textwidth}
\begin{center}
\begin{prooftree}{}
#1
\end{prooftree}
\end{center}
\column{0.45\textwidth}
%\begin{markdown}
#2
%\end{markdown}
\end{columns}
}

\newcommand{\ScaledTreeFrame}[3]{
\begin{columns}[c]
\column{0.5\textwidth}
\begin{center}
\scalebox{#1}{
\begin{prooftree}{}
#2
\end{prooftree}
}
\end{center}
\column{0.45\textwidth}
%\begin{markdown}
#3
%\end{markdown}
\end{columns}
}

\usepackage[bb=boondox]{mathalfa}
\DeclareMathAlphabet{\mathbx}{U}{BOONDOX-ds}{m}{n}
\SetMathAlphabet{\mathbx}{bold}{U}{BOONDOX-ds}{b}{n}
\DeclareMathAlphabet{\mathbbx} {U}{BOONDOX-ds}{b}{n}

\RequirePackage{bussproofs}
\RequirePackage[tableaux]{prooftrees}

\newenvironment{oltableau}{\center\tableau{}} %wff format={anchor = base west}}}
       {\endtableau\endcenter}
       
\newcommand{\formula}[1]{$#1$}

\usepackage{tabulary}
\usepackage{booktabs}

\def\begincols{\begin{columns}}
\def\begincol{\begin{column}}
\def\endcol{\end{column}}
\def\endcols{\end{columns}}

\usepackage[italic]{mathastext}
\usepackage{nicefrac}

\definecolor{mygreen}{RGB}{0, 100, 0}
\definecolor{mypink2}{RGB}{219, 48, 122}
\definecolor{dodgerblue}{RGB}{30,144,255}

\def\True{\textcolor{dodgerblue}{\text{T}}}
\def\False{\textcolor{red}{\text{F}}}

\title{305 Lecture 07 - Direct Derivations}
\author{Brian Weatherson}
\date{July 8, 2020}

\begin{document}
\frame{\titlepage}

\begin{frame}{Plan}
\protect\hypertarget{plan}{}

To introduce the basic idea of a derivation, and start with some simple
examples.

\end{frame}

\begin{frame}{Associated Reading}
\protect\hypertarget{associated-reading}{}

Carnap book, chapter 3.

\end{frame}

\begin{frame}{Basic Idea}
\protect\hypertarget{basic-idea}{}

A derivation is a series of steps that get you from the premises to the
conclusion, with every step falling into one of a small number of
approved kinds of transition.

\end{frame}

\begin{frame}{Justification}
\protect\hypertarget{justification}{}

The big thought is that no step could take you from truth to falsity (or
to non-truth).

\begin{itemize}
\tightlist
\item
  So you can string as many of these steps together as you like, and it
  will never take you from truth to falsity.
\item
  And to justify the procedure, you just need to justify the various
  kinds that are allowed.
\end{itemize}

\end{frame}

\begin{frame}{Example}
\protect\hypertarget{example}{}

\DisplayArg{ \item $P$ \item $\neg \neg P \rightarrow \neg \neg Q$ } { $Q$ }

\end{frame}

\begin{frame}{Intuitive Argument}
\protect\hypertarget{intuitive-argument}{}

\begin{itemize}[<+->]
\tightlist
\item
  Start assuming that the two premises are true.
\item
  If \(P\) is true, then \(\neg \neg P\) is true.
\item
  So \(\neg \neg P\) is true.
\item
  If \(\neg \neg P\) and \(\neg \neg P \rightarrow \neg \neg Q\) are
  true, then \(\neg \neg Q\) is true.
\item
  So \(\neg \neg Q\) is true.
\item
  And that implies \(Q\) is true, as required.
\end{itemize}

\end{frame}

\begin{frame}[fragile]{Formal Argument in Carnap}
\protect\hypertarget{formal-argument-in-carnap}{}

\begin{verbatim}
1. Show: Q
2.     P          :PR
3.     ~~P -> ~~Q :PR
4.     ~~P        :DNI 2
5.     ~~Q        :MP 4, 3
6.     Q          :DNE 5
7. :DD 6
\end{verbatim}

\end{frame}

\begin{frame}[fragile]{Carnap}
\protect\hypertarget{carnap}{}

\begincols
\begincol{.48\textwidth}

\begin{verbatim}
1. Show: Q
2.     P          :PR
3.     ~~P -> ~~Q :PR
4.     ~~P        :DNI 2
5.     ~~Q        :MP 4, 3
6.     Q          :DNE 5
7. :DD 6
\end{verbatim}

\endcol
\begincol{.48\textwidth}

\begin{itemize}
\tightlist
\item
  A lot of what I'm going to say over the next few slides is about
  \textbf{Carnap}, not about logic in general.
\end{itemize}

\endcol
\endcols

\end{frame}

\begin{frame}[fragile]{Natural Deduction}
\protect\hypertarget{natural-deduction}{}

\begincols
\begincol{.48\textwidth}

\begin{verbatim}
1. Show: Q
2.     P          :PR
3.     ~~P -> ~~Q :PR
4.     ~~P        :DNI 2
5.     ~~Q        :MP 4, 3
6.     Q          :DNE 5
7. :DD 6
\end{verbatim}

\endcol
\begincol{.48\textwidth}

\begin{itemize}
\tightlist
\item
  This is a version of what is known as a \textbf{natural deduction}
  proof system.
\item
  It is somewhat non-standard, but that's not to say any one way is
  standard.
\end{itemize}

\endcol
\endcols

\end{frame}

\begin{frame}[fragile]{Natural Deduction}
\protect\hypertarget{natural-deduction-1}{}

\begincols
\begincol{.48\textwidth}

\begin{verbatim}
1. Show: Q
2.     P          :PR
3.     ~~P -> ~~Q :PR
4.     ~~P        :DNI 2
5.     ~~Q        :MP 4, 3
6.     Q          :DNE 5
7. :DD 6
\end{verbatim}

\endcol
\begincol{.48\textwidth}

\begin{itemize}
\tightlist
\item
  What is common to all natural deduction systems is that when you read
  the steps, they read like a (pedantic version of) ordinary language
  reasoning.
\end{itemize}

\endcol
\endcols

\end{frame}

\begin{frame}[fragile]{Starting and Ending}
\protect\hypertarget{starting-and-ending}{}

\begincols
\begincol{.48\textwidth}

\begin{verbatim}
1. Show: Q
2.     P          :PR
3.     ~~P -> ~~Q :PR
4.     ~~P        :DNI 2
5.     ~~Q        :MP 4, 3
6.     Q          :DNE 5
7. :DD 6
\end{verbatim}

\endcol
\begincol{.48\textwidth}

\begin{itemize}
\tightlist
\item
  The most idiosyncratic feature of Carnap is the first and last line of
  the derivation.
\end{itemize}

\endcol
\endcols

\end{frame}

\begin{frame}[fragile]{Starting}
\protect\hypertarget{starting}{}

\begincols
\begincol{.48\textwidth}

\begin{verbatim}
1. Show: Q
2.     P          :PR
3.     ~~P -> ~~Q :PR
4.     ~~P        :DNI 2
5.     ~~Q        :MP 4, 3
6.     Q          :DNE 5
7. :DD 6
\end{verbatim}

\endcol
\begincol{.48\textwidth}

\begin{itemize}
\tightlist
\item
  In Carnap, you have to start a proof by announcing where you are
  headed.
\end{itemize}

\endcol
\endcols

\end{frame}

\begin{frame}[fragile]{Ending}
\protect\hypertarget{ending}{}

\begincols
\begincol{.48\textwidth}

\begin{verbatim}
1. Show: Q
2.     P          :PR
3.     ~~P -> ~~Q :PR
4.     ~~P        :DNI 2
5.     ~~Q        :MP 4, 3
6.     Q          :DNE 5
7. :DD 6
\end{verbatim}

\endcol
\begincol{.48\textwidth}

\begin{itemize}
\tightlist
\item
  And you end the proof by saying which line it is that the conclusion
  is reached.
\end{itemize}

\endcol
\endcols

\end{frame}

\begin{frame}[fragile]{Starting and Ending}
\protect\hypertarget{starting-and-ending-1}{}

\begincols
\begincol{.48\textwidth}

\begin{verbatim}
1. Show: Q
2.     P          :PR
3.     ~~P -> ~~Q :PR
4.     ~~P        :DNI 2
5.     ~~Q        :MP 4, 3
6.     Q          :DNE 5
7. :DD 6
\end{verbatim}

\endcol
\begincol{.48\textwidth}

\begin{itemize}
\tightlist
\item
  Note that these are the only two lines that are not indented.
\item
  Proof systems (Carnap included) are visual, graphic systems, and
  vertical and horizontal arrangements tend to have meaning.
\end{itemize}

\endcol
\endcols

\end{frame}

\begin{frame}[fragile]{Starting and Ending}
\protect\hypertarget{starting-and-ending-2}{}

\begincols
\begincol{.48\textwidth}

\begin{verbatim}
1. Show: Q
2.     P          :PR
3.     ~~P -> ~~Q :PR
4.     ~~P        :DNI 2
5.     ~~Q        :MP 4, 3
6.     Q          :DNE 5
7. :DD 6
\end{verbatim}

\endcol
\begincol{.48\textwidth}

\begin{itemize}
\tightlist
\item
  They are also the only lines here that do not have a justification.
\item
  Those abbreviations and numbers to the right of the other lines are
  justifications - you don't include them on the start or the finish.
\end{itemize}

\endcol
\endcols

\end{frame}

\begin{frame}[fragile]{Ending}
\protect\hypertarget{ending-1}{}

\begincols
\begincol{.48\textwidth}

\begin{verbatim}
1. Show: Q
2.     P          :PR
3.     ~~P -> ~~Q :PR
4.     ~~P        :DNI 2
5.     ~~Q        :MP 4, 3
6.     Q          :DNE 5
7. :DD 6
\end{verbatim}

\endcol
\begincol{.48\textwidth}

\begin{itemize}
\tightlist
\item
  The `DD' at the end is to indicate this is a \textbf{direct}
  derivation.
\item
  We'll get to the contrast with indirect derivations presently.
\end{itemize}

\endcol
\endcols

\end{frame}

\begin{frame}[fragile]{Premises}
\protect\hypertarget{premises}{}

\begincols
\begincol{.48\textwidth}

\begin{verbatim}
1. Show: Q
2.     P          :PR
3.     ~~P -> ~~Q :PR
4.     ~~P        :DNI 2
5.     ~~Q        :MP 4, 3
6.     Q          :DNE 5
7. :DD 6
\end{verbatim}

\endcol
\begincol{.48\textwidth}

After the introductory line, the first lines are the premises - if they
exist.

\endcol
\endcols

\end{frame}

\begin{frame}[fragile]{Premises}
\protect\hypertarget{premises-1}{}

\begincols
\begincol{.48\textwidth}

\begin{verbatim}
1. Show: Q
2.     P          :PR
3.     ~~P -> ~~Q :PR
4.     ~~P        :DNI 2
5.     ~~Q        :MP 4, 3
6.     Q          :DNE 5
7. :DD 6
\end{verbatim}

\endcol
\begincol{.48\textwidth}

The premises need to be noted - that's what the `PR' is for - but they
are not derived.

\endcol
\endcols

\end{frame}

\begin{frame}[fragile]{Premises}
\protect\hypertarget{premises-2}{}

\begincols
\begincol{.48\textwidth}

\begin{verbatim}
1. Show: Q
2.     P          :PR
3.     ~~P -> ~~Q :PR
4.     ~~P        :DNI 2
5.     ~~Q        :MP 4, 3
6.     Q          :DNE 5
7. :DD 6
\end{verbatim}

\endcol
\begincol{.48\textwidth}

Your justification for writing them is that they are the beginning of
what you are trying to prove.

\endcol
\endcols

\end{frame}

\begin{frame}[fragile]{Premises}
\protect\hypertarget{premises-3}{}

\begincols
\begincol{.48\textwidth}

\begin{verbatim}
1. Show: Q
2.     P          :PR
3.     ~~P -> ~~Q :PR
4.     ~~P        :DNI 2
5.     ~~Q        :MP 4, 3
6.     Q          :DNE 5
7. :DD 6
\end{verbatim}

\endcol
\begincol{.48\textwidth}

So they don't get line numbers afterwards.

\endcol
\endcols

\end{frame}

\begin{frame}[fragile]{Derived Lines}
\protect\hypertarget{derived-lines}{}

\begincols
\begincol{.48\textwidth}

\begin{verbatim}
1. Show: Q
2.     P          :PR
3.     ~~P -> ~~Q :PR
4.     ~~P        :DNI 2
5.     ~~Q        :MP 4, 3
6.     Q          :DNE 5
7. :DD 6
\end{verbatim}

\endcol
\begincol{.48\textwidth}

\begin{itemize}
\tightlist
\item
  From now on, every line will be derived from previous lines.
\item
  And the justification for it will be a rule, plus some line or lines.
\end{itemize}

\endcol
\endcols

\end{frame}

\begin{frame}[fragile]{Derived Lines}
\protect\hypertarget{derived-lines-1}{}

\begincols
\begincol{.48\textwidth}

\begin{verbatim}
1. Show: Q
2.     P          :PR
3.     ~~P -> ~~Q :PR
4.     ~~P        :DNI 2
5.     ~~Q        :MP 4, 3
6.     Q          :DNE 5
7. :DD 6
\end{verbatim}

\endcol
\begincol{.48\textwidth}

In Carnap the premises and derived lines are indented.

\begin{itemize}
\tightlist
\item
  The indenting is \textbf{four spaces}. For reasons I don't understand,
  a tab character here won't work.
\end{itemize}

\endcol
\endcols

\end{frame}

\begin{frame}[fragile]{Double Negation Introduction}
\protect\hypertarget{double-negation-introduction}{}

\begincols
\begincol{.48\textwidth}

\begin{verbatim}
1. Show: Q
2.     P          :PR
3.     ~~P -> ~~Q :PR
4.     ~~P        :DNI 2
5.     ~~Q        :MP 4, 3
6.     Q          :DNE 5
7. :DD 6
\end{verbatim}

\endcol
\begincol{.48\textwidth}

If \(\varphi\) is a line, then you can add \(\neg \neg \varphi\) as a
new line.

\endcol
\endcols

\end{frame}

\begin{frame}[fragile]{Double Negation Introduction}
\protect\hypertarget{double-negation-introduction-1}{}

\begincols
\begincol{.48\textwidth}

\begin{verbatim}
1. Show: Q
2.     P          :PR
3.     ~~P -> ~~Q :PR
4.     ~~P        :DNI 2
5.     ~~Q        :MP 4, 3
6.     Q          :DNE 5
7. :DD 6
\end{verbatim}

\endcol
\begincol{.48\textwidth}

The rule that you are using is abbreviated to `DNI', and you have to
justify this by citing the line where \(\varphi\) appears.

\endcol
\endcols

\end{frame}

\begin{frame}[fragile]{Double Negation Introduction}
\protect\hypertarget{double-negation-introduction-2}{}

\begincols
\begincol{.48\textwidth}

\begin{verbatim}
1. Show: ~~~~P
2.     P          :PR
3.     ~~P        :DNI 2
4.     ~~~~P      :DNI 3
5. :DD 4
\end{verbatim}

\endcol
\begincol{.48\textwidth}

This isn't specific to DNI, but note that for any rule, the input lines
can be either a premise or a derived line.

\begin{itemize}
\tightlist
\item
  The rules do not distinguish between premises and derived lines.
\end{itemize}

\endcol
\endcols

\end{frame}

\begin{frame}{Rules}
\protect\hypertarget{rules}{}

A rule says that given sentences of some form, another particular
sentence can be written.

\end{frame}

\begin{frame}{Rules}
\protect\hypertarget{rules-1}{}

To apply the rule correctly, you have to do 3 things

\begin{enumerate}
\tightlist
\item
  The sentence has to be the right one given the constraints of the
  rule.
\item
  You have to write down (immediately after a colon) the abbreviation
  for the rule.
\item
  You have to write down the line, or lines, that provide the inputs.
\end{enumerate}

\end{frame}

\begin{frame}[fragile]{Double Negation Introduction}
\protect\hypertarget{double-negation-introduction-3}{}

\begincols
\begincol{.48\textwidth}

\begin{verbatim}
1. Show: Q
2.     P          :PR
3.     ~~P -> ~~Q :PR
4.     ~~P        :DNI 2
5.     ~~Q        :MP 4, 3
6.     Q          :DNE 5
7. :DD 6
\end{verbatim}

\endcol
\begincol{.48\textwidth}

Line 4 is allowed because you can add \(\neg \neg\) to any line by the
rule Double Negation Introduction.

\endcol
\endcols

\end{frame}

\begin{frame}[fragile]{Double Negation Introduction}
\protect\hypertarget{double-negation-introduction-4}{}

\begincols
\begincol{.48\textwidth}

\begin{verbatim}
1. Show: Q
2.     P          :PR
3.     ~~P -> ~~Q :PR
4.     ~~P        :DNI 2
5.     ~~Q        :MP 4, 3
6.     Q          :DNE 5
7. :DD 6
\end{verbatim}

\endcol
\begincol{.48\textwidth}

The abbreviation for Double Negation Introduction is DNI - so that's
what we write.

\endcol
\endcols

\end{frame}

\begin{frame}[fragile]{Double Negation Introduction}
\protect\hypertarget{double-negation-introduction-5}{}

\begincols
\begincol{.48\textwidth}

\begin{verbatim}
1. Show: Q
2.     P          :PR
3.     ~~P -> ~~Q :PR
4.     ~~P        :DNI 2
5.     ~~Q        :MP 4, 3
6.     Q          :DNE 5
7. :DD 6
\end{verbatim}

\endcol
\begincol{.48\textwidth}

And the input, the line we are adding \(\neg \neg\) to, is line 2, so we
write `2'.

\endcol
\endcols

\end{frame}

\begin{frame}[fragile]{A Trap}
\protect\hypertarget{a-trap}{}

This is not a good proof - why not?

\begin{verbatim}
1. Show: ~~P -> Q
2.     P -> Q     :PR
3.     ~~P -> Q   :DNI 2
4. :DD 3
\end{verbatim}

\end{frame}

\begin{frame}{A Trap}
\protect\hypertarget{a-trap-1}{}

You have to add the negations to \textbf{the whole sentence}.

\begin{itemize}
\tightlist
\item
  So the correct output here is \(\neg \neg (P \rightarrow Q)\)
\end{itemize}

\end{frame}

\begin{frame}[fragile]{Modus Ponens}
\protect\hypertarget{modus-ponens}{}

\begincols
\begincol{.48\textwidth}

\begin{verbatim}
1. Show: Q
2.     P          :PR
3.     ~~P -> ~~Q :PR
4.     ~~P        :DNI 2
5.     ~~Q        :MP 4, 3
6.     Q          :DNE 5
7. :DD 6
\end{verbatim}

\endcol
\begincol{.48\textwidth}

\begin{itemize}
\tightlist
\item
  The rule at line 5 is the most important in this part of the course.
\item
  It even gets a fancy Latin name.
\end{itemize}

\endcol
\endcols

\end{frame}

\begin{frame}{Modus Ponens}
\protect\hypertarget{modus-ponens-1}{}

Given inputs \(\varphi \rightarrow \psi\) and \(\varphi\), infer
\(\psi\)

\end{frame}

\begin{frame}{Modus Ponens}
\protect\hypertarget{modus-ponens-2}{}

\begin{itemize}
\tightlist
\item
  The abbreviation is MP.
\item
  The line numbers are the lines where \(\varphi \rightarrow \psi\) and
  \(\varphi\) appear.
\end{itemize}

\end{frame}

\begin{frame}{Line Numbers}
\protect\hypertarget{line-numbers}{}

\begin{itemize}
\tightlist
\item
  There is a detail that some people get confused by at this point.
\item
  The line numbers are the lines where the immediate inputs to the rule
  come from.
\item
  They don't list all the justifications for those lines.
\item
  So we list line 4, because it is where \(\neg \neg P\) is, but not
  line 2, from where we derived line 2
\item
  At every stage, we are just looking at whether that immediate step is
  ok.
\end{itemize}

\end{frame}

\begin{frame}{A Trap}
\protect\hypertarget{a-trap-2}{}

\begin{itemize}
\tightlist
\item
  As with DNI, it is important to apply the rule only to whole
  sentences.
\item
  The sentence \(\varphi \rightarrow \psi\) has to have \(\rightarrow\)
  as its \textbf{main connective}.
\end{itemize}

\end{frame}

\begin{frame}[fragile]{Modus Ponens}
\protect\hypertarget{modus-ponens-3}{}

\begincols
\begincol{.5\textwidth}

This is OK.

\begin{verbatim}
1. Show: Q \/ R
2.     P -> (Q \/ R) :PR
3.     P             :PR
4.     Q \/ R        :MP 2, 3
5. :DD 4
\end{verbatim}

\endcol
\begincol{.5\textwidth}

This is \textbf{not} OK.

\begin{verbatim}
1. Show: Q \/ R
2.     (P -> Q) \/ R :PR
3.     P             :PR
4.     Q \/ R        :MP 2, 3
5. :DD 4
\end{verbatim}

\endcol
\endcols

\end{frame}

\begin{frame}{Modus Tollens}
\protect\hypertarget{modus-tollens}{}

There is another rule that I haven't included in the example proof -
modus tollens.

\begin{itemize}
\tightlist
\item
  It takes as input a line saying \(\varphi \rightarrow \psi\), and a
  line saying \(\neg \psi\).
\item
  And it outputs a line saying \(\neg \varphi\).
\end{itemize}

\end{frame}

\begin{frame}{Differences between MP and MT}
\protect\hypertarget{differences-between-mp-and-mt}{}

Different input

\begin{itemize}
\tightlist
\item
  In MP, the input is the left hand side, the \textbf{antecedent} of the
  conditional.
\item
  In MT, the input is the \textbf{negation} of the \textbf{right hand
  side}, or \textbf{consequent} of the conditional.
\end{itemize}

Different output

\begin{itemize}
\tightlist
\item
  In MP, the output is the right hand side, the \textbf{consequent} of
  the conditional.
\item
  In MT, the output is the \textbf{negation} of the \textbf{left hand
  side} of the conditional.
\end{itemize}

\end{frame}

\begin{frame}{Double Negation Elimination}
\protect\hypertarget{double-negation-elimination}{}

\begin{itemize}
\tightlist
\item
  This rule takes as input a sentence of the form \(\neg \neg \varphi\).
\item
  And it returns as output the sentence \(\varphi\).
\end{itemize}

\end{frame}

\begin{frame}[fragile]{Double Negation Elimination}
\protect\hypertarget{double-negation-elimination-1}{}

\begincols
\begincol{.48\textwidth}

\begin{verbatim}
1. Show: Q
2.     P          :PR
3.     ~~P -> ~~Q :PR
4.     ~~P        :DNI 2
5.     ~~Q        :MP 4, 3
6.     Q          :DNE 5
7. :DD 6
\end{verbatim}

\endcol
\begincol{.48\textwidth}

\begin{itemize}
\tightlist
\item
  The abbreviation is DNE.
\item
  And because there is only one input, there is only one line cited.
\end{itemize}

\endcol
\endcols

\end{frame}

\begin{frame}[fragile]{That's All!}
\protect\hypertarget{thats-all}{}

\begincols
\begincol{.48\textwidth}

\begin{verbatim}
1. Show: Q
2.     P          :PR
3.     ~~P -> ~~Q :PR
4.     ~~P        :DNI 2
5.     ~~Q        :MP 4, 3
6.     Q          :DNE 5
7. :DD 6
\end{verbatim}

\endcol
\begincol{.48\textwidth}

Since the line matches what was to be shown, we have a complete `direct
derivation'.

\endcol
\endcols

\end{frame}

\begin{frame}{Four Rules}
\protect\hypertarget{four-rules}{}

\begin{description}
\tightlist
\item[Modus Ponens (MP)]
From \(\varphi \rightarrow \psi\) and \(\varphi\), infer \(\psi\)
\item[Modus Tollens (MT)]
From \(\varphi \rightarrow \psi\) and \(\neg \psi\), infer
\(\neg \varphi\)
\item[Double Negation Introduction (DNI)]
From \(\varphi\), infer \(\neg \neg \varphi\)
\item[Double Negation Elimination (DNE)]
From \(\neg \neg \varphi\), infer \(\varphi\)
\end{description}

\end{frame}

\begin{frame}{Restrictions and Things to Remember}
\protect\hypertarget{restrictions-and-things-to-remember}{}

\begin{itemize}
\tightlist
\item
  Apply the negations in DNI to the whole sentence.
\item
  Make sure the arrow is the main connective for MP and MT
\item
  Cite the lines where the `from' sentences appear in the proof.
\end{itemize}

\end{frame}

\begin{frame}{Carnap is fussy about spacing}
\protect\hypertarget{carnap-is-fussy-about-spacing}{}

\begin{itemize}
\tightlist
\item
  Four spaces for the indented sentences.
\item
  No space ever after a colon.
\item
  One space after the abbreviation for the rule.
\item
  These are not part of `logic' in any sense - they are rules for this
  particular computer program.
\end{itemize}

\end{frame}

\end{document}
