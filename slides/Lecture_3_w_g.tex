% Options for packages loaded elsewhere
\PassOptionsToPackage{unicode}{hyperref}
\PassOptionsToPackage{hyphens}{url}
%
\documentclass[
  ignorenonframetext,
]{beamer}
\usepackage{pgfpages}
\setbeamertemplate{caption}[numbered]
\setbeamertemplate{caption label separator}{: }
\setbeamercolor{caption name}{fg=normal text.fg}
\beamertemplatenavigationsymbolsempty
% Prevent slide breaks in the middle of a paragraph
\widowpenalties 1 10000
\raggedbottom
\setbeamertemplate{part page}{
  \centering
  \begin{beamercolorbox}[sep=16pt,center]{part title}
    \usebeamerfont{part title}\insertpart\par
  \end{beamercolorbox}
}
\setbeamertemplate{section page}{
  \centering
  \begin{beamercolorbox}[sep=12pt,center]{part title}
    \usebeamerfont{section title}\insertsection\par
  \end{beamercolorbox}
}
\setbeamertemplate{subsection page}{
  \centering
  \begin{beamercolorbox}[sep=8pt,center]{part title}
    \usebeamerfont{subsection title}\insertsubsection\par
  \end{beamercolorbox}
}
\AtBeginPart{
  \frame{\partpage}
}
\AtBeginSection{
  \ifbibliography
  \else
    \frame{\sectionpage}
  \fi
}
\AtBeginSubsection{
  \frame{\subsectionpage}
}
\usepackage{lmodern}
\usepackage{amssymb,amsmath}
\usepackage{ifxetex,ifluatex}
\ifnum 0\ifxetex 1\fi\ifluatex 1\fi=0 % if pdftex
  \usepackage[T1]{fontenc}
  \usepackage[utf8]{inputenc}
  \usepackage{textcomp} % provide euro and other symbols
\else % if luatex or xetex
  \usepackage{unicode-math}
  \defaultfontfeatures{Scale=MatchLowercase}
  \defaultfontfeatures[\rmfamily]{Ligatures=TeX,Scale=1}
\fi
% Use upquote if available, for straight quotes in verbatim environments
\IfFileExists{upquote.sty}{\usepackage{upquote}}{}
\IfFileExists{microtype.sty}{% use microtype if available
  \usepackage[]{microtype}
  \UseMicrotypeSet[protrusion]{basicmath} % disable protrusion for tt fonts
}{}
\makeatletter
\@ifundefined{KOMAClassName}{% if non-KOMA class
  \IfFileExists{parskip.sty}{%
    \usepackage{parskip}
  }{% else
    \setlength{\parindent}{0pt}
    \setlength{\parskip}{6pt plus 2pt minus 1pt}}
}{% if KOMA class
  \KOMAoptions{parskip=half}}
\makeatother
\usepackage{xcolor}
\IfFileExists{xurl.sty}{\usepackage{xurl}}{} % add URL line breaks if available
\IfFileExists{bookmark.sty}{\usepackage{bookmark}}{\usepackage{hyperref}}
\hypersetup{
  pdftitle={305 Lecture 35 - Updating on Multiple Data Points},
  pdfauthor={Brian Weatherson},
  hidelinks,
  pdfcreator={LaTeX via pandoc}}
\urlstyle{same} % disable monospaced font for URLs
\newif\ifbibliography
\usepackage{longtable,booktabs}
\usepackage{caption}
% Make caption package work with longtable
\makeatletter
\def\fnum@table{\tablename~\thetable}
\makeatother
\setlength{\emergencystretch}{3em} % prevent overfull lines
\providecommand{\tightlist}{%
  \setlength{\itemsep}{0pt}\setlength{\parskip}{0pt}}
\setcounter{secnumdepth}{-\maxdimen} % remove section numbering
\let\Tiny=\tiny

 \setbeamertemplate{navigation symbols}{} 

% \usetheme{Madrid}
 \usetheme[numbering=none, progressbar=foot]{metropolis}
 \usecolortheme{wolverine}
 \usepackage{color}
 \usepackage{MnSymbol}
% \usepackage{movie15}

\usepackage{amssymb}% http://ctan.org/pkg/amssymb
\usepackage{pifont}% http://ctan.org/pkg/pifont
\newcommand{\cmark}{\ding{51}}%
\newcommand{\xmark}{\ding{55}}%

\DeclareSymbolFont{symbolsC}{U}{txsyc}{m}{n}
\DeclareMathSymbol{\boxright}{\mathrel}{symbolsC}{128}
\DeclareMathAlphabet{\mathpzc}{OT1}{pzc}{m}{it}


% \usepackage{tikz-qtree}
% \usepackage{markdown}
% \usepackage{prooftrees}
% \forestset{not line numbering, close with = x}
% Allow for easy commas inside trees
\renewcommand{\,}{\text{, }}


\usepackage{tabulary}

\usepackage{open-logic-config}

\setlength{\parskip}{1ex plus 0.5ex minus 0.2ex}

\AtBeginSection[]
{
\begin{frame}
	\Huge{\color{darkblue} \insertsection}
\end{frame}
}

\renewenvironment*{quote}	
	{\list{}{\rightmargin   \leftmargin} \item } 	
	{\endlist }

\definecolor{darkgreen}{rgb}{0,0.7,0}
\definecolor{darkblue}{rgb}{0,0,0.8}

\newcommand{\starttab}{\begin{center}
\vspace{6pt}
\begin{tabular}}

\newcommand{\stoptab}{\end{tabular}
\vspace{6pt}
\end{center}
\noindent}


\newcommand{\sif}{\rightarrow}
\newcommand{\siff}{\leftrightarrow}
\newcommand{\EF}{\end{frame}}


\newcommand{\TreeStart}[1]{
%\end{frame}
\begin{frame}
\begin{center}
\begin{tikzpicture}[scale=#1]
\tikzset{every tree node/.style={align=center,anchor=north}}
%\Tree
}

\newcommand{\TreeEnd}{
\end{tikzpicture}
%\end{center}
}

\newcommand{\DisplayArg}[2]{
\begin{enumerate}
{#1}
\end{enumerate}
\vspace{-6pt}
\hrulefill

%\hspace{14pt} #2
%{\addtolength{\leftskip}{14pt} #2}
\begin{quote}
{\normalfont #2}
\end{quote}
\vspace{12pt}
}

\newenvironment{ProofTree}[1][1]{
\begin{center}
\begin{tikzpicture}[scale=#1]
\tikzset{every tree node/.style={align=center,anchor=south}}
}
{
\end{tikzpicture}
\end{center}
}

\newcommand{\TreeFrame}[2]{
\begin{columns}[c]
\column{0.5\textwidth}
\begin{center}
\begin{prooftree}{}
#1
\end{prooftree}
\end{center}
\column{0.45\textwidth}
%\begin{markdown}
#2
%\end{markdown}
\end{columns}
}

\newcommand{\ScaledTreeFrame}[3]{
\begin{columns}[c]
\column{0.5\textwidth}
\begin{center}
\scalebox{#1}{
\begin{prooftree}{}
#2
\end{prooftree}
}
\end{center}
\column{0.45\textwidth}
%\begin{markdown}
#3
%\end{markdown}
\end{columns}
}

\usepackage[bb=boondox]{mathalfa}
\DeclareMathAlphabet{\mathbx}{U}{BOONDOX-ds}{m}{n}
\SetMathAlphabet{\mathbx}{bold}{U}{BOONDOX-ds}{b}{n}
\DeclareMathAlphabet{\mathbbx} {U}{BOONDOX-ds}{b}{n}

\RequirePackage{bussproofs}
\RequirePackage[tableaux]{prooftrees}

\newenvironment{oltableau}{\center\tableau{}} %wff format={anchor = base west}}}
       {\endtableau\endcenter}
       
\newcommand{\formula}[1]{$#1$}

\usepackage{tabulary}
\usepackage{booktabs}

\def\begincols{\begin{columns}}
\def\begincol{\begin{column}}
\def\endcol{\end{column}}
\def\endcols{\end{columns}}

\usepackage[italic]{mathastext}
\usepackage{nicefrac}

\definecolor{mygreen}{RGB}{0, 100, 0}
\definecolor{mypink2}{RGB}{219, 48, 122}
\definecolor{dodgerblue}{RGB}{30,144,255}

\def\True{\textcolor{dodgerblue}{\text{T}}}
\def\False{\textcolor{red}{\text{F}}}

\title{305 Lecture 35 - Updating on Multiple Data Points}
\author{Brian Weatherson}
\date{July 22, 2020}

\begin{document}
\frame{\titlepage}

\begin{frame}{Plan}
\protect\hypertarget{plan}{}

\begin{itemize}
\tightlist
\item
  We will end the week by looking at some examples of updating on
  multiple data points.
\end{itemize}

\end{frame}

\begin{frame}{Associated Reading}
\protect\hypertarget{associated-reading}{}

Odds and Ends, Chapter 9

\end{frame}

\begin{frame}{Conditional Independence}
\protect\hypertarget{conditional-independence}{}

In a lot of cases, the two data points we get will not be
probabilistically independent, but they will be \textbf{conditionally
independent}.

That is, if \(B_1\) and \(B_2\) are the data points, and \(X\) is an
arbitrary hypothesis (like \(A, \neg A\)), we will have

\begin{quote}
\(\Pr(B_1 | X)\Pr(B_2 | X) = \Pr(B_1 \wedge B_2 | X)\)
\end{quote}

\end{frame}

\begin{frame}{Biased Coins}
\protect\hypertarget{biased-coins}{}

Here is one kind of case where the happens.

\begin{itemize}
\tightlist
\item
  We have a bunch of biased coins. For each of them, there is a
  probability \(p\) of heads on an arbitrary flip, but we don't know
  what that is. \pause
\item
  The results of two flips of the same coin are not indepdendent.
\item
  If one flip lands heads, that is evidence of a bias towards heads, and
  hence it increases the probability of heads on the next flip. \pause
\item
  But conditional on a hypothesis about the bias of the coin, the flips
  are independent.
\end{itemize}

\end{frame}

\begin{frame}{Skilled Activity}
\protect\hypertarget{skilled-activity}{}

A perhaps more real-life case of this is skilled action, like shooting
free throws.

\begin{itemize}
\tightlist
\item
  The success of one attempt is not independent of the success of the
  previous.
\item
  But conditional on the skill of the actor, the attempts are (probably,
  more or less) independent.
\end{itemize}

\end{frame}

\begin{frame}{Sampling With Replacement}
\protect\hypertarget{sampling-with-replacement}{}

Drawing from a selection \textbf{with replacement} produces conditional
independence.

\begin{itemize}
\tightlist
\item
  If I don't know how many black marbles are in an urn, then drawing a
  black marble \textbf{and replacing it} will be evidence that the next
  marble is black.
\item
  But conditional on a hypothesis about the nature of the urn, the draws
  with replacement will be independent.
\end{itemize}

\end{frame}

\begin{frame}{Yesterday, Today, Tomorrow}
\protect\hypertarget{yesterday-today-tomorrow}{}

This is a little off topic, but a lot of real world phenomena satisfy
(roughly) the following condition.

\begin{itemize}
\tightlist
\item
  How things were yesterday is a good (probabilistic) guide to how
  things will be tomorrow.
\item
  So how things will be tomorrow is not independent of how things were
  yesterday. \pause
\item
  But, conditional on how things are today, how things were yesterday
  and will be tomorrow are independent.
\item
  Knowing how things were yesterday doesn't tell you any more about how
  things will be tomorrow once you know how things are today.
\end{itemize}

\end{frame}

\begin{frame}{Markov Chains}
\protect\hypertarget{markov-chains}{}

A chain of events where every event is probabilistically dependent on
the previous one, but only on the previous one, is called a
\textbf{Markov Chain}. \pause

\begin{itemize}
\tightlist
\item
  Lots of real world processes are (more or less) Markov Chains.
\item
  Weather systems, for instance, are probably more or less Markov
  Chains.
\item
  And lots of ecological models assume that animal populations are
  Markov Chains.
\item
  And the core idea is just conditional independence.
\end{itemize}

\end{frame}

\begin{frame}{Conditional Independence}
\protect\hypertarget{conditional-independence-1}{}

In cases where the data points \(B_1\) and \(B_2\) are independent, we
have an easy story about how to work out the probabilities.

\begin{quote}
\(\Pr(B_1 \wedge B_2 | X) = \Pr(B_1 | X)\Pr(B_2 | X)\)
\end{quote}

\end{frame}

\begin{frame}{Same Event}
\protect\hypertarget{same-event}{}

There is an even simpler formula where \(B_1\) and \(B_2\) are the
`same' event, like the coin landing heads both time, or the same color
marble being drawn.

\begin{quote}
\(\Pr(B_1 \wedge B_2 | X) = \Pr(B_1 | X)^2\)
\end{quote}

\end{frame}

\begin{frame}{An Example}
\protect\hypertarget{an-example}{}

There are two urns in front of us.

\begin{itemize}
\tightlist
\item
  One of them - urn A - has 4 red marbles, 3 green marbles, and 3 blue
  marbles.
\item
  The other - urn B- has 8 red marbles, 1 green marbles and 1 blue
  marbles. \pause
\end{itemize}

One of the urns will be selected at random, and then a marble drawn from
it.

\begin{itemize}
\tightlist
\item
  If the marble is red, what is the probability that Urn A was selected?
\end{itemize}

\end{frame}

\begin{frame}{A Table}
\protect\hypertarget{a-table}{}

I'll just do the column for red marble selected.

\begin{longtable}[]{@{}cc@{}}
\toprule
& Red\tabularnewline
\midrule
\endhead
Urn A & \(0.5 \times 0.4 = 0.2\)\tabularnewline
Urn B & \(0.5 \times 0.8 = 0.4\)\tabularnewline
\textbf{Total} & \(0.2 + 0.4 = 0.6\)\tabularnewline
\bottomrule
\end{longtable}

\pause

\[
\Pr(A | Red) = \frac{\Pr(A \wedge Red)}{\Pr(Red)} = \frac{0.2}{0.6} = \frac{1}{3}
\]

\end{frame}

\begin{frame}{Another Example}
\protect\hypertarget{another-example}{}

There are two urns in front of us.

\begin{itemize}
\tightlist
\item
  One of them - urn A - has 4 red marbles, 3 green marbles, and 3 blue
  marbles.
\item
  The other - urn B- has 8 red marbles, 1 green marbles and 1 blue
  marbles. \pause
\end{itemize}

One of the urns will be selected at random, and then two marbles drawn
from it \textbf{with replacement}.

\begin{itemize}
\tightlist
\item
  If both draws are red, what is the probability that Urn A was
  selected?
\end{itemize}

\end{frame}

\begin{frame}{A Table}
\protect\hypertarget{a-table-1}{}

\begin{longtable}[]{@{}cc@{}}
\toprule
& Red-Red\tabularnewline
\midrule
\endhead
Urn A & \(0.5 \times 0.4^2 = 0.08\)\tabularnewline
Urn B & \(0.5 \times 0.8^2 = 0.32\)\tabularnewline
\textbf{Total} & \(0.08 + 0.32 = 0.4\)\tabularnewline
\bottomrule
\end{longtable}

\pause

\[
\Pr(A | Red-Red) = \frac{\Pr(A \wedge Red-Red)}{\Pr(Red-Red)} = \frac{0.08}{0.4} = \frac{1}{5}
\]

The probability of Urn A fell by a lot.

\end{frame}

\begin{frame}{Yet Another Example}
\protect\hypertarget{yet-another-example}{}

There are two urns in front of us.

\begin{itemize}
\tightlist
\item
  One of them - urn A - has 4 red marbles, 3 green marbles, and 3 blue
  marbles.
\item
  The other - urn B- has 8 red marbles, 1 green marbles and 1 blue
  marbles. \pause
\end{itemize}

One of the urns will be selected at random, and then two marbles drawn
from it \textbf{with replacement}.

\begin{itemize}
\tightlist
\item
  If the first draw is red and the second green, what is the probability
  that Urn A was selected?
\end{itemize}

\end{frame}

\begin{frame}{A Table}
\protect\hypertarget{a-table-2}{}

\begin{longtable}[]{@{}cc@{}}
\toprule
& Red-Green\tabularnewline
\midrule
\endhead
Urn A & \(0.5 \times 0.4 \times 0.3 = 0.06\)\tabularnewline
Urn B & \(0.5 \times 0.8 \times 0.1 = 0.04\)\tabularnewline
\textbf{Total} & \(0.06 + 0.04 = 0.1\)\tabularnewline
\bottomrule
\end{longtable}

\pause

\[
\Pr(A | Red-Green) = \frac{\Pr(A \wedge Red-Green)}{\Pr(Red-Green)} = \frac{0.06}{0.1} = \frac{3}{5}
\]

The probability of Urn A rose by a lot.

\end{frame}

\hypertarget{dependent-events}{%
\section{Dependent Events}\label{dependent-events}}

\begin{frame}{Dependence}
\protect\hypertarget{dependence}{}

What happens if the events \(B_1\) and \(B_2\) are dependent on one or
other of the hypotheses?

\begin{itemize}
\tightlist
\item
  The typical case is that they will be dependent on none or all of the
  hypotheses.
\item
  But it's possible in principle to have independence on some and
  dependence on others.
\item
  And in that case we have to use the more complicated procedure I'm
  about to describe.
\end{itemize}

\end{frame}

\begin{frame}{Sampling Without Replacement}
\protect\hypertarget{sampling-without-replacement}{}

The paradigm example of conditional dependence is sampling
\textbf{without replacement}.

\begin{itemize}
\tightlist
\item
  Assume you know which urn I'm using.
\item
  Then the draws without replacement won't be independent because every
  time you draw a marble, there are fewer marbles of that color to draw
  the next time.
\end{itemize}

\end{frame}

\begin{frame}{Example}
\protect\hypertarget{example}{}

Assume that I am using urn A. (Or assume that we are working out
conditional probabilities conditional on urn A.)

\begin{itemize}
\tightlist
\item
  For the first draw, the probability of red is 4 in 10, or 0.4.
\item
  Conditional on the first draw being red, the probability of the second
  draw being red is 3 in 9, or \(\frac{1}{3}\).
\item
  That's because there are now 9 marbles left, and 3 of them are red.
\end{itemize}

\end{frame}

\begin{frame}{Continuing the Example}
\protect\hypertarget{continuing-the-example}{}

So to work out the probability of some sequence of draws \(D_1, D_2\)
given a hypothesis \(X\) about the urn, we need to use the more
complicated rule.

\[
\Pr(D_1 \wedge D_2 | X) = \Pr(D_1 | X) \Pr(D_2 | X \wedge D_1)
\]

\pause

For example

\[
\Pr(Red_1 \wedge Red_2 | A) = \Pr(Red_1 | A)\Pr(Red_2 | A \wedge Red_1) = \frac{4}{10} \times \frac{3}{9} = \frac{2}{15}
\]

\end{frame}

\begin{frame}{Continuing the Example}
\protect\hypertarget{continuing-the-example-1}{}

So to work out the probability of some sequence of draws \(D_1, D_2\)
given a hypothesis \(X\) about the urn, we need to use the more
complicated rule.

\[
\Pr(D_1 \wedge D_2 | X) = \Pr(D_1 | X) \Pr(D_2 | X \wedge D_1)
\]

For example

\[
\Pr(Red_1 \wedge Red_2 | B) = \Pr(Red_1 | B)\Pr(Red_2 | B \wedge Red_1) = \frac{8}{10} \times \frac{7}{9} = \frac{28}{45}
\]

\end{frame}

\begin{frame}{Another Example}
\protect\hypertarget{another-example-1}{}

There are two urns in front of us.

\begin{itemize}
\tightlist
\item
  One of them - urn A - has 4 red marbles, 3 green marbles, and 3 blue
  marbles.
\item
  The other - urn B- has 8 red marbles, 1 green marbles and 1 blue
  marbles. \pause
\end{itemize}

One of the urns will be selected at random, and then two marbles drawn
from it \textbf{without replacement}.

\begin{itemize}
\tightlist
\item
  If both draws are red, what is the probability that Urn A was
  selected?
\end{itemize}

\end{frame}

\begin{frame}{A Table}
\protect\hypertarget{a-table-3}{}

\begin{longtable}[]{@{}cc@{}}
\toprule
& Red-Red\tabularnewline
\midrule
\endhead
Urn A &
\(0.5 \times \frac{4}{10} \times \frac{3}{9} = \frac{1}{15}\)\tabularnewline
Urn B &
\(0.5 \times \frac{8}{10} \times \frac{7}{9} = \frac{14}{45}\)\tabularnewline
\textbf{Total} &
\(\frac{1}{15} + \frac{14}{45} = \frac{17}{45}\)\tabularnewline
\bottomrule
\end{longtable}

\pause

\[
\Pr(A | Red-Red) = \frac{\Pr(A \wedge Red-Red)}{\Pr(Red-Red)} = \frac{\frac{1}{15}}{\frac{17}{45}} = \frac{3}{17}
\]

The probability of Urn A fell by a bit more.

\end{frame}

\begin{frame}{Yet Another Example}
\protect\hypertarget{yet-another-example-1}{}

There are two urns in front of us.

\begin{itemize}
\tightlist
\item
  One of them - urn A - has 4 red marbles, 3 green marbles, and 3 blue
  marbles.
\item
  The other - urn B- has 8 red marbles, 1 green marbles and 1 blue
  marbles. \pause
\end{itemize}

One of the urns will be selected at random, and then two marbles drawn
from it \textbf{with replacement}.

\begin{itemize}
\tightlist
\item
  If the first draw is red and the second green, what is the probability
  that Urn A was selected?
\end{itemize}

\end{frame}

\begin{frame}{The General Conjunction Rule}
\protect\hypertarget{the-general-conjunction-rule}{}

To work out the probability of some sequence of draws \(D_1, D_2\) given
a hypothesis \(X\) about the urn, we need to use the more complicated
rule.

\[
\Pr(D_1 \wedge D_2 | X) = \Pr(D_1 | X) \Pr(D_2 | X \wedge D_1)
\]

\pause

So in this case we get

\[
\Pr(Red_1 \wedge Green_2 | A) = \Pr(Red_1 | A)\Pr(Green_2 | A \wedge Red_1) = \frac{4}{10} \times \frac{3}{9} = \frac{2}{15}
\]

\end{frame}

\begin{frame}{The General Conjunction Rule}
\protect\hypertarget{the-general-conjunction-rule-1}{}

To work out the probability of some sequence of draws \(D_1, D_2\) given
a hypothesis \(X\) about the urn, we need to use the more complicated
rule.

\[
\Pr(D_1 \wedge D_2 | X) = \Pr(D_1 | X) \Pr(D_2 | X \wedge D_1)
\]

And for Urn B we get

\[
\Pr(Red_1 \wedge Green_2 | B) = \Pr(Red_1 | B)\Pr(Green_2 | B \wedge Red_1) = \frac{8}{10} \times \frac{1}{9} = \frac{4}{45}
\]

\end{frame}

\begin{frame}{A Table}
\protect\hypertarget{a-table-4}{}

\begin{longtable}[]{@{}cc@{}}
\toprule
& Red-Green\tabularnewline
\midrule
\endhead
Urn A &
\(0.5 \times \frac{4}{10} \times \frac{3}{9} = \frac{1}{15}\)\tabularnewline
Urn B &
\(0.5 \times \frac{8}{10} \times \frac{1}{9} = \frac{2}{45}\)\tabularnewline
\textbf{Total} &
\(\frac{1}{15} + \frac{2}{45} = \frac{5}{45}\)\tabularnewline
\bottomrule
\end{longtable}

\pause

\[
\Pr(A | Red-Green) = \frac{\Pr(A \wedge Red-Red)}{\Pr(Red-Red)} = \frac{\frac{1}{15}}{\frac{5}{45}} = \frac{3}{5}
\]

Which, interestingly, is exactly the same as in the with replacement
case.

\end{frame}

\begin{frame}{Last (Difficult) Example}
\protect\hypertarget{last-difficult-example}{}

\begin{itemize}
\tightlist
\item
  There are four urns in the room, three of type X, one of type Y.
\item
  The type X urns have 4 blue marbles and 2 yellow marbles.
\item
  The type Y urn has 5 blue marbles and 3 yellow marbles.
\item
  One of the four urns was selected at random.
\item
  Then two marbles were selected \textbf{without replacement} from the
  randomly selected urn.
\item
  The first was blue, the second was yellow.
\item
  A third marble is about to be selected.
\item
  What is the probability that it is blue?
\end{itemize}

\end{frame}

\begin{frame}{The Table}
\protect\hypertarget{the-table}{}

\begin{longtable}[]{@{}cc@{}}
\toprule
Urn & Blue-then-Yellow\tabularnewline
\midrule
\endhead
Type X &
\(\frac{3}{4} \times \frac{4}{6} \times \frac{2}{5} = \frac{1}{5}\)\tabularnewline
Type Y &\tabularnewline
\textbf{Total} &\tabularnewline
\bottomrule
\end{longtable}

\[
\Pr(X \wedge Blue_1 \wedge Yellow_2) = \Pr(X) \times \Pr(Blue_1 | X) \times \Pr(Yellow_2 | X \wedge Blue_1)
\]

\end{frame}

\begin{frame}{The Table}
\protect\hypertarget{the-table-1}{}

\begin{longtable}[]{@{}cc@{}}
\toprule
Urn & Blue-then-Yellow\tabularnewline
\midrule
\endhead
Type X &
\(\frac{3}{4} \times \frac{4}{6} \times \frac{2}{5} = \frac{1}{5}\)\tabularnewline
Type Y &
\(\frac{1}{4} \times \frac{5}{8} \times \frac{3}{7} = \frac{15}{224}\)\tabularnewline
\textbf{Total} &\tabularnewline
\bottomrule
\end{longtable}

\[
\Pr(Y \wedge Blue_1 \wedge Yellow_2) = \Pr(Y) \times \Pr(Blue_1 | Y) \times \Pr(Yellow_2 | Y \wedge Blue_1)
\]

\end{frame}

\begin{frame}{The Table}
\protect\hypertarget{the-table-2}{}

\begin{longtable}[]{@{}cc@{}}
\toprule
Urn & Blue-then-Yellow\tabularnewline
\midrule
\endhead
Type X & \(\frac{1}{5}\)\tabularnewline
Type Y & \(\frac{15}{224}\)\tabularnewline
\textbf{Total} & \(\frac{299}{1120}\)\tabularnewline
\bottomrule
\end{longtable}

You should double check this, but I think

\[
\frac{1}{5} + \frac{15}{224} = \frac{299}{1120}
\]

So that's \(\Pr(Blue_1 \wedge Yellow_2)\)

\end{frame}

\begin{frame}{Conditional Probabilities}
\protect\hypertarget{conditional-probabilities}{}

\[
\Pr(X | Blue_1 \wedge Yellow_2) = \frac{\Pr(X \wedge Blue_1 \wedge Yellow_2)}{\Pr(Blue_1 \wedge Yellow_2)} = \frac{\frac{1}{5}}{\frac{299}{1120}} = \frac{224}{299}
\]

\[
\Pr(Y | Blue_1 \wedge Yellow_2) = \frac{\Pr(Y \wedge Blue_1 \wedge Yellow_2)}{\Pr(Blue_1 \wedge Yellow_2)} = \frac{\frac{15}{224}}{\frac{299}{1120}} = \frac{75}{299}
\] The probability of Y is ever so fractionally higher than when we
started.

\end{frame}

\begin{frame}{Next Marble}
\protect\hypertarget{next-marble}{}

\begin{itemize}
\tightlist
\item
  If X (and Blue-followed-by-Yellow), the probability of next marble
  being blue is \(\frac{3}{4}\).
\item
  If Y (and Blue-followed-by-Yellow), the probability of next marble
  being blue is \(\frac{2}{3}\). \pause
\item
  So overall probability of next marble being blue is
\end{itemize}

\[
 \frac{224}{299} \times \frac{3}{4} + \frac{75}{299} \times \frac{2}{3} = \frac{218}{299} \approx 0.729
\]

\end{frame}

\begin{frame}{General Strategy of Last Slide}
\protect\hypertarget{general-strategy-of-last-slide}{}

\begin{itemize}
\tightlist
\item
  If there are two hypotheses X and Y, and you want to know the
  probability of some event E, it will be given by
\end{itemize}

\[
\Pr(E) = \Pr(X)\Pr(E | X) + \Pr(Y)\Pr(E | Y)
\]

And that generalises to the case where there are multiple hypotheses
\(H_1, \dots H_n\)

\[
\Pr(E) = \Pr(H_1)\Pr(E | H_1) + \dots +  \Pr(H_n)\Pr(E | H_n)
\]

\end{frame}

\begin{frame}{For Next Time}
\protect\hypertarget{for-next-time}{}

Next week we will look at the use of probability in decision making, and
in science.

\end{frame}

\end{document}
