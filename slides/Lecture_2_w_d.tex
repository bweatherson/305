% Options for packages loaded elsewhere
\PassOptionsToPackage{unicode}{hyperref}
\PassOptionsToPackage{hyphens}{url}
%
\documentclass[
  ignorenonframetext,
]{beamer}
\usepackage{pgfpages}
\setbeamertemplate{caption}[numbered]
\setbeamertemplate{caption label separator}{: }
\setbeamercolor{caption name}{fg=normal text.fg}
\beamertemplatenavigationsymbolsempty
% Prevent slide breaks in the middle of a paragraph
\widowpenalties 1 10000
\raggedbottom
\setbeamertemplate{part page}{
  \centering
  \begin{beamercolorbox}[sep=16pt,center]{part title}
    \usebeamerfont{part title}\insertpart\par
  \end{beamercolorbox}
}
\setbeamertemplate{section page}{
  \centering
  \begin{beamercolorbox}[sep=12pt,center]{part title}
    \usebeamerfont{section title}\insertsection\par
  \end{beamercolorbox}
}
\setbeamertemplate{subsection page}{
  \centering
  \begin{beamercolorbox}[sep=8pt,center]{part title}
    \usebeamerfont{subsection title}\insertsubsection\par
  \end{beamercolorbox}
}
\AtBeginPart{
  \frame{\partpage}
}
\AtBeginSection{
  \ifbibliography
  \else
    \frame{\sectionpage}
  \fi
}
\AtBeginSubsection{
  \frame{\subsectionpage}
}
\usepackage{lmodern}
\usepackage{amssymb,amsmath}
\usepackage{ifxetex,ifluatex}
\ifnum 0\ifxetex 1\fi\ifluatex 1\fi=0 % if pdftex
  \usepackage[T1]{fontenc}
  \usepackage[utf8]{inputenc}
  \usepackage{textcomp} % provide euro and other symbols
\else % if luatex or xetex
  \usepackage{unicode-math}
  \defaultfontfeatures{Scale=MatchLowercase}
  \defaultfontfeatures[\rmfamily]{Ligatures=TeX,Scale=1}
\fi
% Use upquote if available, for straight quotes in verbatim environments
\IfFileExists{upquote.sty}{\usepackage{upquote}}{}
\IfFileExists{microtype.sty}{% use microtype if available
  \usepackage[]{microtype}
  \UseMicrotypeSet[protrusion]{basicmath} % disable protrusion for tt fonts
}{}
\makeatletter
\@ifundefined{KOMAClassName}{% if non-KOMA class
  \IfFileExists{parskip.sty}{%
    \usepackage{parskip}
  }{% else
    \setlength{\parindent}{0pt}
    \setlength{\parskip}{6pt plus 2pt minus 1pt}}
}{% if KOMA class
  \KOMAoptions{parskip=half}}
\makeatother
\usepackage{xcolor}
\IfFileExists{xurl.sty}{\usepackage{xurl}}{} % add URL line breaks if available
\IfFileExists{bookmark.sty}{\usepackage{bookmark}}{\usepackage{hyperref}}
\hypersetup{
  pdftitle={305 Lecture 23 - Truth Trees in Logic},
  pdfauthor={Brian Weatherson},
  hidelinks,
  pdfcreator={LaTeX via pandoc}}
\urlstyle{same} % disable monospaced font for URLs
\newif\ifbibliography
\setlength{\emergencystretch}{3em} % prevent overfull lines
\providecommand{\tightlist}{%
  \setlength{\itemsep}{0pt}\setlength{\parskip}{0pt}}
\setcounter{secnumdepth}{-\maxdimen} % remove section numbering
\let\Tiny=\tiny

 \setbeamertemplate{navigation symbols}{} 

% \usetheme{Madrid}
 \usetheme[numbering=none, progressbar=foot]{metropolis}
 \usecolortheme{wolverine}
 \usepackage{color}
 \usepackage{MnSymbol}
% \usepackage{movie15}

\usepackage{amssymb}% http://ctan.org/pkg/amssymb
\usepackage{pifont}% http://ctan.org/pkg/pifont
\newcommand{\cmark}{\ding{51}}%
\newcommand{\xmark}{\ding{55}}%

\DeclareSymbolFont{symbolsC}{U}{txsyc}{m}{n}
\DeclareMathSymbol{\boxright}{\mathrel}{symbolsC}{128}
\DeclareMathAlphabet{\mathpzc}{OT1}{pzc}{m}{it}


% \usepackage{tikz-qtree}
% \usepackage{markdown}
% \usepackage{prooftrees}
% \forestset{not line numbering, close with = x}
% Allow for easy commas inside trees
\renewcommand{\,}{\text{, }}


\usepackage{tabulary}

\usepackage{open-logic-config}

\setlength{\parskip}{1ex plus 0.5ex minus 0.2ex}

\AtBeginSection[]
{
\begin{frame}
	\Huge{\color{darkblue} \insertsection}
\end{frame}
}

\renewenvironment*{quote}	
	{\list{}{\rightmargin   \leftmargin} \item } 	
	{\endlist }

\definecolor{darkgreen}{rgb}{0,0.7,0}
\definecolor{darkblue}{rgb}{0,0,0.8}

\newcommand{\starttab}{\begin{center}
\vspace{6pt}
\begin{tabular}}

\newcommand{\stoptab}{\end{tabular}
\vspace{6pt}
\end{center}
\noindent}


\newcommand{\sif}{\rightarrow}
\newcommand{\siff}{\leftrightarrow}
\newcommand{\EF}{\end{frame}}


\newcommand{\TreeStart}[1]{
%\end{frame}
\begin{frame}
\begin{center}
\begin{tikzpicture}[scale=#1]
\tikzset{every tree node/.style={align=center,anchor=north}}
%\Tree
}

\newcommand{\TreeEnd}{
\end{tikzpicture}
%\end{center}
}

\newcommand{\DisplayArg}[2]{
\begin{enumerate}
{#1}
\end{enumerate}
\vspace{-6pt}
\hrulefill

%\hspace{14pt} #2
%{\addtolength{\leftskip}{14pt} #2}
\begin{quote}
{\normalfont #2}
\end{quote}
\vspace{12pt}
}

\newenvironment{ProofTree}[1][1]{
\begin{center}
\begin{tikzpicture}[scale=#1]
\tikzset{every tree node/.style={align=center,anchor=south}}
}
{
\end{tikzpicture}
\end{center}
}

\newcommand{\TreeFrame}[2]{
\begin{columns}[c]
\column{0.5\textwidth}
\begin{center}
\begin{prooftree}{}
#1
\end{prooftree}
\end{center}
\column{0.45\textwidth}
%\begin{markdown}
#2
%\end{markdown}
\end{columns}
}

\newcommand{\ScaledTreeFrame}[3]{
\begin{columns}[c]
\column{0.5\textwidth}
\begin{center}
\scalebox{#1}{
\begin{prooftree}{}
#2
\end{prooftree}
}
\end{center}
\column{0.45\textwidth}
%\begin{markdown}
#3
%\end{markdown}
\end{columns}
}

\usepackage[bb=boondox]{mathalfa}
\DeclareMathAlphabet{\mathbx}{U}{BOONDOX-ds}{m}{n}
\SetMathAlphabet{\mathbx}{bold}{U}{BOONDOX-ds}{b}{n}
\DeclareMathAlphabet{\mathbbx} {U}{BOONDOX-ds}{b}{n}

\RequirePackage{bussproofs}
\RequirePackage[tableaux]{prooftrees}

\newenvironment{oltableau}{\center\tableau{}} %wff format={anchor = base west}}}
       {\endtableau\endcenter}
       
\newcommand{\formula}[1]{$#1$}

\usepackage{tabulary}
\usepackage{booktabs}

\def\begincols{\begin{columns}}
\def\begincol{\begin{column}}
\def\endcol{\end{column}}
\def\endcols{\end{columns}}

\usepackage[italic]{mathastext}
\usepackage{nicefrac}

\definecolor{mygreen}{RGB}{0, 100, 0}
\definecolor{mypink2}{RGB}{219, 48, 122}
\definecolor{dodgerblue}{RGB}{30,144,255}

\def\True{\textcolor{dodgerblue}{\text{T}}}
\def\False{\textcolor{red}{\text{F}}}

\title{305 Lecture 23 - Truth Trees in Logic}
\author{Brian Weatherson}
\date{July 15, 2020}

\begin{document}
\frame{\titlepage}

\begin{frame}{Plan}
\protect\hypertarget{plan}{}

This lecture goes over how truth trees help us think about validity.

\end{frame}

\begin{frame}{Simple Deduction Theorem}
\protect\hypertarget{simple-deduction-theorem}{}

The following two things are equivalent; if either is true, the other
must be true too.

\begin{enumerate}
\tightlist
\item
  The sentence \(X \rightarrow Y\) is a logical truth.
\item
  The argument \(X \vdash Y\) is valid. \pause
\end{enumerate}

Proof: In each case, we get a tree that has \(X\) is true and \(Y\) is
false, and test whether it closes. If so, both are true. If not, both
are false.

\end{frame}

\begin{frame}{General Deduction Theorem}
\protect\hypertarget{general-deduction-theorem}{}

The following two things are equivalent; if either is true, the other
must be true too.

\begin{enumerate}
\tightlist
\item
  The argument \(\Gamma \vdash X \rightarrow Y\) is valid.
\item
  The argument \(\Gamma, X \vdash Y\) is valid. \pause
\end{enumerate}

Proof: In each case, we get a tree that has (after applying the rule for
\(X \rightarrow Y\) is false) every sentence in \(\Gamma\) being true,
and \(X\) being true and \(Y\) being false, and test whether it closes.
If so, both are true. If not, both are false.

\end{frame}

\begin{frame}{Terminology}
\protect\hypertarget{terminology}{}

\begin{itemize}
\tightlist
\item
  \(\Gamma \vDash A\) means that the truth table for \(\Gamma\) and
  \(A\) does not have any rows where everything in \(\Gamma\) is true
  and \(A\) is false.
\item
  More generally, it means every \textbf{model} for \(\Gamma\) is also a
  model for \(A\). \pause
\item
  \(\Gamma \vdash A\) means that the tree that starts with \(\Gamma\)
  being true and \(A\) being false closes.
\item
  More generally, it means that there is a \textbf{proof} of \(A\) on
  the basis of \(\Gamma\). \pause
\end{itemize}

We will just talk about the specific case today (relating truth tables
and tableaux), but we'll occasionally come back to the general case
(relating models and proofs) when we cover modal logic.

\end{frame}

\begin{frame}{Soundness}
\protect\hypertarget{soundness}{}

Saying that the tableaux system is \textbf{sound} means this:

\begin{quote}
If \(\Gamma \vdash A\), then \(\Gamma \vDash A\).
\end{quote}

That is, if the tree shows that \(\Gamma\) implies \(A\), then there is
no row on the truth table where everything in \(\Gamma\) is true and
\(A\) is false.

\end{frame}

\begin{frame}{Completeness}
\protect\hypertarget{completeness}{}

Saying that the tableaux system is \textbf{complete} means this:

\begin{quote}
If \(\Gamma \vdash A\), then \(\Gamma \vDash A\).
\end{quote}

That is, if every row on the truth table where \(\Gamma\) is true is
also a row where \(A\) is true, then the tableaux that starts with
\(\Gamma\) being true and \(A\) being false will close.

\end{frame}

\begin{frame}{Proving These}
\protect\hypertarget{proving-these}{}

The proofs of soundness and completeness are not at all trivial.

\begin{itemize}
\tightlist
\item
  And in general, these results do not hold.
\item
  Famously, natural proof systems for mathematics do not prove all of
  the mathematical truths.
\item
  But that's precisely what we're not getting into here.
\item
  I'm just explaining the \(\vDash\) and \(\vdash\) symbols.
\item
  Since soundness and completeness do hold here, we can move between
  them freely.
\end{itemize}

\end{frame}

\begin{frame}{General Principle}
\protect\hypertarget{general-principle}{}

If something is a logical truth, then uniformly replacing one of the
sentence letters with another sentence will still be a logical truth.

\begin{itemize}
\tightlist
\item
  This is kind of obvious if the replacement is another sentence letter.
\item
  But it's a powerful technique if it is not.
\end{itemize}

\end{frame}

\begin{frame}{Substitution Example}
\protect\hypertarget{substitution-example}{}

Start with this logical truth, which I'll do the tree for on the tablet.

\begin{quote}
\((A \rightarrow B) \rightarrow (\neg B \rightarrow \neg A) \pause\)
\end{quote}

Now replace \(B\) with \(C \wedge D\). We get, without even having to do
a new tree, that this is a logical truth.

\begin{quote}
\((A \rightarrow (C \wedge D)) \rightarrow (\neg (C \wedge D) \rightarrow \neg A)\)
\end{quote}

\end{frame}

\begin{frame}{Congruence}
\protect\hypertarget{congruence}{}

Here's another important fact about logical truth.

\begin{quote}
If you substitute in one \textbf{logically equivalent} sentence for
another, you always preserve logical validity.
\end{quote}

Sentences \(X\) and \(Y\) are logically equivalent when they have the
same truth table, i.e., \(X \vDash Y\) and \(Y \vDash X\).

\end{frame}

\begin{frame}{Putting these together}
\protect\hypertarget{putting-these-together}{}

So if 1 is a logical truth, so is 2.

\begin{enumerate}
\tightlist
\item
  \((A \rightarrow (C \wedge D)) \rightarrow (\neg (C \wedge D) \rightarrow \neg A)\)
\item
  \((A \rightarrow (C \wedge D)) \rightarrow ((\neg C \vee \neg D) \rightarrow \neg A) \pause\)
\end{enumerate}

All I've done here is replace \(\neg (C \wedge D)\) with
\(\neg C \vee \neg D\). And those are equivalent, though we should check
that.

\end{frame}

\begin{frame}{For Next Time}
\protect\hypertarget{for-next-time}{}

We will go over some examples of truth trees.

\end{frame}

\end{document}
