% Options for packages loaded elsewhere
\PassOptionsToPackage{unicode}{hyperref}
\PassOptionsToPackage{hyphens}{url}
%
\documentclass[
  ignorenonframetext,
]{beamer}
\usepackage{pgfpages}
\setbeamertemplate{caption}[numbered]
\setbeamertemplate{caption label separator}{: }
\setbeamercolor{caption name}{fg=normal text.fg}
\beamertemplatenavigationsymbolsempty
% Prevent slide breaks in the middle of a paragraph
\widowpenalties 1 10000
\raggedbottom
\setbeamertemplate{part page}{
  \centering
  \begin{beamercolorbox}[sep=16pt,center]{part title}
    \usebeamerfont{part title}\insertpart\par
  \end{beamercolorbox}
}
\setbeamertemplate{section page}{
  \centering
  \begin{beamercolorbox}[sep=12pt,center]{part title}
    \usebeamerfont{section title}\insertsection\par
  \end{beamercolorbox}
}
\setbeamertemplate{subsection page}{
  \centering
  \begin{beamercolorbox}[sep=8pt,center]{part title}
    \usebeamerfont{subsection title}\insertsubsection\par
  \end{beamercolorbox}
}
\AtBeginPart{
  \frame{\partpage}
}
\AtBeginSection{
  \ifbibliography
  \else
    \frame{\sectionpage}
  \fi
}
\AtBeginSubsection{
  \frame{\subsectionpage}
}
\usepackage{lmodern}
\usepackage{amssymb,amsmath}
\usepackage{ifxetex,ifluatex}
\ifnum 0\ifxetex 1\fi\ifluatex 1\fi=0 % if pdftex
  \usepackage[T1]{fontenc}
  \usepackage[utf8]{inputenc}
  \usepackage{textcomp} % provide euro and other symbols
\else % if luatex or xetex
  \usepackage{unicode-math}
  \defaultfontfeatures{Scale=MatchLowercase}
  \defaultfontfeatures[\rmfamily]{Ligatures=TeX,Scale=1}
\fi
% Use upquote if available, for straight quotes in verbatim environments
\IfFileExists{upquote.sty}{\usepackage{upquote}}{}
\IfFileExists{microtype.sty}{% use microtype if available
  \usepackage[]{microtype}
  \UseMicrotypeSet[protrusion]{basicmath} % disable protrusion for tt fonts
}{}
\makeatletter
\@ifundefined{KOMAClassName}{% if non-KOMA class
  \IfFileExists{parskip.sty}{%
    \usepackage{parskip}
  }{% else
    \setlength{\parindent}{0pt}
    \setlength{\parskip}{6pt plus 2pt minus 1pt}}
}{% if KOMA class
  \KOMAoptions{parskip=half}}
\makeatother
\usepackage{xcolor}
\IfFileExists{xurl.sty}{\usepackage{xurl}}{} % add URL line breaks if available
\IfFileExists{bookmark.sty}{\usepackage{bookmark}}{\usepackage{hyperref}}
\hypersetup{
  pdftitle={305 Lecture 09 - Nested Derivations},
  pdfauthor={Brian Weatherson},
  hidelinks,
  pdfcreator={LaTeX via pandoc}}
\urlstyle{same} % disable monospaced font for URLs
\newif\ifbibliography
\setlength{\emergencystretch}{3em} % prevent overfull lines
\providecommand{\tightlist}{%
  \setlength{\itemsep}{0pt}\setlength{\parskip}{0pt}}
\setcounter{secnumdepth}{-\maxdimen} % remove section numbering
\let\Tiny=\tiny

 \setbeamertemplate{navigation symbols}{} 

% \usetheme{Madrid}
 \usetheme[numbering=none, progressbar=foot]{metropolis}
 \usecolortheme{wolverine}
 \usepackage{color}
 \usepackage{MnSymbol}
% \usepackage{movie15}

\usepackage{amssymb}% http://ctan.org/pkg/amssymb
\usepackage{pifont}% http://ctan.org/pkg/pifont
\newcommand{\cmark}{\ding{51}}%
\newcommand{\xmark}{\ding{55}}%

\DeclareSymbolFont{symbolsC}{U}{txsyc}{m}{n}
\DeclareMathSymbol{\boxright}{\mathrel}{symbolsC}{128}
\DeclareMathAlphabet{\mathpzc}{OT1}{pzc}{m}{it}


% \usepackage{tikz-qtree}
% \usepackage{markdown}
% \usepackage{prooftrees}
% \forestset{not line numbering, close with = x}
% Allow for easy commas inside trees
\renewcommand{\,}{\text{, }}


\usepackage{tabulary}

\usepackage{open-logic-config}

\setlength{\parskip}{1ex plus 0.5ex minus 0.2ex}

\AtBeginSection[]
{
\begin{frame}
	\Huge{\color{darkblue} \insertsection}
\end{frame}
}

\renewenvironment*{quote}	
	{\list{}{\rightmargin   \leftmargin} \item } 	
	{\endlist }

\definecolor{darkgreen}{rgb}{0,0.7,0}
\definecolor{darkblue}{rgb}{0,0,0.8}

\newcommand{\starttab}{\begin{center}
\vspace{6pt}
\begin{tabular}}

\newcommand{\stoptab}{\end{tabular}
\vspace{6pt}
\end{center}
\noindent}


\newcommand{\sif}{\rightarrow}
\newcommand{\siff}{\leftrightarrow}
\newcommand{\EF}{\end{frame}}


\newcommand{\TreeStart}[1]{
%\end{frame}
\begin{frame}
\begin{center}
\begin{tikzpicture}[scale=#1]
\tikzset{every tree node/.style={align=center,anchor=north}}
%\Tree
}

\newcommand{\TreeEnd}{
\end{tikzpicture}
%\end{center}
}

\newcommand{\DisplayArg}[2]{
\begin{enumerate}
{#1}
\end{enumerate}
\vspace{-6pt}
\hrulefill

%\hspace{14pt} #2
%{\addtolength{\leftskip}{14pt} #2}
\begin{quote}
{\normalfont #2}
\end{quote}
\vspace{12pt}
}

\newenvironment{ProofTree}[1][1]{
\begin{center}
\begin{tikzpicture}[scale=#1]
\tikzset{every tree node/.style={align=center,anchor=south}}
}
{
\end{tikzpicture}
\end{center}
}

\newcommand{\TreeFrame}[2]{
\begin{columns}[c]
\column{0.5\textwidth}
\begin{center}
\begin{prooftree}{}
#1
\end{prooftree}
\end{center}
\column{0.45\textwidth}
%\begin{markdown}
#2
%\end{markdown}
\end{columns}
}

\newcommand{\ScaledTreeFrame}[3]{
\begin{columns}[c]
\column{0.5\textwidth}
\begin{center}
\scalebox{#1}{
\begin{prooftree}{}
#2
\end{prooftree}
}
\end{center}
\column{0.45\textwidth}
%\begin{markdown}
#3
%\end{markdown}
\end{columns}
}

\usepackage[bb=boondox]{mathalfa}
\DeclareMathAlphabet{\mathbx}{U}{BOONDOX-ds}{m}{n}
\SetMathAlphabet{\mathbx}{bold}{U}{BOONDOX-ds}{b}{n}
\DeclareMathAlphabet{\mathbbx} {U}{BOONDOX-ds}{b}{n}

\RequirePackage{bussproofs}
\RequirePackage[tableaux]{prooftrees}

\newenvironment{oltableau}{\center\tableau{}} %wff format={anchor = base west}}}
       {\endtableau\endcenter}
       
\newcommand{\formula}[1]{$#1$}

\usepackage{tabulary}
\usepackage{booktabs}

\def\begincols{\begin{columns}}
\def\begincol{\begin{column}}
\def\endcol{\end{column}}
\def\endcols{\end{columns}}

\usepackage[italic]{mathastext}
\usepackage{nicefrac}

\definecolor{mygreen}{RGB}{0, 100, 0}
\definecolor{mypink2}{RGB}{219, 48, 122}
\definecolor{dodgerblue}{RGB}{30,144,255}

\def\True{\textcolor{dodgerblue}{\text{T}}}
\def\False{\textcolor{red}{\text{F}}}

\title{305 Lecture 09 - Nested Derivations}
\author{Brian Weatherson}
\date{July 8, 2020}

\begin{document}
\frame{\titlepage}

\begin{frame}{Plan}
\protect\hypertarget{plan}{}

We're going to go over what happens when there are multiple indirect
derivations in a single argument.

\end{frame}

\begin{frame}{Associated Reading}
\protect\hypertarget{associated-reading}{}

Carnap Book, chapter 5.

\end{frame}

\begin{frame}{Using Conditionals}
\protect\hypertarget{using-conditionals}{}

Sometimes we need to prove conditionals along the way of an argument.

\begin{itemize}
\tightlist
\item
  The big picture is that we can introduce `Show' lines at any stage.
\item
  These will introduce what I'll call a `sub-proof'.
\end{itemize}

\end{frame}

\begin{frame}[fragile]{Example}
\protect\hypertarget{example}{}

To prove:
\(P \rightarrow Q, (\neg Q \rightarrow \neg P) \rightarrow R \vdash R\)

\bigskip

\begincols
\begincol{.6\textwidth}

\begin{verbatim}
1. Show: R
2.     P -> Q            :PR
3.     (~Q -> ~P) -> R   :PR
4.     Show: ~Q -> ~P
5.         ~Q            :AS
6.         ~P            :MT 2, 5
7.     :CD 6
8.     R                 :MP 3, 4
9. :DD 8
\end{verbatim}

\endcol
\begincol{.36\textwidth}

There is a lot to unpack here, and we'll spend some time going over it
all.

\endcol
\endcols

\end{frame}

\begin{frame}[fragile]{Example}
\protect\hypertarget{example-1}{}

To prove:
\(P \rightarrow Q, (\neg Q \rightarrow \neg P) \rightarrow R \vdash R\)

\bigskip

\begincols
\begincol{.6\textwidth}

\begin{verbatim}
1. Show: R
2.     P -> Q            :PR
3.     (~Q -> ~P) -> R   :PR
4.     Show: ~Q -> ~P
5.         ~Q            :AS
6.         ~P            :MT 2, 5
7.     :CD 6
8.     R                 :MP 3, 4
9. :DD 8
\end{verbatim}

\endcol
\begincol{.36\textwidth}

The first line is simply the conclusion of the argument - that's nothing
new.

\endcol
\endcols

\end{frame}

\begin{frame}[fragile]{Example}
\protect\hypertarget{example-2}{}

To prove:
\(P \rightarrow Q, (\neg Q \rightarrow \neg P) \rightarrow R \vdash R\)

\bigskip

\begincols
\begincol{.6\textwidth}

\begin{verbatim}
1. Show: R
2.     P -> Q            :PR
3.     (~Q -> ~P) -> R   :PR
4.     Show: ~Q -> ~P
5.         ~Q            :AS
6.         ~P            :MT 2, 5
7.     :CD 6
8.     R                 :MP 3, 4
9. :DD 8
\end{verbatim}

\endcol
\begincol{.36\textwidth}

The next two lines are the premises of the argument - again nothing new
here.

\endcol
\endcols

\end{frame}

\begin{frame}[fragile]{Example}
\protect\hypertarget{example-3}{}

To prove:
\(P \rightarrow Q, (\neg Q \rightarrow \neg P) \rightarrow R \vdash R\)

\bigskip

\begincols
\begincol{.6\textwidth}

\begin{verbatim}
1. Show: R
2.     P -> Q            :PR
3.     (~Q -> ~P) -> R   :PR
4.     Show: ~Q -> ~P
5.         ~Q            :AS
6.         ~P            :MT 2, 5
7.     :CD 6
8.     R                 :MP 3, 4
9. :DD 8
\end{verbatim}

\endcol
\begincol{.36\textwidth}

The big new step is at line 4.

\begin{itemize}
\tightlist
\item
  This is the first use of `Show' after line 1 we've seen.
\end{itemize}

\endcol
\endcols

\end{frame}

\begin{frame}{Proving a Conditional}
\protect\hypertarget{proving-a-conditional}{}

The way you prove a conditional that you need along the way is to:

\begin{enumerate}
\tightlist
\item
  Use `Show' to say you're going to prove it.
\item
  Assume the antecedent.
\item
  Derive the consequent.
\item
  End the subproof with a CD statement.
\end{enumerate}

\end{frame}

\begin{frame}[fragile]{Example}
\protect\hypertarget{example-4}{}

To prove:
\(P \rightarrow Q, (\neg Q \rightarrow \neg P) \rightarrow R \vdash R\)

\bigskip

\begincols
\begincol{.6\textwidth}

\begin{verbatim}
1. Show: R
2.     P -> Q            :PR
3.     (~Q -> ~P) -> R   :PR
4.     Show: ~Q -> ~P
5.         ~Q            :AS
6.         ~P            :MT 2, 5
7.     :CD 6
8.     R                 :MP 3, 4
9. :DD 8
\end{verbatim}

\endcol
\begincol{.36\textwidth}

At line 4 we say what we're doing.

\endcol
\endcols

\end{frame}

\begin{frame}[fragile]{Example}
\protect\hypertarget{example-5}{}

To prove:
\(P \rightarrow Q, (\neg Q \rightarrow \neg P) \rightarrow R \vdash R\)

\bigskip

\begincols
\begincol{.6\textwidth}

\begin{verbatim}
1. Show: R
2.     P -> Q            :PR
3.     (~Q -> ~P) -> R   :PR
4.     Show: ~Q -> ~P
5.         ~Q            :AS
6.         ~P            :MT 2, 5
7.     :CD 6
8.     R                 :MP 3, 4
9. :DD 8
\end{verbatim}

\endcol
\begincol{.36\textwidth}

At line 5 we assume the antecedent of the conditional - the left-hand
side.

\endcol
\endcols

\end{frame}

\begin{frame}[fragile]{Example}
\protect\hypertarget{example-6}{}

To prove:
\(P \rightarrow Q, (\neg Q \rightarrow \neg P) \rightarrow R \vdash R\)

\bigskip

\begincols
\begincol{.6\textwidth}

\begin{verbatim}
1. Show: R
2.     P -> Q            :PR
3.     (~Q -> ~P) -> R   :PR
4.     Show: ~Q -> ~P
5.         ~Q            :AS
6.         ~P            :MT 2, 5
7.     :CD 6
8.     R                 :MP 3, 4
9. :DD 8
\end{verbatim}

\endcol
\begincol{.36\textwidth}

From here we just start using familiar rules.

\endcol
\endcols

\end{frame}

\begin{frame}[fragile]{Example}
\protect\hypertarget{example-7}{}

To prove:
\(P \rightarrow Q, (\neg Q \rightarrow \neg P) \rightarrow R \vdash R\)

\bigskip

\begincols
\begincol{.6\textwidth}

\begin{verbatim}
1. Show: R
2.     P -> Q            :PR
3.     (~Q -> ~P) -> R   :PR
4.     Show: ~Q -> ~P
5.         ~Q            :AS
6.         ~P            :MT 2, 5
7.     :CD 6
8.     R                 :MP 3, 4
9. :DD 8
\end{verbatim}

\endcol
\begincol{.36\textwidth}

It turns out we just need one step - MT gets from 2 and 5 to \(P\).

\endcol
\endcols

\end{frame}

\begin{frame}[fragile]{Example}
\protect\hypertarget{example-8}{}

To prove:
\(P \rightarrow Q, (\neg Q \rightarrow \neg P) \rightarrow R \vdash R\)

\bigskip

\begincols
\begincol{.6\textwidth}

\begin{verbatim}
1. Show: R
2.     P -> Q            :PR
3.     (~Q -> ~P) -> R   :PR
4.     Show: ~Q -> ~P
5.         ~Q            :AS
6.         ~P            :MT 2, 5
7.     :CD 6
8.     R                 :MP 3, 4
9. :DD 8
\end{verbatim}

\endcol
\begincol{.36\textwidth}

Note that these are doubly indented.

\endcol
\endcols

\end{frame}

\begin{frame}[fragile]{Example}
\protect\hypertarget{example-9}{}

To prove:
\(P \rightarrow Q, (\neg Q \rightarrow \neg P) \rightarrow R \vdash R\)

\bigskip

\begincols
\begincol{.6\textwidth}

\begin{verbatim}
1. Show: R
2.     P -> Q            :PR
3.     (~Q -> ~P) -> R   :PR
4.     Show: ~Q -> ~P
5.         ~Q            :AS
6.         ~P            :MT 2, 5
7.     :CD 6
8.     R                 :MP 3, 4
9. :DD 8
\end{verbatim}

\endcol
\begincol{.36\textwidth}

Every time we start a sub-proof, we indent by more spaces. I'm using
four, though I don't think it insists.

\endcol
\endcols

\end{frame}

\begin{frame}[fragile]{Example}
\protect\hypertarget{example-10}{}

To prove:
\(P \rightarrow Q, (\neg Q \rightarrow \neg P) \rightarrow R \vdash R\)

\bigskip

\begincols
\begincol{.6\textwidth}

\begin{verbatim}
1. Show: R
2.     P -> Q            :PR
3.     (~Q -> ~P) -> R   :PR
4.     Show: ~Q -> ~P
5.         ~Q            :AS
6.         ~P            :MT 2, 5
7.     :CD 6
8.     R                 :MP 3, 4
9. :DD 8
\end{verbatim}

\endcol
\begincol{.36\textwidth}

Now we've got from antecedent to consequent, so we can end.

\endcol
\endcols

\end{frame}

\begin{frame}[fragile]{Example}
\protect\hypertarget{example-11}{}

To prove:
\(P \rightarrow Q, (\neg Q \rightarrow \neg P) \rightarrow R \vdash R\)

\bigskip

\begincols
\begincol{.6\textwidth}

\begin{verbatim}
1. Show: R
2.     P -> Q            :PR
3.     (~Q -> ~P) -> R   :PR
4.     Show: ~Q -> ~P
5.         ~Q            :AS
6.         ~P            :MT 2, 5
7.     :CD 6
8.     R                 :MP 3, 4
9. :DD 8
\end{verbatim}

\endcol
\begincol{.36\textwidth}

At line 7 we record that we've got from \(\neg Q\) to \(\neg P\), so we
can say we've shown that \(\neg Q \rightarrow \neg P\).

\endcol
\endcols

\end{frame}

\begin{frame}[fragile]{Example}
\protect\hypertarget{example-12}{}

To prove:
\(P \rightarrow Q, (\neg Q \rightarrow \neg P) \rightarrow R \vdash R\)

\bigskip

\begincols
\begincol{.6\textwidth}

\begin{verbatim}
1. Show: R
2.     P -> Q            :PR
3.     (~Q -> ~P) -> R   :PR
4.     Show: ~Q -> ~P
5.         ~Q            :AS
6.         ~P            :MT 2, 5
7.     :CD 6
8.     R                 :MP 3, 4
9. :DD 8
\end{verbatim}

\endcol
\begincol{.36\textwidth}

We are done with that part of the proof, so we can go back to normal
indenting.

\endcol
\endcols

\end{frame}

\begin{frame}[fragile]{Example}
\protect\hypertarget{example-13}{}

To prove:
\(P \rightarrow Q, (\neg Q \rightarrow \neg P) \rightarrow R \vdash R\)

\bigskip

\begincols
\begincol{.6\textwidth}

\begin{verbatim}
1. Show: R
2.     P -> Q            :PR
3.     (~Q -> ~P) -> R   :PR
4.     Show: ~Q -> ~P
5.         ~Q            :AS
6.         ~P            :MT 2, 5
7.     :CD 6
8.     R                 :MP 3, 4
9. :DD 8
\end{verbatim}

\endcol
\begincol{.36\textwidth}

Given \(\neg Q \rightarrow \neg P\) we can apply MP to line 3, and
that's what we do at line 8.

\endcol
\endcols

\end{frame}

\begin{frame}[fragile]{Example}
\protect\hypertarget{example-14}{}

To prove:
\(P \rightarrow Q, (\neg Q \rightarrow \neg P) \rightarrow R \vdash R\)

\bigskip

\begincols
\begincol{.6\textwidth}

\begin{verbatim}
1. Show: R
2.     P -> Q            :PR
3.     (~Q -> ~P) -> R   :PR
4.     Show: ~Q -> ~P
5.         ~Q            :AS
6.         ~P            :MT 2, 5
7.     :CD 6
8.     R                 :MP 3, 4
9. :DD 8
\end{verbatim}

\endcol
\begincol{.36\textwidth}

The next bit is one I didn't expect - we cite the `show' line, not the
CD line.

\begin{itemize}
\tightlist
\item
  That is, at line 8 we don't cite line 7 - just the `show' at line 4.
\end{itemize}

\endcol
\endcols

\end{frame}

\begin{frame}[fragile]{Example}
\protect\hypertarget{example-15}{}

To prove:
\(P \rightarrow Q, (\neg Q \rightarrow \neg P) \rightarrow R \vdash R\)

\bigskip

\begincols
\begincol{.6\textwidth}

\begin{verbatim}
1. Show: R
2.     P -> Q            :PR
3.     (~Q -> ~P) -> R   :PR
4.     Show: ~Q -> ~P
5.         ~Q            :AS
6.         ~P            :MT 2, 5
7.     :CD 6
8.     R                 :MP 3, 4
9. :DD 8
\end{verbatim}

\endcol
\begincol{.36\textwidth}

And now we're done - we've proven R as required.

\endcol
\endcols

\end{frame}

\begin{frame}{A Big Restriction}
\protect\hypertarget{a-big-restriction}{}

When you do this kind of nesting, the `nested' lines are not available
for later reasoning.

\begin{itemize}
\tightlist
\item
  Anything between `Show' and `CD' is off-limits for later reasoning.
\item
  That's why we are indenting those lines - to say that they are all a
  bit of suppositional reasoning that is out-of-bounds once the
  supposition has been lifted.
\end{itemize}

\end{frame}

\begin{frame}[fragile]{A Bad Attempt at a Proof}
\protect\hypertarget{a-bad-attempt-at-a-proof}{}

To prove:
\(P \rightarrow (P \rightarrow Q), Q \rightarrow (P \rightarrow R) \vdash (P \rightarrow Q) \rightarrow R\)

\begin{verbatim}
1. Show: (P -> Q) -> R
2.     P -> Q             :AS
3.     P -> (P -> Q)      :PR
4.     Q -> (P -> R)      :PR
5.     Show: P -> Q
6.         P              :AS
7.         P -> Q         :MP 3, 6
8.         Q              :MP 6, 7
9.         P -> R         :MP 6, 7
10.    :CD 8
11.    P -> R             :MP 4, 8
12.    R                  :MP 6, 11
13.    :CD 12
\end{verbatim}

\end{frame}

\begin{frame}{A Bad Attempt at a Proof}
\protect\hypertarget{a-bad-attempt-at-a-proof-1}{}

Three mistakes on previous slide.

\begin{enumerate}
\tightlist
\item
  No reason to try to show something that you already have.
\item
  At line 11, cite a line inside a subproof.

  \begin{itemize}
  \tightlist
  \item
    It's to prove something already seen but that's actually ok; getting
    something outside of the subproof could be useful. But it's an
    illegal step.
  \end{itemize}
\item
  At line 12, cite a line inside a subproof.
\end{enumerate}

\end{frame}

\begin{frame}{Subproof}
\protect\hypertarget{subproof}{}

The Carnap book doesn't use the term `subproof', but I find it useful.

\begin{itemize}
\tightlist
\item
  I mean the lines from one of these `show' statements not on line 1
  until the `:CD' that closes it off.
\end{itemize}

\end{frame}

\begin{frame}{For Next Time}
\protect\hypertarget{for-next-time}{}

We'll start talking about how to do something that pop culture sometimes
says is impossible: prove a negative.

\end{frame}

\end{document}
