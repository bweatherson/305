% Options for packages loaded elsewhere
\PassOptionsToPackage{unicode}{hyperref}
\PassOptionsToPackage{hyphens}{url}
%
\documentclass[
  ignorenonframetext,
]{beamer}
\usepackage{pgfpages}
\setbeamertemplate{caption}[numbered]
\setbeamertemplate{caption label separator}{: }
\setbeamercolor{caption name}{fg=normal text.fg}
\beamertemplatenavigationsymbolsempty
% Prevent slide breaks in the middle of a paragraph
\widowpenalties 1 10000
\raggedbottom
\setbeamertemplate{part page}{
  \centering
  \begin{beamercolorbox}[sep=16pt,center]{part title}
    \usebeamerfont{part title}\insertpart\par
  \end{beamercolorbox}
}
\setbeamertemplate{section page}{
  \centering
  \begin{beamercolorbox}[sep=12pt,center]{part title}
    \usebeamerfont{section title}\insertsection\par
  \end{beamercolorbox}
}
\setbeamertemplate{subsection page}{
  \centering
  \begin{beamercolorbox}[sep=8pt,center]{part title}
    \usebeamerfont{subsection title}\insertsubsection\par
  \end{beamercolorbox}
}
\AtBeginPart{
  \frame{\partpage}
}
\AtBeginSection{
  \ifbibliography
  \else
    \frame{\sectionpage}
  \fi
}
\AtBeginSubsection{
  \frame{\subsectionpage}
}
\usepackage{lmodern}
\usepackage{amssymb,amsmath}
\usepackage{ifxetex,ifluatex}
\ifnum 0\ifxetex 1\fi\ifluatex 1\fi=0 % if pdftex
  \usepackage[T1]{fontenc}
  \usepackage[utf8]{inputenc}
  \usepackage{textcomp} % provide euro and other symbols
\else % if luatex or xetex
  \usepackage{unicode-math}
  \defaultfontfeatures{Scale=MatchLowercase}
  \defaultfontfeatures[\rmfamily]{Ligatures=TeX,Scale=1}
\fi
% Use upquote if available, for straight quotes in verbatim environments
\IfFileExists{upquote.sty}{\usepackage{upquote}}{}
\IfFileExists{microtype.sty}{% use microtype if available
  \usepackage[]{microtype}
  \UseMicrotypeSet[protrusion]{basicmath} % disable protrusion for tt fonts
}{}
\makeatletter
\@ifundefined{KOMAClassName}{% if non-KOMA class
  \IfFileExists{parskip.sty}{%
    \usepackage{parskip}
  }{% else
    \setlength{\parindent}{0pt}
    \setlength{\parskip}{6pt plus 2pt minus 1pt}}
}{% if KOMA class
  \KOMAoptions{parskip=half}}
\makeatother
\usepackage{xcolor}
\IfFileExists{xurl.sty}{\usepackage{xurl}}{} % add URL line breaks if available
\IfFileExists{bookmark.sty}{\usepackage{bookmark}}{\usepackage{hyperref}}
\hypersetup{
  pdftitle={305 Lecture 45 - Significance Testing},
  pdfauthor={Brian Weatherson},
  hidelinks,
  pdfcreator={LaTeX via pandoc}}
\urlstyle{same} % disable monospaced font for URLs
\newif\ifbibliography
\usepackage{longtable,booktabs}
\usepackage{caption}
% Make caption package work with longtable
\makeatletter
\def\fnum@table{\tablename~\thetable}
\makeatother
\setlength{\emergencystretch}{3em} % prevent overfull lines
\providecommand{\tightlist}{%
  \setlength{\itemsep}{0pt}\setlength{\parskip}{0pt}}
\setcounter{secnumdepth}{-\maxdimen} % remove section numbering
\let\Tiny=\tiny

 \setbeamertemplate{navigation symbols}{} 

% \usetheme{Madrid}
 \usetheme[numbering=none, progressbar=foot]{metropolis}
 \usecolortheme{wolverine}
 \usepackage{color}
 \usepackage{MnSymbol}
% \usepackage{movie15}

\usepackage{amssymb}% http://ctan.org/pkg/amssymb
\usepackage{pifont}% http://ctan.org/pkg/pifont
\newcommand{\cmark}{\ding{51}}%
\newcommand{\xmark}{\ding{55}}%

\DeclareSymbolFont{symbolsC}{U}{txsyc}{m}{n}
\DeclareMathSymbol{\boxright}{\mathrel}{symbolsC}{128}
\DeclareMathAlphabet{\mathpzc}{OT1}{pzc}{m}{it}


% \usepackage{tikz-qtree}
% \usepackage{markdown}
% \usepackage{prooftrees}
% \forestset{not line numbering, close with = x}
% Allow for easy commas inside trees
\renewcommand{\,}{\text{, }}


\usepackage{tabulary}

\usepackage{open-logic-config}

\setlength{\parskip}{1ex plus 0.5ex minus 0.2ex}

\AtBeginSection[]
{
\begin{frame}
	\Huge{\color{darkblue} \insertsection}
\end{frame}
}

\renewenvironment*{quote}	
	{\list{}{\rightmargin   \leftmargin} \item } 	
	{\endlist }

\definecolor{darkgreen}{rgb}{0,0.7,0}
\definecolor{darkblue}{rgb}{0,0,0.8}

\newcommand{\starttab}{\begin{center}
\vspace{6pt}
\begin{tabular}}

\newcommand{\stoptab}{\end{tabular}
\vspace{6pt}
\end{center}
\noindent}


\newcommand{\sif}{\rightarrow}
\newcommand{\siff}{\leftrightarrow}
\newcommand{\EF}{\end{frame}}


\newcommand{\TreeStart}[1]{
%\end{frame}
\begin{frame}
\begin{center}
\begin{tikzpicture}[scale=#1]
\tikzset{every tree node/.style={align=center,anchor=north}}
%\Tree
}

\newcommand{\TreeEnd}{
\end{tikzpicture}
%\end{center}
}

\newcommand{\DisplayArg}[2]{
\begin{enumerate}
{#1}
\end{enumerate}
\vspace{-6pt}
\hrulefill

%\hspace{14pt} #2
%{\addtolength{\leftskip}{14pt} #2}
\begin{quote}
{\normalfont #2}
\end{quote}
\vspace{12pt}
}

\newenvironment{ProofTree}[1][1]{
\begin{center}
\begin{tikzpicture}[scale=#1]
\tikzset{every tree node/.style={align=center,anchor=south}}
}
{
\end{tikzpicture}
\end{center}
}

\newcommand{\TreeFrame}[2]{
\begin{columns}[c]
\column{0.5\textwidth}
\begin{center}
\begin{prooftree}{}
#1
\end{prooftree}
\end{center}
\column{0.45\textwidth}
%\begin{markdown}
#2
%\end{markdown}
\end{columns}
}

\newcommand{\ScaledTreeFrame}[3]{
\begin{columns}[c]
\column{0.5\textwidth}
\begin{center}
\scalebox{#1}{
\begin{prooftree}{}
#2
\end{prooftree}
}
\end{center}
\column{0.45\textwidth}
%\begin{markdown}
#3
%\end{markdown}
\end{columns}
}

\usepackage[bb=boondox]{mathalfa}
\DeclareMathAlphabet{\mathbx}{U}{BOONDOX-ds}{m}{n}
\SetMathAlphabet{\mathbx}{bold}{U}{BOONDOX-ds}{b}{n}
\DeclareMathAlphabet{\mathbbx} {U}{BOONDOX-ds}{b}{n}

\RequirePackage{bussproofs}
\RequirePackage[tableaux]{prooftrees}

\newenvironment{oltableau}{\center\tableau{}} %wff format={anchor = base west}}}
       {\endtableau\endcenter}
       
\newcommand{\formula}[1]{$#1$}

\usepackage{tabulary}
\usepackage{booktabs}

\def\begincols{\begin{columns}}
\def\begincol{\begin{column}}
\def\endcol{\end{column}}
\def\endcols{\end{columns}}

\usepackage[italic]{mathastext}
\usepackage{nicefrac}

\definecolor{mygreen}{RGB}{0, 100, 0}
\definecolor{mypink2}{RGB}{219, 48, 122}
\definecolor{dodgerblue}{RGB}{30,144,255}

\def\True{\textcolor{dodgerblue}{\text{T}}}
\def\False{\textcolor{red}{\text{F}}}

\title{305 Lecture 45 - Significance Testing}
\author{Brian Weatherson}
\date{July 29, 2020}

\begin{document}
\frame{\titlepage}

\begin{frame}{Plan}
\protect\hypertarget{plan}{}

\begin{itemize}
\tightlist
\item
  We're going to end with a discussion of the role of significance
  testing in contemporary statistics.
\end{itemize}

\end{frame}

\begin{frame}{Associated Reading}
\protect\hypertarget{associated-reading}{}

\begin{itemize}
\tightlist
\item
  Odds and Ends, Chapter 19 (and part of chapter 20)
\end{itemize}

\end{frame}

\begin{frame}{Significance Testing}
\protect\hypertarget{significance-testing}{}

\begin{itemize}
\tightlist
\item
  The subjective approach is very popular within philosophy, but not
  within statistics or a lot of social sciences.
\item
  This is in part because of its subjectivity.
\item
  A lot of sciences want methods that are more objective.
\end{itemize}

\end{frame}

\begin{frame}{Significance Testing}
\protect\hypertarget{significance-testing-1}{}

What is known as `classical statistics' is based around the idea of
significance testing.

\begin{itemize}
\tightlist
\item
  The intuitive idea is that we say that a correlation reflects a real
  pattern in the underlying data if (but only if) it would be really
  improbable if we got the data we did by chance.
\item
  Intuitively, three heads in a row might be a coincidence, but ten
  heads in a row suggests something odd is going on.
\end{itemize}

\end{frame}

\begin{frame}{Significance Testing}
\protect\hypertarget{significance-testing-2}{}

So say that we want to argue that two things are connected.

\begin{itemize}
\tightlist
\item
  To make this concrete, say that we want to argue that the survival
  rates for people who take our company's drug are higher than for
  people who do not.
\item
  What we do is give a bunch of people the drug (after getting all the
  approvals!)
\item
  We then look at their survival rates, and ask \emph{How likely would
  this data be if our drug had no effect at all}?
\item
  If that number is low enough, we conclude that our drug works.
  (Profit!)
\end{itemize}

\end{frame}

\begin{frame}{Significance Testing}
\protect\hypertarget{significance-testing-3}{}

\begin{itemize}
\tightlist
\item
  That is, we work out something like \(\Pr(E | H_0)\), where \(E\) is
  the data that we get, and \(H_0\) is a \textbf{null hypothesis} that
  says there is nothing of interest here.
\item
  If \(\Pr(E | H_0)\) is low enough, we conclude that \(H_0\) is false,
  and hopefully there is a natural alternative to \(H_0\), such as that
  the drug works, that we infer.
\item
  In those cases, we will say that the data shows a significant
  correlation between taking the drug and survival rates.
\item
  This literally means that it would be really improbable to get this
  data by chance.
\end{itemize}

\end{frame}

\begin{frame}{Significance Testing}
\protect\hypertarget{significance-testing-4}{}

How low is `really low'?

\begin{itemize}
\tightlist
\item
  As the book says, this is mostly a matter of convention.
\item
  Which is ironic given the whole point was to get away from
  subjectivity.
\item
  But a common idea is that it is less than 5\%.
\item
  So a correlation is significant if it is outside the central 95\% of
  the distribution.
\end{itemize}

\end{frame}

\begin{frame}{Inverting}
\protect\hypertarget{inverting}{}

\begin{itemize}
\tightlist
\item
  Isn't this all wrong though?
\item
  Isn't it inferring from the fact that \(\Pr(E | H_0)\) is low to the
  conclusion that \(\Pr(H_0 | E)\) is low, something that we've said
  over and over again not to do? \pause
\item
  Yes, but\ldots{} \pause
\item
  In practice the method is supplemented by practical rules that avoid
  the worst consequences of allowing these inversions.
\end{itemize}

\end{frame}

\begin{frame}{Stopping Rules}
\protect\hypertarget{stopping-rules}{}

\begin{itemize}
\tightlist
\item
  As it stands, this method leads to all sorts of nonsense and, frankly,
  fraud.
\item
  It needs to be supplemented with extra rules to avoid obvious
  mistakes.
\item
  The first thing that is needed, as was realised fairly early on, was a
  commitment to external `stopping rules'.
\item
  If you are allowed to keep collecting data until the probability of
  the evidence given the null is low, you will almost always get to
  reject the null.
\end{itemize}

\end{frame}

\begin{frame}{An Experiment}
\protect\hypertarget{an-experiment}{}

\begin{itemize}
\tightlist
\item
  This is actually an experiment; the data are randomly generated anew
  every time I compile these slides.
\item
  I'm going to compile these slides, and then present them whether the
  data comes up the way I hope it does or not.
\item
  So it could go horribly wrong!
\end{itemize}

\end{frame}

\begin{frame}{An Experiment}
\protect\hypertarget{an-experiment-1}{}

\begin{itemize}
\tightlist
\item
  I'm going to simulate tossing a coin 100,000 times.
\item
  It uses the computer's random number generator, and it is set up so
  the probability of heads on each toss is 0.5, and the tosses are
  independent.
\item
  But I'm going to measure the probability of the evidence given the
  null after each toss, not just at the end.
\item
  And by `the probability of the evidence', I mean the probability of
  getting at least that many heads.
\item
  If that is outside \([0.025, 0.975]\) then we have, at this point, a
  rejection of the null.
\item
  This is absurd of course; the null is programmed to be true.
\item
  I'll run it five times to see how it goes.
\end{itemize}

\end{frame}

\begin{frame}[fragile]{Take One}
\protect\hypertarget{take-one}{}

\begin{longtable}[]{@{}rrrrrr@{}}
\toprule
t & result & heads & frequency & prob & distance\tabularnewline
\midrule
\endhead
4685 & 1 & 2399 & 0.5120598 & 0.9521008 & 0.4521008\tabularnewline
4687 & 1 & 2400 & 0.5120546 & 0.9520654 & 0.4520654\tabularnewline
4684 & 1 & 2398 & 0.5119556 & 0.9506448 & 0.4506448\tabularnewline
4686 & 0 & 2399 & 0.5119505 & 0.9506088 & 0.4506088\tabularnewline
4688 & 0 & 2400 & 0.5119454 & 0.9505728 & 0.4505728\tabularnewline
\bottomrule
\end{longtable}

Here t is the trial number, result is how the coin landed on that trial,
heads is how many heads to that point, frequency is frequency of heads
to that point, prob is probability of getting at least that many heads,
and distance is the distance of that number from 0.5. In this trial, we
ended up with this many heads.

\begin{verbatim}
## [1] 50019
\end{verbatim}

\end{frame}

\begin{frame}[fragile]{Take Two}
\protect\hypertarget{take-two}{}

\begin{longtable}[]{@{}rrrrrr@{}}
\toprule
t & result & heads & frequency & prob & distance\tabularnewline
\midrule
\endhead
35413 & 0 & 17587 & 0.4966255 & 0.1029849 & 0.3970151\tabularnewline
35417 & 0 & 17589 & 0.4966259 & 0.1029977 & 0.3970023\tabularnewline
35433 & 0 & 17597 & 0.4966274 & 0.1030489 & 0.3969511\tabularnewline
35412 & 0 & 17587 & 0.4966396 & 0.1039377 & 0.3960623\tabularnewline
35414 & 1 & 17588 & 0.4966397 & 0.1039441 & 0.3960559\tabularnewline
\bottomrule
\end{longtable}

Here t is the trial number, result is how the coin landed on that trial,
heads is how many heads to that point, frequency is frequency of heads
to that point, prob is probability of getting at least that many heads,
and distance is the distance of that number from 0.5. In this trial, we
ended up with this many heads.

\begin{verbatim}
## [1] 50012
\end{verbatim}

\end{frame}

\begin{frame}[fragile]{Take Three}
\protect\hypertarget{take-three}{}

\begin{longtable}[]{@{}rrrrrr@{}}
\toprule
t & result & heads & frequency & prob & distance\tabularnewline
\midrule
\endhead
43629 & 1 & 21997 & 0.5041830 & 0.9601343 & 0.4601343\tabularnewline
43482 & 1 & 21923 & 0.5041856 & 0.9599762 & 0.4599762\tabularnewline
43628 & 1 & 21996 & 0.5041716 & 0.9597228 & 0.4597228\tabularnewline
43630 & 0 & 21997 & 0.5041714 & 0.9597194 & 0.4597194\tabularnewline
43632 & 1 & 21998 & 0.5041713 & 0.9597159 & 0.4597159\tabularnewline
\bottomrule
\end{longtable}

Here t is the trial number, result is how the coin landed on that trial,
heads is how many heads to that point, frequency is frequency of heads
to that point, prob is probability of getting at least that many heads,
and distance is the distance of that number from 0.5. In this trial, we
ended up with this many heads.

\begin{verbatim}
## [1] 50068
\end{verbatim}

\end{frame}

\begin{frame}[fragile]{Take Four}
\protect\hypertarget{take-four}{}

\begin{longtable}[]{@{}rrrrrr@{}}
\toprule
t & result & heads & frequency & prob & distance\tabularnewline
\midrule
\endhead
70523 & 0 & 34936 & 0.4953845 & 0.0071894 & 0.4928106\tabularnewline
70623 & 0 & 34986 & 0.4953910 & 0.0072241 & 0.4927759\tabularnewline
70522 & 0 & 34936 & 0.4953915 & 0.0072646 & 0.4927354\tabularnewline
70524 & 1 & 34937 & 0.4953916 & 0.0072653 & 0.4927347\tabularnewline
70395 & 0 & 34873 & 0.4953903 & 0.0072961 & 0.4927039\tabularnewline
\bottomrule
\end{longtable}

Here t is the trial number, result is how the coin landed on that trial,
heads is how many heads to that point, frequency is frequency of heads
to that point, prob is probability of getting at least that many heads,
and distance is the distance of that number from 0.5. In this trial, we
ended up with this many heads.

\begin{verbatim}
## [1] 49860
\end{verbatim}

\end{frame}

\begin{frame}[fragile]{Take Five}
\protect\hypertarget{take-five}{}

\begin{longtable}[]{@{}rrrrrr@{}}
\toprule
t & result & heads & frequency & prob & distance\tabularnewline
\midrule
\endhead
61387 & 0 & 30456 & 0.4961311 & 0.0278665 & 0.4721335\tabularnewline
61391 & 0 & 30458 & 0.4961314 & 0.0278705 & 0.4721295\tabularnewline
61395 & 0 & 30460 & 0.4961316 & 0.0278745 & 0.4721255\tabularnewline
61397 & 0 & 30461 & 0.4961317 & 0.0278765 & 0.4721235\tabularnewline
61386 & 0 & 30456 & 0.4961392 & 0.0281248 & 0.4718752\tabularnewline
\bottomrule
\end{longtable}

Here t is the trial number, result is how the coin landed on that trial,
heads is how many heads to that point, frequency is frequency of heads
to that point, prob is probability of getting at least that many heads,
and distance is the distance of that number from 0.5. In this trial, we
ended up with this many heads.

\begin{verbatim}
## [1] 49868
\end{verbatim}

\end{frame}

\begin{frame}{Stopping Rules}
\protect\hypertarget{stopping-rules-1}{}

\begin{itemize}
\tightlist
\item
  As I said, this is a known bug in significance testing, and every
  responsible scientist who uses it knows that you aren't meant to stop
  just when you get the data you want.
\item
  But there is another kind of problem that we've recently discovered
  the importance of.
\item
  Here I was testing just one hypothesis.
\item
  What if we try to test many more hypotheses at once?
\end{itemize}

\end{frame}

\begin{frame}{P-Hacking}
\protect\hypertarget{p-hacking}{}

\begin{itemize}
\tightlist
\item
  This practice is known as \textbf{p-hacking}.
\item
  It is the tactic of looking at results within all sorts of sub-groups
  within the data in the hope of finding a significant correlation
  somewhere.
\item
  And with enough subgroups, the odds are pretty good that you'll find
  one.
\end{itemize}

\end{frame}

\begin{frame}{Another Experiment}
\protect\hypertarget{another-experiment}{}

\begin{itemize}
\tightlist
\item
  Here I'm doing 32000 coin flips, but each flip is randomly assigned to
  either having or not having three different characteristics: C1, C2
  and C3.
\item
  In medical contexts, think of these as being like sex, age, race, etc.
\item
  Again, I'm just doing coin flips here.
\item
  And I'm going to run the trials to completion.
\item
  But we're going to look for significant correlations among each group.
\item
  I'll walk through three attempts to see how likely it is we get one.
\item
  I haven't seen the data, so there isn't commentary on the slides, but
  it is quite unlikely that we'll get a significant correlation on the
  whole set. On the sub-samples though\ldots{}
\end{itemize}

\end{frame}

\begin{frame}{Experiment One}
\protect\hypertarget{experiment-one}{}

\begin{longtable}[]{@{}lrrrr@{}}
\toprule
Characteristic & Heads & Trials & Probability & Distance\tabularnewline
\midrule
\endhead
All & 15906 & 32000 & 0.1479278 & 0.3520722\tabularnewline
\bottomrule
\end{longtable}

\end{frame}

\begin{frame}{Experiment One - With One Characteristic}
\protect\hypertarget{experiment-one---with-one-characteristic}{}

\begin{longtable}[]{@{}lrrrr@{}}
\toprule
Characteristic & Heads & Trials & Probability & Distance\tabularnewline
\midrule
\endhead
Yes C3 & 7885 & 16024 & 0.0228219 & 0.4771781\tabularnewline
No C2 & 7885 & 15996 & 0.0376179 & 0.4623821\tabularnewline
No C3 & 7885 & 15976 & 0.0524126 & 0.4475874\tabularnewline
Yes C1 & 7923 & 15981 & 0.1445738 & 0.3554262\tabularnewline
No C1 & 7983 & 16019 & 0.3405911 & 0.1594089\tabularnewline
Yes C2 & 8021 & 16004 & 0.6210662 & 0.1210662\tabularnewline
\bottomrule
\end{longtable}

\end{frame}

\begin{frame}{Experiment One - With Two Characteristics}
\protect\hypertarget{experiment-one---with-two-characteristics}{}

\begin{longtable}[]{@{}lrrrr@{}}
\toprule
Characteristic & Heads & Trials & Probability & Distance\tabularnewline
\midrule
\endhead
+C3-C2 & 3835 & 7866 & 0.0139480 & 0.4860520\tabularnewline
-C1-C2 & 3933 & 8024 & 0.0398247 & 0.4601753\tabularnewline
+C3+C2 & 4151 & 8158 & 0.9457971 & 0.4457971\tabularnewline
+C1-C3 & 3964 & 8036 & 0.1163135 & 0.3836865\tabularnewline
-C1+C2 & 4050 & 7995 & 0.8820884 & 0.3820884\tabularnewline
-C3+C2 & 3870 & 7846 & 0.1179285 & 0.3820715\tabularnewline
+C1-C2 & 3952 & 7972 & 0.2265088 & 0.2734912\tabularnewline
+C1+C2 & 3971 & 8009 & 0.2304140 & 0.2695860\tabularnewline
-C3-C2 & 4050 & 8130 & 0.3738687 & 0.1261313\tabularnewline
-C1-C3 & 3956 & 7940 & 0.3809431 & 0.1190569\tabularnewline
+C1+C3 & 3959 & 7945 & 0.3852617 & 0.1147383\tabularnewline
-C1+C3 & 4027 & 8079 & 0.3947306 & 0.1052694\tabularnewline
\bottomrule
\end{longtable}

\end{frame}

\begin{frame}{Experiment One - With All Three Characteristics}
\protect\hypertarget{experiment-one---with-all-three-characteristics}{}

\begin{longtable}[]{@{}lrrrr@{}}
\toprule
Characteristic & Heads & Trials & Probability & Distance\tabularnewline
\midrule
\endhead
-C1-C2+C3 & 1904 & 3967 & 0.0060563 & 0.4939437\tabularnewline
-C1+C2+C3 & 2123 & 4112 & 0.9823723 & 0.4823723\tabularnewline
+C1+C2-C3 & 1943 & 3963 & 0.1136634 & 0.3863366\tabularnewline
+C1-C2+C3 & 1931 & 3899 & 0.2821295 & 0.2178705\tabularnewline
+C1-C2-C3 & 2021 & 4073 & 0.3191546 & 0.1808454\tabularnewline
-C1+C2-C3 & 1927 & 3883 & 0.3265963 & 0.1734037\tabularnewline
+C1+C2+C3 & 2028 & 4046 & 0.5686469 & 0.0686469\tabularnewline
-C1-C2-C3 & 2029 & 4057 & 0.5125244 & 0.0125244\tabularnewline
\bottomrule
\end{longtable}

\end{frame}

\begin{frame}{Experiment Two}
\protect\hypertarget{experiment-two}{}

\begin{longtable}[]{@{}lrrrr@{}}
\toprule
Characteristic & Heads & Trials & Probability & Distance\tabularnewline
\midrule
\endhead
All & 15996 & 32000 & 0.4843929 & 0.0156071\tabularnewline
\bottomrule
\end{longtable}

\end{frame}

\begin{frame}{Experiment Two - With One Characteristic}
\protect\hypertarget{experiment-two---with-one-characteristic}{}

\begin{longtable}[]{@{}lrrrr@{}}
\toprule
Characteristic & Heads & Trials & Probability & Distance\tabularnewline
\midrule
\endhead
Yes C1 & 7974 & 16014 & 0.3037506 & 0.1962494\tabularnewline
No C1 & 8022 & 15986 & 0.6796206 & 0.1796206\tabularnewline
Yes C3 & 8008 & 15990 & 0.5845391 & 0.0845391\tabularnewline
No C2 & 8008 & 16044 & 0.4156012 & 0.0843988\tabularnewline
Yes C2 & 7988 & 15956 & 0.5660189 & 0.0660189\tabularnewline
No C3 & 8008 & 16010 & 0.5220591 & 0.0220591\tabularnewline
\bottomrule
\end{longtable}

\end{frame}

\begin{frame}{Experiment Two - With Two Characteristics}
\protect\hypertarget{experiment-two---with-two-characteristics}{}

\begin{longtable}[]{@{}lrrrr@{}}
\toprule
Characteristic & Heads & Trials & Probability & Distance\tabularnewline
\midrule
\endhead
-C1+C3 & 4055 & 8007 & 0.8774333 & 0.3774333\tabularnewline
+C1+C3 & 3956 & 7983 & 0.2166804 & 0.2833196\tabularnewline
+C1-C2 & 4002 & 8064 & 0.2555862 & 0.2444138\tabularnewline
-C3-C2 & 3990 & 8027 & 0.3038267 & 0.1961733\tabularnewline
-C1-C3 & 3967 & 7979 & 0.3111554 & 0.1888446\tabularnewline
-C1-C2 & 4006 & 7980 & 0.6440889 & 0.1440889\tabularnewline
-C1+C2 & 4016 & 8006 & 0.6185799 & 0.1185799\tabularnewline
+C3-C2 & 4018 & 8017 & 0.5883752 & 0.0883752\tabularnewline
+C3+C2 & 3993 & 7973 & 0.5622940 & 0.0622940\tabularnewline
-C3+C2 & 3995 & 7983 & 0.5356724 & 0.0356724\tabularnewline
+C1-C3 & 4018 & 8031 & 0.5266899 & 0.0266899\tabularnewline
+C1+C2 & 3972 & 7950 & 0.4776404 & 0.0223596\tabularnewline
\bottomrule
\end{longtable}

\end{frame}

\begin{frame}{Experiment Two - With All Three Characteristics}
\protect\hypertarget{experiment-two---with-all-three-characteristics}{}

\begin{longtable}[]{@{}lrrrr@{}}
\toprule
Characteristic & Heads & Trials & Probability & Distance\tabularnewline
\midrule
\endhead
-C1-C2+C3 & 2033 & 4002 & 0.8479038 & 0.3479038\tabularnewline
+C1-C2+C3 & 1985 & 4015 & 0.2437179 & 0.2562821\tabularnewline
-C1+C2+C3 & 2022 & 4005 & 0.7363238 & 0.2363238\tabularnewline
-C1-C2-C3 & 1973 & 3978 & 0.3115369 & 0.1884631\tabularnewline
+C1+C2+C3 & 1971 & 3968 & 0.3457323 & 0.1542677\tabularnewline
+C1+C2-C3 & 2001 & 3982 & 0.6303508 & 0.1303508\tabularnewline
+C1-C2-C3 & 2017 & 4049 & 0.4129310 & 0.0870690\tabularnewline
-C1+C2-C3 & 1994 & 4001 & 0.4247685 & 0.0752315\tabularnewline
\bottomrule
\end{longtable}

\end{frame}

\begin{frame}{Experiment Three}
\protect\hypertarget{experiment-three}{}

\begin{longtable}[]{@{}lrrrr@{}}
\toprule
Characteristic & Heads & Trials & Probability & Distance\tabularnewline
\midrule
\endhead
All & 15898 & 32000 & 0.1282284 & 0.3717716\tabularnewline
\bottomrule
\end{longtable}

\end{frame}

\begin{frame}{Experiment Three - With One Characteristic}
\protect\hypertarget{experiment-three---with-one-characteristic}{}

\begin{longtable}[]{@{}lrrrr@{}}
\toprule
Characteristic & Heads & Trials & Probability & Distance\tabularnewline
\midrule
\endhead
No C2 & 7977 & 16086 & 0.1508314 & 0.3491686\tabularnewline
Yes C1 & 7920 & 15942 & 0.2118772 & 0.2881228\tabularnewline
No C1 & 7978 & 16058 & 0.2127167 & 0.2872833\tabularnewline
Yes C3 & 7977 & 16052 & 0.2219549 & 0.2780451\tabularnewline
Yes C2 & 7921 & 15914 & 0.2867798 & 0.2132202\tabularnewline
No C3 & 7977 & 15948 & 0.5221019 & 0.0221019\tabularnewline
\bottomrule
\end{longtable}

\end{frame}

\begin{frame}{Experiment Three - With Two Characteristics}
\protect\hypertarget{experiment-three---with-two-characteristics}{}

\begin{longtable}[]{@{}lrrrr@{}}
\toprule
Characteristic & Heads & Trials & Probability & Distance\tabularnewline
\midrule
\endhead
-C1-C2 & 4011 & 8142 & 0.0936151 & 0.4063849\tabularnewline
-C3-C2 & 3954 & 7999 & 0.1571372 & 0.3428628\tabularnewline
+C1+C2 & 3954 & 7998 & 0.1598257 & 0.3401743\tabularnewline
+C1-C3 & 3918 & 7907 & 0.2155796 & 0.2844204\tabularnewline
-C1+C3 & 3980 & 8017 & 0.2658441 & 0.2341559\tabularnewline
+C3+C2 & 3959 & 7965 & 0.3031296 & 0.1968704\tabularnewline
-C1-C3 & 3998 & 8041 & 0.3118280 & 0.1881720\tabularnewline
+C3-C2 & 4023 & 8087 & 0.3282329 & 0.1717671\tabularnewline
+C1+C3 & 4002 & 8035 & 0.3689348 & 0.1310652\tabularnewline
-C3+C2 & 3962 & 7949 & 0.3938937 & 0.1061063\tabularnewline
-C1+C2 & 3967 & 7916 & 0.5845504 & 0.0845504\tabularnewline
+C1-C2 & 3966 & 7944 & 0.4508892 & 0.0491108\tabularnewline
\bottomrule
\end{longtable}

\end{frame}

\begin{frame}{Experiment Three - With All Three Characteristics}
\protect\hypertarget{experiment-three---with-all-three-characteristics}{}

\begin{longtable}[]{@{}lrrrr@{}}
\toprule
Characteristic & Heads & Trials & Probability & Distance\tabularnewline
\midrule
\endhead
-C1-C2-C3 & 1985 & 4040 & 0.1388334 & 0.3611666\tabularnewline
+C1+C2-C3 & 1949 & 3948 & 0.2177434 & 0.2822566\tabularnewline
-C1-C2+C3 & 2026 & 4102 & 0.2221184 & 0.2778816\tabularnewline
+C1+C2+C3 & 2005 & 4050 & 0.2699983 & 0.2300017\tabularnewline
-C1+C2-C3 & 2013 & 4001 & 0.6594775 & 0.1594775\tabularnewline
+C1-C2-C3 & 1969 & 3959 & 0.3752975 & 0.1247025\tabularnewline
+C1-C2+C3 & 1997 & 3985 & 0.5629323 & 0.0629323\tabularnewline
-C1+C2+C3 & 1954 & 3915 & 0.4618037 & 0.0381963\tabularnewline
\bottomrule
\end{longtable}

\end{frame}

\begin{frame}{Pre-Registration}
\protect\hypertarget{pre-registration}{}

\begin{itemize}
\tightlist
\item
  The solution to this problem (which I hope the numbers came out right
  on!) is to require that scientists \textbf{pre-register} their
  hypotheses.
\item
  So you can't collect data then see what it supports, but have to say
  that one particular thing is what you're testing.
\item
  This is still in the process of becoming a universal requirement in
  respectable science; it was very much not part of standard scientific
  practice 10-20 years ago.
\end{itemize}

\end{frame}

\begin{frame}{Philosophical Question}
\protect\hypertarget{philosophical-question}{}

\begin{itemize}
\tightlist
\item
  Should we trust a method if it requires these ad hoc rules like
  announced stopping rules and pre-registration? \pause
\item
  Maybe! It depends on the alternatives.
\item
  But when you see a report on a statistical finding, you should really
  check if it satisfies these conditions.
\item
  And if it comes from a for-profit entity, you should be really
  sceptical that it does unless they are super transparent.
\end{itemize}

\end{frame}

\begin{frame}{For Next Time}
\protect\hypertarget{for-next-time}{}

We'll start looking at modal logic, the logic of `must' and `might'.

\end{frame}

\end{document}
