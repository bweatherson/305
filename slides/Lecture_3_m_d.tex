% Options for packages loaded elsewhere
\PassOptionsToPackage{unicode}{hyperref}
\PassOptionsToPackage{hyphens}{url}
%
\documentclass[
  ignorenonframetext,
]{beamer}
\usepackage{pgfpages}
\setbeamertemplate{caption}[numbered]
\setbeamertemplate{caption label separator}{: }
\setbeamercolor{caption name}{fg=normal text.fg}
\beamertemplatenavigationsymbolsempty
% Prevent slide breaks in the middle of a paragraph
\widowpenalties 1 10000
\raggedbottom
\setbeamertemplate{part page}{
  \centering
  \begin{beamercolorbox}[sep=16pt,center]{part title}
    \usebeamerfont{part title}\insertpart\par
  \end{beamercolorbox}
}
\setbeamertemplate{section page}{
  \centering
  \begin{beamercolorbox}[sep=12pt,center]{part title}
    \usebeamerfont{section title}\insertsection\par
  \end{beamercolorbox}
}
\setbeamertemplate{subsection page}{
  \centering
  \begin{beamercolorbox}[sep=8pt,center]{part title}
    \usebeamerfont{subsection title}\insertsubsection\par
  \end{beamercolorbox}
}
\AtBeginPart{
  \frame{\partpage}
}
\AtBeginSection{
  \ifbibliography
  \else
    \frame{\sectionpage}
  \fi
}
\AtBeginSubsection{
  \frame{\subsectionpage}
}
\usepackage{lmodern}
\usepackage{amssymb,amsmath}
\usepackage{ifxetex,ifluatex}
\ifnum 0\ifxetex 1\fi\ifluatex 1\fi=0 % if pdftex
  \usepackage[T1]{fontenc}
  \usepackage[utf8]{inputenc}
  \usepackage{textcomp} % provide euro and other symbols
\else % if luatex or xetex
  \usepackage{unicode-math}
  \defaultfontfeatures{Scale=MatchLowercase}
  \defaultfontfeatures[\rmfamily]{Ligatures=TeX,Scale=1}
\fi
% Use upquote if available, for straight quotes in verbatim environments
\IfFileExists{upquote.sty}{\usepackage{upquote}}{}
\IfFileExists{microtype.sty}{% use microtype if available
  \usepackage[]{microtype}
  \UseMicrotypeSet[protrusion]{basicmath} % disable protrusion for tt fonts
}{}
\makeatletter
\@ifundefined{KOMAClassName}{% if non-KOMA class
  \IfFileExists{parskip.sty}{%
    \usepackage{parskip}
  }{% else
    \setlength{\parindent}{0pt}
    \setlength{\parskip}{6pt plus 2pt minus 1pt}}
}{% if KOMA class
  \KOMAoptions{parskip=half}}
\makeatother
\usepackage{xcolor}
\IfFileExists{xurl.sty}{\usepackage{xurl}}{} % add URL line breaks if available
\IfFileExists{bookmark.sty}{\usepackage{bookmark}}{\usepackage{hyperref}}
\hypersetup{
  pdftitle={305 Lecture 28 - Probability Trees},
  pdfauthor={Brian Weatherson},
  hidelinks,
  pdfcreator={LaTeX via pandoc}}
\urlstyle{same} % disable monospaced font for URLs
\newif\ifbibliography
\usepackage{longtable,booktabs}
\usepackage{caption}
% Make caption package work with longtable
\makeatletter
\def\fnum@table{\tablename~\thetable}
\makeatother
\setlength{\emergencystretch}{3em} % prevent overfull lines
\providecommand{\tightlist}{%
  \setlength{\itemsep}{0pt}\setlength{\parskip}{0pt}}
\setcounter{secnumdepth}{-\maxdimen} % remove section numbering
\let\Tiny=\tiny

 \setbeamertemplate{navigation symbols}{} 

% \usetheme{Madrid}
 \usetheme[numbering=none, progressbar=foot]{metropolis}
 \usecolortheme{wolverine}
 \usepackage{color}
 \usepackage{MnSymbol}
% \usepackage{movie15}

\usepackage{amssymb}% http://ctan.org/pkg/amssymb
\usepackage{pifont}% http://ctan.org/pkg/pifont
\newcommand{\cmark}{\ding{51}}%
\newcommand{\xmark}{\ding{55}}%

\DeclareSymbolFont{symbolsC}{U}{txsyc}{m}{n}
\DeclareMathSymbol{\boxright}{\mathrel}{symbolsC}{128}
\DeclareMathAlphabet{\mathpzc}{OT1}{pzc}{m}{it}


% \usepackage{tikz-qtree}
% \usepackage{markdown}
% \usepackage{prooftrees}
% \forestset{not line numbering, close with = x}
% Allow for easy commas inside trees
\renewcommand{\,}{\text{, }}


\usepackage{tabulary}

\usepackage{open-logic-config}

\setlength{\parskip}{1ex plus 0.5ex minus 0.2ex}

\AtBeginSection[]
{
\begin{frame}
	\Huge{\color{darkblue} \insertsection}
\end{frame}
}

\renewenvironment*{quote}	
	{\list{}{\rightmargin   \leftmargin} \item } 	
	{\endlist }

\definecolor{darkgreen}{rgb}{0,0.7,0}
\definecolor{darkblue}{rgb}{0,0,0.8}

\newcommand{\starttab}{\begin{center}
\vspace{6pt}
\begin{tabular}}

\newcommand{\stoptab}{\end{tabular}
\vspace{6pt}
\end{center}
\noindent}


\newcommand{\sif}{\rightarrow}
\newcommand{\siff}{\leftrightarrow}
\newcommand{\EF}{\end{frame}}


\newcommand{\TreeStart}[1]{
%\end{frame}
\begin{frame}
\begin{center}
\begin{tikzpicture}[scale=#1]
\tikzset{every tree node/.style={align=center,anchor=north}}
%\Tree
}

\newcommand{\TreeEnd}{
\end{tikzpicture}
%\end{center}
}

\newcommand{\DisplayArg}[2]{
\begin{enumerate}
{#1}
\end{enumerate}
\vspace{-6pt}
\hrulefill

%\hspace{14pt} #2
%{\addtolength{\leftskip}{14pt} #2}
\begin{quote}
{\normalfont #2}
\end{quote}
\vspace{12pt}
}

\newenvironment{ProofTree}[1][1]{
\begin{center}
\begin{tikzpicture}[scale=#1]
\tikzset{every tree node/.style={align=center,anchor=south}}
}
{
\end{tikzpicture}
\end{center}
}

\newcommand{\TreeFrame}[2]{
\begin{columns}[c]
\column{0.5\textwidth}
\begin{center}
\begin{prooftree}{}
#1
\end{prooftree}
\end{center}
\column{0.45\textwidth}
%\begin{markdown}
#2
%\end{markdown}
\end{columns}
}

\newcommand{\ScaledTreeFrame}[3]{
\begin{columns}[c]
\column{0.5\textwidth}
\begin{center}
\scalebox{#1}{
\begin{prooftree}{}
#2
\end{prooftree}
}
\end{center}
\column{0.45\textwidth}
%\begin{markdown}
#3
%\end{markdown}
\end{columns}
}

\usepackage[bb=boondox]{mathalfa}
\DeclareMathAlphabet{\mathbx}{U}{BOONDOX-ds}{m}{n}
\SetMathAlphabet{\mathbx}{bold}{U}{BOONDOX-ds}{b}{n}
\DeclareMathAlphabet{\mathbbx} {U}{BOONDOX-ds}{b}{n}

\RequirePackage{bussproofs}
\RequirePackage[tableaux]{prooftrees}

\newenvironment{oltableau}{\center\tableau{}} %wff format={anchor = base west}}}
       {\endtableau\endcenter}
       
\newcommand{\formula}[1]{$#1$}

\usepackage{tabulary}
\usepackage{booktabs}

\def\begincols{\begin{columns}}
\def\begincol{\begin{column}}
\def\endcol{\end{column}}
\def\endcols{\end{columns}}

\usepackage[italic]{mathastext}
\usepackage{nicefrac}

\definecolor{mygreen}{RGB}{0, 100, 0}
\definecolor{mypink2}{RGB}{219, 48, 122}
\definecolor{dodgerblue}{RGB}{30,144,255}

\def\True{\textcolor{dodgerblue}{\text{T}}}
\def\False{\textcolor{red}{\text{F}}}

\title{305 Lecture 28 - Probability Trees}
\author{Brian Weatherson}
\date{July 20, 2020}

\begin{document}
\frame{\titlepage}

\begin{frame}{Plan}
\protect\hypertarget{plan}{}

\begin{itemize}
\tightlist
\item
  This lecture walks through a worked example of how to use probability
  trees to calculate a probability.
\end{itemize}

\end{frame}

\hypertarget{a-tree-example}{%
\section{A Tree Example}\label{a-tree-example}}

\begin{frame}{Soccer Tournament}
\protect\hypertarget{soccer-tournament}{}

There is a big soccer tournament this weekend. The teams competing are

\begin{itemize}
\tightlist
\item
  Fireflies
\item
  Penguins
\item
  Huskies
\item
  Bluebirds
\end{itemize}

\end{frame}

\begin{frame}{Tournament Structure}
\protect\hypertarget{tournament-structure}{}

There will be three games.

\begin{enumerate}
\tightlist
\item
  Fireflies vs Penguins
\item
  Huskies vs Bluebirds
\item
  Winner of Game 1 vs Winner of Game 2
\end{enumerate}

Each game will have a winner one way or the other (maybe via penalty
kicks or extra time).

\end{frame}

\begin{frame}{Team Strength}
\protect\hypertarget{team-strength}{}

The teams are not all equally good. They each have a `strength'. Here is
their respective strengths

\begin{longtable}[]{@{}cc@{}}
\toprule
Team & Strength\tabularnewline
\midrule
\endhead
Fireflies & 5\tabularnewline
Penguins & 4\tabularnewline
Huskies & 3\tabularnewline
Bluebirds & 1\tabularnewline
\bottomrule
\end{longtable}

\end{frame}

\begin{frame}{Win Probabilities}
\protect\hypertarget{win-probabilities}{}

If a team with strength \(x\) plays a team with strength \(y\), the team
with strength \(x\) will win with probability

\[
\frac{x}{x+ y}
\]

\bigskip

And the team with strength \(y\) will win with probability

\[
\frac{y}{x + y}
\]

\end{frame}

\begin{frame}{Question}
\protect\hypertarget{question}{}

What is the probability that each team will win the tournament?

\begin{itemize}[<+->]
\tightlist
\item
  We will answer this by doing a tree.
\end{itemize}

\end{frame}

\begin{frame}[fragile]{Tournament Tree}
\protect\hypertarget{tournament-tree}{}

\newcommand{\pictext}[3]{
\put(#1, #2){\makebox(0, 0)[b]{#3}}}
\setlength{\unitlength}{0.9pt}
\begin{picture}(350, 150)

\put(175, 0){\makebox(0, 0)[b]{G1}}
\put(175, 12){\circle{4}}\put(173, 13){\line(-2, 1){69}}
\put(177, 13){\line(2, 1){69}}

\put(135, 20){\makebox(0, 0)[b]{F}}

\put(215, 20){\makebox(0, 0)[b]{P}}

\put(105, 35){\makebox(0, 0)[b]{G2}}
\put(105, 47){\circle*{4}}
\put(105, 47){\line(-1, 1){35}}
\put(105, 47){\line(1, 1){35}}

\put(80, 55){\makebox(0, 0)[b]{H}}

\put(130, 55){\makebox(0, 0)[b]{B}}

\put(245, 35){\makebox(0, 0)[b]{G2}}
\put(245, 47){\circle*{4}}
\put(245, 47){\line(-1, 1){35}}
\put(245, 47){\line(1, 1){35}}

\put(220, 55){\makebox(0, 0)[b]{H}}

\put(270, 55){\makebox(0, 0)[b]{B}}

\put(70, 68){\makebox(0, 0)[b]{G3}}
\put(70, 82){\circle*{4}}
\put(70, 82){\line(-1, 2){20}}
%\pictext{50}{125}{$\mathpzc{L}$}
\put(70, 82){\line(1, 2){20}}
%\pictext{90}{125}{$\mathpzc{L}$}

\put(55, 95){\makebox(0, 0)[b]{F}}

\put(85, 95){\makebox(0, 0)[b]{H}}

\put(140, 68){\makebox(0, 0)[b]{G3}}
\put(140, 82){\circle*{4}}
\put(140, 82){\line(-1, 2){20}}
%\pictext{120}{125}{$\mathpzc{L}$}
\put(140, 82){\line(1, 2){20}}
%\pictext{160}{125}{$\mathpzc{W}$}

\put(125, 95){\makebox(0, 0)[b]{F}}

\put(155, 95){\makebox(0, 0)[b]{B}}

\put(210, 68){\makebox(0, 0)[b]{G3}}
\put(210, 82){\circle*{4}}
\put(210, 82){\line(-1, 2){20}}
%\pictext{190}{125}{$\mathpzc{L}$}
\put(210, 82){\line(1, 2){20}}
%\pictext{230}{125}{$\mathpzc{W}$}

\put(195, 95){\makebox(0, 0)[b]{P}}

\put(225, 95){\makebox(0, 0)[b]{H}}

\put(280, 68){\makebox(0, 0)[b]{G3}}
\put(280, 82){\circle*{4}}
\put(280, 82){\line(-1, 2){20}}
%\pictext{260}{125}{$\mathpzc{W}$}
\put(280, 82){\line(1, 2){20}}
%\pictext{300}{125}{$\mathpzc{L}$}

\put(265, 95){\makebox(0, 0)[b]{P}}

\put(295, 95){\makebox(0, 0)[b]{B}}
\end{picture}

\pause

Now we have to add the probabilities to it.

\end{frame}

\begin{frame}[fragile]{Tournament Tree}
\protect\hypertarget{tournament-tree-1}{}

\setlength{\unitlength}{0.9pt}
\begin{picture}(350, 150)

\put(175, 0){\makebox(0, 0)[b]{G1}}
\put(175, 12){\circle{4}}\put(173, 13){\line(-2, 1){69}}
\put(177, 13){\line(2, 1){69}}

\put(135, 20){\makebox(0, 0)[b]{F}}

\put(135, 8){\makebox(0, 0)[b]{$\nicefrac{5}{9}$}}

\put(215, 20){\makebox(0, 0)[b]{P}}

\put(215, 8){\makebox(0, 0)[b]{$\nicefrac{4}{9}$}}

\put(105, 35){\makebox(0, 0)[b]{G2}}
\put(105, 47){\circle*{4}}
\put(105, 47){\line(-1, 1){35}}
\put(105, 47){\line(1, 1){35}}

\put(80, 55){\makebox(0, 0)[b]{H}}

\put(130, 55){\makebox(0, 0)[b]{B}}

\put(245, 35){\makebox(0, 0)[b]{G2}}
\put(245, 47){\circle*{4}}
\put(245, 47){\line(-1, 1){35}}
\put(245, 47){\line(1, 1){35}}

\put(220, 55){\makebox(0, 0)[b]{H}}

\put(270, 55){\makebox(0, 0)[b]{B}}

\put(70, 68){\makebox(0, 0)[b]{G3}}
\put(70, 82){\circle*{4}}
\put(70, 82){\line(-1, 2){20}}
%\pictext{50}{125}{$\mathpzc{L}$}
\put(70, 82){\line(1, 2){20}}
%\pictext{90}{125}{$\mathpzc{L}$}

\put(55, 95){\makebox(0, 0)[b]{F}}

\put(85, 95){\makebox(0, 0)[b]{H}}

\put(140, 68){\makebox(0, 0)[b]{G3}}
\put(140, 82){\circle*{4}}
\put(140, 82){\line(-1, 2){20}}
%\pictext{120}{125}{$\mathpzc{L}$}
\put(140, 82){\line(1, 2){20}}
%\pictext{160}{125}{$\mathpzc{W}$}

\put(125, 95){\makebox(0, 0)[b]{F}}

\put(155, 95){\makebox(0, 0)[b]{B}}

\put(210, 68){\makebox(0, 0)[b]{G3}}
\put(210, 82){\circle*{4}}
\put(210, 82){\line(-1, 2){20}}
%\pictext{190}{125}{$\mathpzc{L}$}
\put(210, 82){\line(1, 2){20}}
%\pictext{230}{125}{$\mathpzc{W}$}

\put(195, 95){\makebox(0, 0)[b]{P}}

\put(225, 95){\makebox(0, 0)[b]{H}}

\put(280, 68){\makebox(0, 0)[b]{G3}}
\put(280, 82){\circle*{4}}
\put(280, 82){\line(-1, 2){20}}
%\pictext{260}{125}{$\mathpzc{W}$}
\put(280, 82){\line(1, 2){20}}
%\pictext{300}{125}{$\mathpzc{L}$}

\put(265, 95){\makebox(0, 0)[b]{P}}

\put(295, 95){\makebox(0, 0)[b]{B}}
\end{picture}

The first game is strength 5 vs strength 4, so the win probability for
the stronger team is \(\nicefrac{5}{5+4}\), i.e., \(\nicefrac{5}{9}\).

\end{frame}

\begin{frame}[fragile]{Tournament Tree}
\protect\hypertarget{tournament-tree-2}{}

\setlength{\unitlength}{0.9pt}
\begin{picture}(350, 150)

\put(175, 0){\makebox(0, 0)[b]{G1}}
\put(175, 12){\circle{4}}\put(173, 13){\line(-2, 1){69}}
\put(177, 13){\line(2, 1){69}}

\put(135, 20){\makebox(0, 0)[b]{F}}

\put(135, 8){\makebox(0, 0)[b]{$\nicefrac{5}{9}$}}

\put(215, 20){\makebox(0, 0)[b]{P}}

\put(215, 8){\makebox(0, 0)[b]{$\nicefrac{4}{9}$}}

\put(105, 35){\makebox(0, 0)[b]{G2}}
\put(105, 47){\circle*{4}}
\put(105, 47){\line(-1, 1){35}}
\put(105, 47){\line(1, 1){35}}

\put(80, 55){\makebox(0, 0)[b]{H}}

\put(80, 43){\makebox(0, 0)[b]{$\nicefrac{3}{4}$}}

\put(130, 55){\makebox(0, 0)[b]{B}}

\put(130, 43){\makebox(0, 0)[b]{$\nicefrac{1}{4}$}}

\put(245, 35){\makebox(0, 0)[b]{G2}}
\put(245, 47){\circle*{4}}
\put(245, 47){\line(-1, 1){35}}
\put(245, 47){\line(1, 1){35}}

\put(220, 55){\makebox(0, 0)[b]{H}}

\put(220, 43){\makebox(0, 0)[b]{$\nicefrac{3}{4}$}}

\put(270, 55){\makebox(0, 0)[b]{B}}

\put(270, 43){\makebox(0, 0)[b]{$\nicefrac{1}{4}$}}

\put(70, 68){\makebox(0, 0)[b]{G3}}
\put(70, 82){\circle*{4}}
\put(70, 82){\line(-1, 2){20}}
%\pictext{50}{125}{$\mathpzc{L}$}
\put(70, 82){\line(1, 2){20}}
%\pictext{90}{125}{$\mathpzc{L}$}

\put(55, 95){\makebox(0, 0)[b]{F}}

\put(85, 95){\makebox(0, 0)[b]{H}}

\put(140, 68){\makebox(0, 0)[b]{G3}}
\put(140, 82){\circle*{4}}
\put(140, 82){\line(-1, 2){20}}
%\pictext{120}{125}{$\mathpzc{L}$}
\put(140, 82){\line(1, 2){20}}
%\pictext{160}{125}{$\mathpzc{W}$}

\put(125, 95){\makebox(0, 0)[b]{F}}

\put(155, 95){\makebox(0, 0)[b]{B}}

\put(210, 68){\makebox(0, 0)[b]{G3}}
\put(210, 82){\circle*{4}}
\put(210, 82){\line(-1, 2){20}}
%\pictext{190}{125}{$\mathpzc{L}$}
\put(210, 82){\line(1, 2){20}}
%\pictext{230}{125}{$\mathpzc{W}$}

\put(195, 95){\makebox(0, 0)[b]{P}}

\put(225, 95){\makebox(0, 0)[b]{H}}

\put(280, 68){\makebox(0, 0)[b]{G3}}
\put(280, 82){\circle*{4}}
\put(280, 82){\line(-1, 2){20}}
%\pictext{260}{125}{$\mathpzc{W}$}
\put(280, 82){\line(1, 2){20}}
%\pictext{300}{125}{$\mathpzc{L}$}

\put(265, 95){\makebox(0, 0)[b]{P}}

\put(295, 95){\makebox(0, 0)[b]{B}}
\end{picture}

The second game is strength 3 vs strength 1, so the win probability for
the stronger team is \(\nicefrac{3}{3+1}\), i.e., \(\nicefrac{3}{4}\).
And it doesn't matter how the first game went - that's the probability
for the second game.

\end{frame}

\begin{frame}[fragile]{Tournament Tree}
\protect\hypertarget{tournament-tree-3}{}

\setlength{\unitlength}{0.9pt}
\begin{picture}(350, 150)

\put(175, 0){\makebox(0, 0)[b]{G1}}
\put(175, 12){\circle{4}}\put(173, 13){\line(-2, 1){69}}
\put(177, 13){\line(2, 1){69}}

\put(135, 20){\makebox(0, 0)[b]{F}}

\put(135, 8){\makebox(0, 0)[b]{$\nicefrac{5}{9}$}}

\put(215, 20){\makebox(0, 0)[b]{P}}

\put(215, 8){\makebox(0, 0)[b]{$\nicefrac{4}{9}$}}

\put(105, 35){\makebox(0, 0)[b]{G2}}
\put(105, 47){\circle*{4}}
\put(105, 47){\line(-1, 1){35}}
\put(105, 47){\line(1, 1){35}}

\put(80, 55){\makebox(0, 0)[b]{H}}

\put(80, 43){\makebox(0, 0)[b]{$\nicefrac{3}{4}$}}

\put(130, 55){\makebox(0, 0)[b]{B}}

\put(130, 43){\makebox(0, 0)[b]{$\nicefrac{1}{4}$}}

\put(245, 35){\makebox(0, 0)[b]{G2}}
\put(245, 47){\circle*{4}}
\put(245, 47){\line(-1, 1){35}}
\put(245, 47){\line(1, 1){35}}

\put(220, 55){\makebox(0, 0)[b]{H}}

\put(220, 43){\makebox(0, 0)[b]{$\nicefrac{3}{4}$}}

\put(270, 55){\makebox(0, 0)[b]{B}}

\put(270, 43){\makebox(0, 0)[b]{$\nicefrac{1}{4}$}}

\put(70, 68){\makebox(0, 0)[b]{G3}}
\put(70, 82){\circle*{4}}
\put(70, 82){\line(-1, 2){20}}
%\pictext{50}{125}{$\mathpzc{L}$}
\put(70, 82){\line(1, 2){20}}
%\pictext{90}{125}{$\mathpzc{L}$}

\put(55, 95){\makebox(0, 0)[b]{F}}

\put(55, 83){\makebox(0, 0)[b]{$\nicefrac{5}{8}$}}

\put(85, 95){\makebox(0, 0)[b]{H}}

\put(85, 83){\makebox(0, 0)[b]{$\nicefrac{3}{8}$}}

\put(140, 68){\makebox(0, 0)[b]{G3}}
\put(140, 82){\circle*{4}}
\put(140, 82){\line(-1, 2){20}}
%\pictext{120}{125}{$\mathpzc{L}$}
\put(140, 82){\line(1, 2){20}}
%\pictext{160}{125}{$\mathpzc{W}$}

\put(125, 95){\makebox(0, 0)[b]{F}}

\put(125, 83){\makebox(0, 0)[b]{$\nicefrac{5}{6}$}}

\put(155, 95){\makebox(0, 0)[b]{B}}

\put(155, 83){\makebox(0, 0)[b]{$\nicefrac{1}{6}$}}

\put(210, 68){\makebox(0, 0)[b]{G3}}
\put(210, 82){\circle*{4}}
\put(210, 82){\line(-1, 2){20}}
%\pictext{190}{125}{$\mathpzc{L}$}
\put(210, 82){\line(1, 2){20}}
%\pictext{230}{125}{$\mathpzc{W}$}

\put(195, 95){\makebox(0, 0)[b]{P}}

\put(195, 83){\makebox(0, 0)[b]{$\nicefrac{4}{7}$}}

\put(225, 95){\makebox(0, 0)[b]{H}}

\put(225, 83){\makebox(0, 0)[b]{$\nicefrac{3}{7}$}}

\put(280, 68){\makebox(0, 0)[b]{G3}}
\put(280, 82){\circle*{4}}
\put(280, 82){\line(-1, 2){20}}
%\pictext{260}{125}{$\mathpzc{W}$}
\put(280, 82){\line(1, 2){20}}
%\pictext{300}{125}{$\mathpzc{L}$}

\put(265, 95){\makebox(0, 0)[b]{P}}

\put(265, 83){\makebox(0, 0)[b]{$\nicefrac{4}{5}$}}

\put(295, 95){\makebox(0, 0)[b]{B}}

\put(295, 83){\makebox(0, 0)[b]{$\nicefrac{1}{5}$}}
\end{picture}

And now for each possible match up in game 3, we apply the formula to
get the win probability for each team.

\end{frame}

\begin{frame}[fragile]{Tournament Tree}
\protect\hypertarget{tournament-tree-4}{}

\setlength{\unitlength}{0.9pt}
\begin{picture}(350, 150)

\put(175, 0){\makebox(0, 0)[b]{G1}}
\put(175, 12){\circle{4}}\put(173, 13){\line(-2, 1){69}}
\put(177, 13){\line(2, 1){69}}

\put(135, 20){\makebox(0, 0)[b]{F}}

\put(135, 8){\makebox(0, 0)[b]{$\nicefrac{5}{9}$}}

\put(215, 20){\makebox(0, 0)[b]{P}}

\put(215, 8){\makebox(0, 0)[b]{$\nicefrac{4}{9}$}}

\put(105, 35){\makebox(0, 0)[b]{G2}}
\put(105, 47){\circle*{4}}
\put(105, 47){\line(-1, 1){35}}
\put(105, 47){\line(1, 1){35}}

\put(80, 55){\makebox(0, 0)[b]{H}}

\put(80, 43){\makebox(0, 0)[b]{$\nicefrac{3}{4}$}}

\put(130, 55){\makebox(0, 0)[b]{B}}

\put(130, 43){\makebox(0, 0)[b]{$\nicefrac{1}{4}$}}

\put(245, 35){\makebox(0, 0)[b]{G2}}
\put(245, 47){\circle*{4}}
\put(245, 47){\line(-1, 1){35}}
\put(245, 47){\line(1, 1){35}}

\put(220, 55){\makebox(0, 0)[b]{H}}

\put(220, 43){\makebox(0, 0)[b]{$\nicefrac{3}{4}$}}

\put(270, 55){\makebox(0, 0)[b]{B}}

\put(270, 43){\makebox(0, 0)[b]{$\nicefrac{1}{4}$}}

\put(70, 68){\makebox(0, 0)[b]{G3}}
\put(70, 82){\circle*{4}}
\put(70, 82){\line(-1, 2){20}}
%\pictext{50}{125}{$\mathpzc{L}$}
\put(70, 82){\line(1, 2){20}}
%\pictext{90}{125}{$\mathpzc{L}$}

\put(55, 95){\makebox(0, 0)[b]{F}}

\put(55, 83){\makebox(0, 0)[b]{$\nicefrac{5}{8}$}}

\put(85, 95){\makebox(0, 0)[b]{H}}

\put(85, 83){\makebox(0, 0)[b]{$\nicefrac{3}{8}$}}

\put(140, 68){\makebox(0, 0)[b]{G3}}
\put(140, 82){\circle*{4}}
\put(140, 82){\line(-1, 2){20}}
%\pictext{120}{125}{$\mathpzc{L}$}
\put(140, 82){\line(1, 2){20}}
%\pictext{160}{125}{$\mathpzc{W}$}

\put(125, 95){\makebox(0, 0)[b]{F}}

\put(125, 83){\makebox(0, 0)[b]{$\nicefrac{5}{6}$}}

\put(155, 95){\makebox(0, 0)[b]{B}}

\put(155, 83){\makebox(0, 0)[b]{$\nicefrac{1}{6}$}}

\put(210, 68){\makebox(0, 0)[b]{G3}}
\put(210, 82){\circle*{4}}
\put(210, 82){\line(-1, 2){20}}
%\pictext{190}{125}{$\mathpzc{L}$}
\put(210, 82){\line(1, 2){20}}

\put(230, 125){\makebox(0, 0)[b]{$\frac{1}{7}$}}

\put(195, 95){\makebox(0, 0)[b]{P}}

\put(195, 83){\makebox(0, 0)[b]{$\nicefrac{4}{7}$}}

\put(225, 95){\makebox(0, 0)[b]{H}}

\put(225, 83){\makebox(0, 0)[b]{$\nicefrac{3}{7}$}}

\put(280, 68){\makebox(0, 0)[b]{G3}}
\put(280, 82){\circle*{4}}
\put(280, 82){\line(-1, 2){20}}
%\pictext{260}{125}{$\mathpzc{W}$}
\put(280, 82){\line(1, 2){20}}
%\pictext{300}{125}{$\mathpzc{L}$}

\put(265, 95){\makebox(0, 0)[b]{P}}

\put(265, 83){\makebox(0, 0)[b]{$\nicefrac{4}{5}$}}

\put(295, 95){\makebox(0, 0)[b]{B}}

\put(295, 83){\makebox(0, 0)[b]{$\nicefrac{1}{5}$}}
\end{picture}

\begin{itemize}
\tightlist
\item
  The probability of each completed branch is the product of each of the
  smaller branches.
\item
  So the one I've marked is
  \(\frac{4}{9} \times \frac{3}{4} \times \frac{3}{7} = \frac{1}{7}\).
\end{itemize}

\end{frame}

\begin{frame}[fragile]{Tournament Tree}
\protect\hypertarget{tournament-tree-5}{}

\setlength{\unitlength}{0.9pt}
\begin{picture}(350, 150)

\put(175, 0){\makebox(0, 0)[b]{G1}}
\put(175, 12){\circle{4}}\put(173, 13){\line(-2, 1){69}}
\put(177, 13){\line(2, 1){69}}

\put(135, 20){\makebox(0, 0)[b]{F}}

\put(135, 8){\makebox(0, 0)[b]{$\nicefrac{5}{9}$}}

\put(215, 20){\makebox(0, 0)[b]{P}}

\put(215, 8){\makebox(0, 0)[b]{$\nicefrac{4}{9}$}}

\put(105, 35){\makebox(0, 0)[b]{G2}}
\put(105, 47){\circle*{4}}
\put(105, 47){\line(-1, 1){35}}
\put(105, 47){\line(1, 1){35}}

\put(80, 55){\makebox(0, 0)[b]{H}}

\put(80, 43){\makebox(0, 0)[b]{$\nicefrac{3}{4}$}}

\put(130, 55){\makebox(0, 0)[b]{B}}

\put(130, 43){\makebox(0, 0)[b]{$\nicefrac{1}{4}$}}

\put(245, 35){\makebox(0, 0)[b]{G2}}
\put(245, 47){\circle*{4}}
\put(245, 47){\line(-1, 1){35}}
\put(245, 47){\line(1, 1){35}}

\put(220, 55){\makebox(0, 0)[b]{H}}

\put(220, 43){\makebox(0, 0)[b]{$\nicefrac{3}{4}$}}

\put(270, 55){\makebox(0, 0)[b]{B}}

\put(270, 43){\makebox(0, 0)[b]{$\nicefrac{1}{4}$}}

\put(70, 68){\makebox(0, 0)[b]{G3}}
\put(70, 82){\circle*{4}}
\put(70, 82){\line(-1, 2){20}}

\put(50, 125){\makebox(0, 0)[b]{$\frac{25}{96}$}}
\put(70, 82){\line(1, 2){20}}

\put(90, 125){\makebox(0, 0)[b]{$\frac{5}{32}$}}

\put(55, 95){\makebox(0, 0)[b]{F}}

\put(55, 83){\makebox(0, 0)[b]{$\nicefrac{5}{8}$}}

\put(85, 95){\makebox(0, 0)[b]{H}}

\put(85, 83){\makebox(0, 0)[b]{$\nicefrac{3}{8}$}}

\put(140, 68){\makebox(0, 0)[b]{G3}}
\put(140, 82){\circle*{4}}
\put(140, 82){\line(-1, 2){20}}

\put(120, 125){\makebox(0, 0)[b]{$\frac{25}{216}$}}
\put(140, 82){\line(1, 2){20}}

\put(160, 125){\makebox(0, 0)[b]{$\frac{5}{216}$}}

\put(125, 95){\makebox(0, 0)[b]{F}}

\put(125, 83){\makebox(0, 0)[b]{$\nicefrac{5}{6}$}}

\put(155, 95){\makebox(0, 0)[b]{B}}

\put(155, 83){\makebox(0, 0)[b]{$\nicefrac{1}{6}$}}

\put(210, 68){\makebox(0, 0)[b]{G3}}
\put(210, 82){\circle*{4}}
\put(210, 82){\line(-1, 2){20}}

\put(190, 125){\makebox(0, 0)[b]{$\frac{4}{21}$}}
\put(210, 82){\line(1, 2){20}}

\put(230, 125){\makebox(0, 0)[b]{$\frac{1}{7}$}}

\put(195, 95){\makebox(0, 0)[b]{P}}

\put(195, 83){\makebox(0, 0)[b]{$\nicefrac{4}{7}$}}

\put(225, 95){\makebox(0, 0)[b]{H}}

\put(225, 83){\makebox(0, 0)[b]{$\nicefrac{3}{7}$}}

\put(280, 68){\makebox(0, 0)[b]{G3}}
\put(280, 82){\circle*{4}}
\put(280, 82){\line(-1, 2){20}}

\put(260, 125){\makebox(0, 0)[b]{$\frac{4}{45}$}}
\put(280, 82){\line(1, 2){20}}

\put(300, 125){\makebox(0, 0)[b]{$\frac{1}{45}$}}

\put(265, 95){\makebox(0, 0)[b]{P}}

\put(265, 83){\makebox(0, 0)[b]{$\nicefrac{4}{5}$}}

\put(295, 95){\makebox(0, 0)[b]{B}}

\put(295, 83){\makebox(0, 0)[b]{$\nicefrac{1}{5}$}}
\end{picture}

I've included all the others - they usually don't cancel as nicely as
that one.

\end{frame}

\begin{frame}{Tournament Table}
\protect\hypertarget{tournament-table}{}

It might be easier to see the results in a table

\begin{longtable}[]{@{}cccc@{}}
\toprule
Winner & Runner-Up & Probability & Approx\tabularnewline
\midrule
\endhead
Fireflies & Huskies & \(\frac{25}{96}\) & 0.260\tabularnewline
Huskies & Fireflies & \(\frac{5}{32}\) & 0.156\tabularnewline
Fireflies & Bluebirds & \(\frac{25}{216}\) & 0.116\tabularnewline
Bluebirds & Fireflies & \(\frac{5}{216}\) & 0.023\tabularnewline
Penguins & Huskies & \(\frac{4}{21}\) & 0.190\tabularnewline
Huskies & Penguins & \(\frac{1}{7}\) & 0.143\tabularnewline
Penguins & Bluebirds & \(\frac{4}{45}\) & 0.089\tabularnewline
Bluebirds & Penguins & \(\frac{1}{45}\) & 0.022\tabularnewline
\bottomrule
\end{longtable}

\end{frame}

\begin{frame}{Tournament Table}
\protect\hypertarget{tournament-table-1}{}

And we can rearrange that so the rows where each team wins are adjacent.

\begin{longtable}[]{@{}cccc@{}}
\toprule
Winner & Runner-Up & Probability & Approx\tabularnewline
\midrule
\endhead
Fireflies & Huskies & \(\frac{25}{96}\) & 0.260\tabularnewline
Fireflies & Bluebirds & \(\frac{25}{216}\) & 0.116\tabularnewline
Huskies & Fireflies & \(\frac{5}{32}\) & 0.156\tabularnewline
Huskies & Penguins & \(\frac{1}{7}\) & 0.143\tabularnewline
Penguins & Huskies & \(\frac{4}{21}\) & 0.190\tabularnewline
Penguins & Bluebirds & \(\frac{4}{45}\) & 0.089\tabularnewline
Bluebirds & Fireflies & \(\frac{5}{216}\) & 0.023\tabularnewline
Bluebirds & Penguins & \(\frac{1}{45}\) & 0.022\tabularnewline
\bottomrule
\end{longtable}

\end{frame}

\begin{frame}{Tournament Table}
\protect\hypertarget{tournament-table-2}{}

And then just adding up the probabilities for the two ways each team can
win, we get the actual probabilities of each win. (I'm just doing the
decimals now.)

\begin{longtable}[]{@{}cc@{}}
\toprule
Winner & Approx Probability\tabularnewline
\midrule
\endhead
Fireflies & 0.376\tabularnewline
Huskies & 0.299\tabularnewline
Penguins & 0.279\tabularnewline
Bluebirds & 0.045\tabularnewline
\bottomrule
\end{longtable}

(Those numbers don't sum to 1 precisely because of rounding.)

\end{frame}

\begin{frame}{For Next Time}
\protect\hypertarget{for-next-time}{}

\begin{itemize}
\tightlist
\item
  We will talk about a well known piece of fallacious reasoning: the
  gamblers' fallacy.
\end{itemize}

\end{frame}

\end{document}
