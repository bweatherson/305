% Options for packages loaded elsewhere
\PassOptionsToPackage{unicode}{hyperref}
\PassOptionsToPackage{hyphens}{url}
%
\documentclass[
  ignorenonframetext,
]{beamer}
\usepackage{pgfpages}
\setbeamertemplate{caption}[numbered]
\setbeamertemplate{caption label separator}{: }
\setbeamercolor{caption name}{fg=normal text.fg}
\beamertemplatenavigationsymbolsempty
% Prevent slide breaks in the middle of a paragraph
\widowpenalties 1 10000
\raggedbottom
\setbeamertemplate{part page}{
  \centering
  \begin{beamercolorbox}[sep=16pt,center]{part title}
    \usebeamerfont{part title}\insertpart\par
  \end{beamercolorbox}
}
\setbeamertemplate{section page}{
  \centering
  \begin{beamercolorbox}[sep=12pt,center]{part title}
    \usebeamerfont{section title}\insertsection\par
  \end{beamercolorbox}
}
\setbeamertemplate{subsection page}{
  \centering
  \begin{beamercolorbox}[sep=8pt,center]{part title}
    \usebeamerfont{subsection title}\insertsubsection\par
  \end{beamercolorbox}
}
\AtBeginPart{
  \frame{\partpage}
}
\AtBeginSection{
  \ifbibliography
  \else
    \frame{\sectionpage}
  \fi
}
\AtBeginSubsection{
  \frame{\subsectionpage}
}
\usepackage{lmodern}
\usepackage{amssymb,amsmath}
\usepackage{ifxetex,ifluatex}
\ifnum 0\ifxetex 1\fi\ifluatex 1\fi=0 % if pdftex
  \usepackage[T1]{fontenc}
  \usepackage[utf8]{inputenc}
  \usepackage{textcomp} % provide euro and other symbols
\else % if luatex or xetex
  \usepackage{unicode-math}
  \defaultfontfeatures{Scale=MatchLowercase}
  \defaultfontfeatures[\rmfamily]{Ligatures=TeX,Scale=1}
\fi
% Use upquote if available, for straight quotes in verbatim environments
\IfFileExists{upquote.sty}{\usepackage{upquote}}{}
\IfFileExists{microtype.sty}{% use microtype if available
  \usepackage[]{microtype}
  \UseMicrotypeSet[protrusion]{basicmath} % disable protrusion for tt fonts
}{}
\makeatletter
\@ifundefined{KOMAClassName}{% if non-KOMA class
  \IfFileExists{parskip.sty}{%
    \usepackage{parskip}
  }{% else
    \setlength{\parindent}{0pt}
    \setlength{\parskip}{6pt plus 2pt minus 1pt}}
}{% if KOMA class
  \KOMAoptions{parskip=half}}
\makeatother
\usepackage{xcolor}
\IfFileExists{xurl.sty}{\usepackage{xurl}}{} % add URL line breaks if available
\IfFileExists{bookmark.sty}{\usepackage{bookmark}}{\usepackage{hyperref}}
\hypersetup{
  pdftitle={305 Lecture 01 - Getting Started},
  pdfauthor={Brian Weatherson},
  hidelinks,
  pdfcreator={LaTeX via pandoc}}
\urlstyle{same} % disable monospaced font for URLs
\newif\ifbibliography
\usepackage{graphicx,grffile}
\makeatletter
\def\maxwidth{\ifdim\Gin@nat@width>\linewidth\linewidth\else\Gin@nat@width\fi}
\def\maxheight{\ifdim\Gin@nat@height>\textheight\textheight\else\Gin@nat@height\fi}
\makeatother
% Scale images if necessary, so that they will not overflow the page
% margins by default, and it is still possible to overwrite the defaults
% using explicit options in \includegraphics[width, height, ...]{}
\setkeys{Gin}{width=\maxwidth,height=\maxheight,keepaspectratio}
% Set default figure placement to htbp
\makeatletter
\def\fps@figure{htbp}
\makeatother
\setlength{\emergencystretch}{3em} % prevent overfull lines
\providecommand{\tightlist}{%
  \setlength{\itemsep}{0pt}\setlength{\parskip}{0pt}}
\setcounter{secnumdepth}{-\maxdimen} % remove section numbering
\let\Tiny=\tiny

 \setbeamertemplate{navigation symbols}{} 

% \usetheme{Madrid}
 \usetheme[numbering=none, progressbar=foot]{metropolis}
 \usecolortheme{wolverine}
 \usepackage{color}
 \usepackage{MnSymbol}
% \usepackage{movie15}

\usepackage{amssymb}% http://ctan.org/pkg/amssymb
\usepackage{pifont}% http://ctan.org/pkg/pifont
\newcommand{\cmark}{\ding{51}}%
\newcommand{\xmark}{\ding{55}}%

\DeclareSymbolFont{symbolsC}{U}{txsyc}{m}{n}
\DeclareMathSymbol{\boxright}{\mathrel}{symbolsC}{128}
\DeclareMathAlphabet{\mathpzc}{OT1}{pzc}{m}{it}


% \usepackage{tikz-qtree}
% \usepackage{markdown}
% \usepackage{prooftrees}
% \forestset{not line numbering, close with = x}
% Allow for easy commas inside trees
\renewcommand{\,}{\text{, }}


\usepackage{tabulary}

\usepackage{open-logic-config}

\setlength{\parskip}{1ex plus 0.5ex minus 0.2ex}

\AtBeginSection[]
{
\begin{frame}
	\Huge{\color{darkblue} \insertsection}
\end{frame}
}

\renewenvironment*{quote}	
	{\list{}{\rightmargin   \leftmargin} \item } 	
	{\endlist }

\definecolor{darkgreen}{rgb}{0,0.7,0}
\definecolor{darkblue}{rgb}{0,0,0.8}

\newcommand{\starttab}{\begin{center}
\vspace{6pt}
\begin{tabular}}

\newcommand{\stoptab}{\end{tabular}
\vspace{6pt}
\end{center}
\noindent}


\newcommand{\sif}{\rightarrow}
\newcommand{\siff}{\leftrightarrow}
\newcommand{\EF}{\end{frame}}


\newcommand{\TreeStart}[1]{
%\end{frame}
\begin{frame}
\begin{center}
\begin{tikzpicture}[scale=#1]
\tikzset{every tree node/.style={align=center,anchor=north}}
%\Tree
}

\newcommand{\TreeEnd}{
\end{tikzpicture}
%\end{center}
}

\newcommand{\DisplayArg}[2]{
\begin{enumerate}
{#1}
\end{enumerate}
\vspace{-6pt}
\hrulefill

%\hspace{14pt} #2
%{\addtolength{\leftskip}{14pt} #2}
\begin{quote}
{\normalfont #2}
\end{quote}
\vspace{12pt}
}

\newenvironment{ProofTree}[1][1]{
\begin{center}
\begin{tikzpicture}[scale=#1]
\tikzset{every tree node/.style={align=center,anchor=south}}
}
{
\end{tikzpicture}
\end{center}
}

\newcommand{\TreeFrame}[2]{
\begin{columns}[c]
\column{0.5\textwidth}
\begin{center}
\begin{prooftree}{}
#1
\end{prooftree}
\end{center}
\column{0.45\textwidth}
%\begin{markdown}
#2
%\end{markdown}
\end{columns}
}

\newcommand{\ScaledTreeFrame}[3]{
\begin{columns}[c]
\column{0.5\textwidth}
\begin{center}
\scalebox{#1}{
\begin{prooftree}{}
#2
\end{prooftree}
}
\end{center}
\column{0.45\textwidth}
%\begin{markdown}
#3
%\end{markdown}
\end{columns}
}

\usepackage[bb=boondox]{mathalfa}
\DeclareMathAlphabet{\mathbx}{U}{BOONDOX-ds}{m}{n}
\SetMathAlphabet{\mathbx}{bold}{U}{BOONDOX-ds}{b}{n}
\DeclareMathAlphabet{\mathbbx} {U}{BOONDOX-ds}{b}{n}

\RequirePackage{bussproofs}
\RequirePackage[tableaux]{prooftrees}

\newenvironment{oltableau}{\center\tableau{}} %wff format={anchor = base west}}}
       {\endtableau\endcenter}
       
\newcommand{\formula}[1]{$#1$}

\usepackage{tabulary}
\usepackage{booktabs}

\def\begincols{\begin{columns}}
\def\begincol{\begin{column}}
\def\endcol{\end{column}}
\def\endcols{\end{columns}}

\usepackage[italic]{mathastext}
\usepackage{nicefrac}

\definecolor{mygreen}{RGB}{0, 100, 0}
\definecolor{mypink2}{RGB}{219, 48, 122}
\definecolor{dodgerblue}{RGB}{30,144,255}

\def\True{\textcolor{dodgerblue}{\text{T}}}
\def\False{\textcolor{red}{\text{F}}}

\title{305 Lecture 01 - Getting Started}
\author{Brian Weatherson}
\date{July 1, 2020}

\begin{document}
\frame{\titlepage}

\begin{frame}{Aim of Course}
\protect\hypertarget{aim-of-course}{}

Introductory survey of some formal methods that are of broad
philosophical use.

\end{frame}

\begin{frame}{Three Sections}
\protect\hypertarget{three-sections}{}

\begin{enumerate}[<+->]
\tightlist
\item
  Propositional Logic
\item
  Probability and Statistical Reasoning
\item
  Modal Logic and Conditionals
\end{enumerate}

\end{frame}

\begin{frame}{Propositional Logic}
\protect\hypertarget{propositional-logic}{}

\begin{itemize}
\tightlist
\item
  This is the logic of sentences that can be true or false, and that can
  combine to form longer sentences.
\item
  So as well as looking at simple sentences, like \emph{Nadia sings}, we
  will look at sentences that are built from simple sentences.
\item
  Examples of such sentences are \emph{Nadia doesn't sing}, \emph{Nadia
  sings and Bethany dances}, and \emph{If Nadia sings, Simone sleeps}.
\end{itemize}

\end{frame}

\begin{frame}{Probability and Statistical Reasoning}
\protect\hypertarget{probability-and-statistical-reasoning}{}

\begin{itemize}[<+->]
\tightlist
\item
  Sometimes we can't infer that a conclusion is definitely true, but we
  can infer that it is probably true.
\item
  We will look at some tools for regimenting how and when we make such
  inference.
\end{itemize}

\end{frame}

\begin{frame}{Modal Logic}
\protect\hypertarget{modal-logic}{}

This is the logic of `must' and `might'. It has as many applications as
there are interpretations of `must' and `might'. The primary
interpretations we'll look at are:

\begin{itemize}[<+->]
\tightlist
\item
  Metaphysical
\item
  Epistemological\\
\item
  Moral
\end{itemize}

\end{frame}

\begin{frame}{Textbooks}
\protect\hypertarget{textbooks}{}

There are three - all of them available through Canvas.

\begin{enumerate}
\tightlist
\item
  \emph{The Carnap Book} by Graham Leach-Krouse
\item
  \emph{Odds and Ends} by Jonathan Weisberg
\item
  \emph{Boxes and Diamonds, Ann Arbor remix}
\end{enumerate}

The three books are for the three parts of the course.

\end{frame}

\begin{frame}{The Carnap Book}
\protect\hypertarget{the-carnap-book}{}

\begin{figure}
\centering
\includegraphics[width=\textwidth,height=0.8\textheight]{../images/0_1_a_Carnap.png}
\caption{\url{https://carnap.io}}
\end{figure}

\end{frame}

\begin{frame}{Registering with Carnap}
\protect\hypertarget{registering-with-carnap}{}

\begin{figure}
\centering
\includegraphics[width=\textwidth,height=0.8\textheight]{../images/0_1_a_Carnap_Need_Register.png}
\caption{One more registration step}
\end{figure}

\end{frame}

\begin{frame}

\begin{figure}
\centering
\includegraphics[width=\textwidth,height=0.8\textheight]{../images/0_1_a_Registration.png}
\caption{Register in the right course}
\end{figure}

\end{frame}

\begin{frame}{Odds and Ends}
\protect\hypertarget{odds-and-ends}{}

\begin{figure}
\centering
\includegraphics[width=\textwidth,height=0.8\textheight]{../images/0_1_a_Odds_and_Ends.png}
\caption{\url{https://jonathanweisberg.org/vip/}}
\end{figure}

\end{frame}

\begin{frame}{Boxes and Diamonds}
\protect\hypertarget{boxes-and-diamonds}{}

\begin{figure}
\centering
\includegraphics[width=\textwidth,height=0.8\textheight]{../images/0_1_a_Boxes_and_Diamonds.png}
\caption{\url{https://bd.openlogicproject.org}}
\end{figure}

\end{frame}

\begin{frame}

\begin{figure}
\centering
\includegraphics[width=\textwidth,height=0.8\textheight]{../images/0_1_a_Boxes_and_Diamonds_AA.png}
\caption{Boxes and Diamonds - Ann Arbor}
\end{figure}

\end{frame}

\begin{frame}{Logistics}
\protect\hypertarget{logistics}{}

\begin{itemize}
\tightlist
\item
  These lectures are going to be very short.
\item
  That's in part because it's really hard to retain focus through a long
  logic video, and in part because it's easier to manage uploads and
  downloads with smaller files.
\item
  So we'll typically have somewhere between 6 and 10 `lectures' each
  week, though each will be 5 to 15 minutes.
\end{itemize}

\end{frame}

\begin{frame}{Access}
\protect\hypertarget{access}{}

\begin{itemize}
\tightlist
\item
  These slides don't have captioning.
\item
  But the script I'm reading off is available on Canvas, and that should
  be more reliable than automated captions.
\item
  The script also says where the slide jumps are, so if you don't want
  to listen to me, you can just read along the script plus the slides.
\end{itemize}

\end{frame}

\begin{frame}{Assessment}
\protect\hypertarget{assessment}{}

\begin{itemize}
\tightlist
\item
  The primary assessment will be weekly assignments, which will be
  administered through Canvas.
\item
  These are already all posted, and they will be due each week on Friday
  at 5pm.
\item
  There will also be an end of term exam, also through Canvas.
\end{itemize}

\end{frame}

\begin{frame}{Office Hours}
\protect\hypertarget{office-hours}{}

\begin{itemize}
\tightlist
\item
  The lectures as you can tell are all going to be posted, and can be
  viewed at any time.
\item
  There aren't any official discussion sections.
\item
  But we will need to have some time where we go over stuff.
\item
  I'm planning to have office hours twice a week: Tuesday and Thursday
  10-12. But I'll also be available by appointment if those times don't
  work, as I'm sure they won't for some people.
\item
  We'll start with these on Zoom, but maybe it will turn out to be more
  productive to use something else.
\end{itemize}

\end{frame}

\begin{frame}{For Next Time}
\protect\hypertarget{for-next-time}{}

We'll start on saying what logic is, and why we are studying it.

\end{frame}

\end{document}
