% Options for packages loaded elsewhere
\PassOptionsToPackage{unicode}{hyperref}
\PassOptionsToPackage{hyphens}{url}
%
\documentclass[
  ignorenonframetext,
]{beamer}
\usepackage{pgfpages}
\setbeamertemplate{caption}[numbered]
\setbeamertemplate{caption label separator}{: }
\setbeamercolor{caption name}{fg=normal text.fg}
\beamertemplatenavigationsymbolsempty
% Prevent slide breaks in the middle of a paragraph
\widowpenalties 1 10000
\raggedbottom
\setbeamertemplate{part page}{
  \centering
  \begin{beamercolorbox}[sep=16pt,center]{part title}
    \usebeamerfont{part title}\insertpart\par
  \end{beamercolorbox}
}
\setbeamertemplate{section page}{
  \centering
  \begin{beamercolorbox}[sep=12pt,center]{part title}
    \usebeamerfont{section title}\insertsection\par
  \end{beamercolorbox}
}
\setbeamertemplate{subsection page}{
  \centering
  \begin{beamercolorbox}[sep=8pt,center]{part title}
    \usebeamerfont{subsection title}\insertsubsection\par
  \end{beamercolorbox}
}
\AtBeginPart{
  \frame{\partpage}
}
\AtBeginSection{
  \ifbibliography
  \else
    \frame{\sectionpage}
  \fi
}
\AtBeginSubsection{
  \frame{\subsectionpage}
}
\usepackage{lmodern}
\usepackage{amssymb,amsmath}
\usepackage{ifxetex,ifluatex}
\ifnum 0\ifxetex 1\fi\ifluatex 1\fi=0 % if pdftex
  \usepackage[T1]{fontenc}
  \usepackage[utf8]{inputenc}
  \usepackage{textcomp} % provide euro and other symbols
\else % if luatex or xetex
  \usepackage{unicode-math}
  \defaultfontfeatures{Scale=MatchLowercase}
  \defaultfontfeatures[\rmfamily]{Ligatures=TeX,Scale=1}
\fi
% Use upquote if available, for straight quotes in verbatim environments
\IfFileExists{upquote.sty}{\usepackage{upquote}}{}
\IfFileExists{microtype.sty}{% use microtype if available
  \usepackage[]{microtype}
  \UseMicrotypeSet[protrusion]{basicmath} % disable protrusion for tt fonts
}{}
\makeatletter
\@ifundefined{KOMAClassName}{% if non-KOMA class
  \IfFileExists{parskip.sty}{%
    \usepackage{parskip}
  }{% else
    \setlength{\parindent}{0pt}
    \setlength{\parskip}{6pt plus 2pt minus 1pt}}
}{% if KOMA class
  \KOMAoptions{parskip=half}}
\makeatother
\usepackage{xcolor}
\IfFileExists{xurl.sty}{\usepackage{xurl}}{} % add URL line breaks if available
\IfFileExists{bookmark.sty}{\usepackage{bookmark}}{\usepackage{hyperref}}
\hypersetup{
  pdftitle={305 Lecture 37 - Utilities},
  pdfauthor={Brian Weatherson},
  hidelinks,
  pdfcreator={LaTeX via pandoc}}
\urlstyle{same} % disable monospaced font for URLs
\newif\ifbibliography
\usepackage{longtable,booktabs}
\usepackage{caption}
% Make caption package work with longtable
\makeatletter
\def\fnum@table{\tablename~\thetable}
\makeatother
\setlength{\emergencystretch}{3em} % prevent overfull lines
\providecommand{\tightlist}{%
  \setlength{\itemsep}{0pt}\setlength{\parskip}{0pt}}
\setcounter{secnumdepth}{-\maxdimen} % remove section numbering
\let\Tiny=\tiny

 \setbeamertemplate{navigation symbols}{} 

% \usetheme{Madrid}
 \usetheme[numbering=none, progressbar=foot]{metropolis}
 \usecolortheme{wolverine}
 \usepackage{color}
 \usepackage{MnSymbol}
% \usepackage{movie15}

\usepackage{amssymb}% http://ctan.org/pkg/amssymb
\usepackage{pifont}% http://ctan.org/pkg/pifont
\newcommand{\cmark}{\ding{51}}%
\newcommand{\xmark}{\ding{55}}%

\DeclareSymbolFont{symbolsC}{U}{txsyc}{m}{n}
\DeclareMathSymbol{\boxright}{\mathrel}{symbolsC}{128}
\DeclareMathAlphabet{\mathpzc}{OT1}{pzc}{m}{it}


% \usepackage{tikz-qtree}
% \usepackage{markdown}
% \usepackage{prooftrees}
% \forestset{not line numbering, close with = x}
% Allow for easy commas inside trees
\renewcommand{\,}{\text{, }}


\usepackage{tabulary}

\usepackage{open-logic-config}

\setlength{\parskip}{1ex plus 0.5ex minus 0.2ex}

\AtBeginSection[]
{
\begin{frame}
	\Huge{\color{darkblue} \insertsection}
\end{frame}
}

\renewenvironment*{quote}	
	{\list{}{\rightmargin   \leftmargin} \item } 	
	{\endlist }

\definecolor{darkgreen}{rgb}{0,0.7,0}
\definecolor{darkblue}{rgb}{0,0,0.8}

\newcommand{\starttab}{\begin{center}
\vspace{6pt}
\begin{tabular}}

\newcommand{\stoptab}{\end{tabular}
\vspace{6pt}
\end{center}
\noindent}


\newcommand{\sif}{\rightarrow}
\newcommand{\siff}{\leftrightarrow}
\newcommand{\EF}{\end{frame}}


\newcommand{\TreeStart}[1]{
%\end{frame}
\begin{frame}
\begin{center}
\begin{tikzpicture}[scale=#1]
\tikzset{every tree node/.style={align=center,anchor=north}}
%\Tree
}

\newcommand{\TreeEnd}{
\end{tikzpicture}
%\end{center}
}

\newcommand{\DisplayArg}[2]{
\begin{enumerate}
{#1}
\end{enumerate}
\vspace{-6pt}
\hrulefill

%\hspace{14pt} #2
%{\addtolength{\leftskip}{14pt} #2}
\begin{quote}
{\normalfont #2}
\end{quote}
\vspace{12pt}
}

\newenvironment{ProofTree}[1][1]{
\begin{center}
\begin{tikzpicture}[scale=#1]
\tikzset{every tree node/.style={align=center,anchor=south}}
}
{
\end{tikzpicture}
\end{center}
}

\newcommand{\TreeFrame}[2]{
\begin{columns}[c]
\column{0.5\textwidth}
\begin{center}
\begin{prooftree}{}
#1
\end{prooftree}
\end{center}
\column{0.45\textwidth}
%\begin{markdown}
#2
%\end{markdown}
\end{columns}
}

\newcommand{\ScaledTreeFrame}[3]{
\begin{columns}[c]
\column{0.5\textwidth}
\begin{center}
\scalebox{#1}{
\begin{prooftree}{}
#2
\end{prooftree}
}
\end{center}
\column{0.45\textwidth}
%\begin{markdown}
#3
%\end{markdown}
\end{columns}
}

\usepackage[bb=boondox]{mathalfa}
\DeclareMathAlphabet{\mathbx}{U}{BOONDOX-ds}{m}{n}
\SetMathAlphabet{\mathbx}{bold}{U}{BOONDOX-ds}{b}{n}
\DeclareMathAlphabet{\mathbbx} {U}{BOONDOX-ds}{b}{n}

\RequirePackage{bussproofs}
\RequirePackage[tableaux]{prooftrees}

\newenvironment{oltableau}{\center\tableau{}} %wff format={anchor = base west}}}
       {\endtableau\endcenter}
       
\newcommand{\formula}[1]{$#1$}

\usepackage{tabulary}
\usepackage{booktabs}

\def\begincols{\begin{columns}}
\def\begincol{\begin{column}}
\def\endcol{\end{column}}
\def\endcols{\end{columns}}

\usepackage[italic]{mathastext}
\usepackage{nicefrac}

\definecolor{mygreen}{RGB}{0, 100, 0}
\definecolor{mypink2}{RGB}{219, 48, 122}
\definecolor{dodgerblue}{RGB}{30,144,255}

\def\True{\textcolor{dodgerblue}{\text{T}}}
\def\False{\textcolor{red}{\text{F}}}

\title{305 Lecture 37 - Utilities}
\author{Brian Weatherson}
\date{July 27, 2020}

\begin{document}
\frame{\titlepage}

\begin{frame}{Plan}
\protect\hypertarget{plan}{}

\begin{itemize}
\tightlist
\item
  In this lecture we'll talk about the notion of utility, the idea that
  we can numerically measure how good an option is for a chooser. \#
  Ordinal and Cardinal Utilities
\end{itemize}

\end{frame}

\begin{frame}{Ranking}
\protect\hypertarget{ranking}{}

\begin{itemize}
\tightlist
\item
  The dominance view makes recommendations just looking at the
  \textbf{ranking} of various options.
\item
  It doesn't look at how much we prefer one option over another, just on
  what is preferred to what.
\end{itemize}

\end{frame}

\begin{frame}{Ordinal Utility}
\protect\hypertarget{ordinal-utility}{}

\begin{itemize}
\tightlist
\item
  To use the technical language, dominance just depends on
  \textbf{ordinal utilities}.
\item
  The term \textbf{ordinal} here means that we only look at the
  \textbf{order} of the options.
\end{itemize}

\end{frame}

\begin{frame}{Cardinal Utility}
\protect\hypertarget{cardinal-utility}{}

\begin{itemize}
\tightlist
\item
  The rules that we'll look at rely on \textbf{cardinal utilities}.
\item
  Whenever we're associating outcomes with numbers in a way that the
  magnitudes of the differences between the numbers matters, we're using
  cardinal utilities.
\end{itemize}

\end{frame}

\begin{frame}{Why More Than Order Matters (An Example)}
\protect\hypertarget{why-more-than-order-matters-an-example}{}

\begin{itemize}
\tightlist
\item
  Chris and Robin each have to make a decision between two airlines to
  fly them from Detroit to Los Angeles.
\item
  One airline is more expensive, the other is more reliable.
\item
  To oversimplify things, let's say the unreliable airline runs well in
  good weather, but in bad weather, things go wrong.
\item
  And Chris and Robin have no way of finding out what the weather along
  the way will be.
\item
  They would prefer to save money, but they'd certainly not prefer for
  things to go badly wrong.
\end{itemize}

\end{frame}

\begin{frame}{A Table}
\protect\hypertarget{a-table}{}

So they face the following decision table.

\begin{longtable}[]{@{}rcc@{}}
\toprule
& Good weather & Bad Weather\tabularnewline
\midrule
\endhead
Fly Cheap Airline & 4 & 1\tabularnewline
Fly Good Airline & 3 & 2\tabularnewline
\bottomrule
\end{longtable}

If we're just looking at the ordering of outcomes, that is the decision
problem facing both Chris and Robin.

\end{frame}

\begin{frame}{Filling in Details}
\protect\hypertarget{filling-in-details}{}

\begin{itemize}
\tightlist
\item
  The cheap airline that Chris might fly has a problem with luggage.
\item
  If the weather is bad, their passengers' luggage will be a day late
  getting to Los Angeles. \pause 
\item
  The cheap airline that Robin might fly has a problem with staying in
  the air.
\item
  If the weather is bad, their plane will crash.
\end{itemize}

\end{frame}

\begin{frame}{Details Matter}
\protect\hypertarget{details-matter}{}

\begin{itemize}
\tightlist
\item
  Those seem like very different decision problems.
\item
  It might be worth risking one's luggage being a day late in order to
  get a cheap plane ticket.
\item
  It's not worth risking, seriously risking, a plane crash.
\item
  That's to say, Chris and Robin are facing very different decision
  problems, even though the ranking of the four possible outcomes is the
  same in each of their cases.
\item
  So it seems like some decision rules should be sensitive to magnitudes
  of differences between options.
\end{itemize}

\end{frame}

\begin{frame}{Utility}
\protect\hypertarget{utility}{}

\begin{itemize}
\tightlist
\item
  Intuitively, think of utilities as measuring how good an outcome is.
\item
  The theory we're building towards is thoroughly subjectivist, so think
  of `how good' as meaning `how good along all and only dimensions the
  agent making the decision cares about'.
\end{itemize}

\end{frame}

\begin{frame}{Scale}
\protect\hypertarget{scale}{}

\begin{itemize}
\tightlist
\item
  Utilities aren't really measured on any scale.
\item
  Indeed, like temperature measures, they don't even have a fixed zero
  point.
\item
  It is usually convenient to associate 0 utility with the status quo,
  and then have negative numbers for outcomes worse than status quo, and
  positive numbers for outcomes better than status quo.
\item
  But that's just a convention; you can set the 0 wherever you like.
\item
  And you can set the utility 1 point at anything better than 0.
\end{itemize}

\end{frame}

\begin{frame}{Scale (continued)}
\protect\hypertarget{scale-continued}{}

\begin{itemize}
\tightlist
\item
  But that's where the convention stops.
\item
  Once you fix the 0 and 1 points, nothing else is fixed by pure
  convention.
\item
  Temperatures are like this too.
\end{itemize}

\end{frame}

\begin{frame}{Dollars and Utility}
\protect\hypertarget{dollars-and-utility}{}

Orthodox utility theory takes the following two facts to be important,
and in need of explanation, and to ultimately have the same explanation:

\begin{enumerate}
\tightlist
\item
  If you or I got a windfall prize of \$1,000,000, it would be an
  enormous, life-altering, change. But if Mark Zuckerberg got a windfall
  prize of \$1,000,000, he'd barely notice it.
\item
  If given a choice between a guaranteed \$1,000,000, and a 50/50 chance
  of winning \$2,000,000, you would almost certainly take the
  \$1,000,000. Indeed, most of you would do so with barely a moment's
  hesitation.
\end{enumerate}

\end{frame}

\begin{frame}{The Explanation}
\protect\hypertarget{the-explanation}{}

\begin{itemize}
\tightlist
\item
  The more money you have, the less utility you get from each extra
  dollar.
\item
  There is a \textbf{declining marginal utility} to money.
\item
  The marginal utility of a good is how much utility a person gets,
  relative to where they are now, from a little extra of that good.
\item
  For most goods, the more of them you have, the less useful an extra
  one is.
\item
  This is especially true for money.
\end{itemize}

\end{frame}

\begin{frame}{For Next Time}
\protect\hypertarget{for-next-time}{}

\begin{itemize}
\tightlist
\item
  We will look at how to think about decisions where dominance reasoning
  doesn't apply.
\end{itemize}

\end{frame}

\end{document}
