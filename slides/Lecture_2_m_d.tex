% Options for packages loaded elsewhere
\PassOptionsToPackage{unicode}{hyperref}
\PassOptionsToPackage{hyphens}{url}
%
\documentclass[
  ignorenonframetext,
]{beamer}
\usepackage{pgfpages}
\setbeamertemplate{caption}[numbered]
\setbeamertemplate{caption label separator}{: }
\setbeamercolor{caption name}{fg=normal text.fg}
\beamertemplatenavigationsymbolsempty
% Prevent slide breaks in the middle of a paragraph
\widowpenalties 1 10000
\raggedbottom
\setbeamertemplate{part page}{
  \centering
  \begin{beamercolorbox}[sep=16pt,center]{part title}
    \usebeamerfont{part title}\insertpart\par
  \end{beamercolorbox}
}
\setbeamertemplate{section page}{
  \centering
  \begin{beamercolorbox}[sep=12pt,center]{part title}
    \usebeamerfont{section title}\insertsection\par
  \end{beamercolorbox}
}
\setbeamertemplate{subsection page}{
  \centering
  \begin{beamercolorbox}[sep=8pt,center]{part title}
    \usebeamerfont{subsection title}\insertsubsection\par
  \end{beamercolorbox}
}
\AtBeginPart{
  \frame{\partpage}
}
\AtBeginSection{
  \ifbibliography
  \else
    \frame{\sectionpage}
  \fi
}
\AtBeginSubsection{
  \frame{\subsectionpage}
}
\usepackage{lmodern}
\usepackage{amssymb,amsmath}
\usepackage{ifxetex,ifluatex}
\ifnum 0\ifxetex 1\fi\ifluatex 1\fi=0 % if pdftex
  \usepackage[T1]{fontenc}
  \usepackage[utf8]{inputenc}
  \usepackage{textcomp} % provide euro and other symbols
\else % if luatex or xetex
  \usepackage{unicode-math}
  \defaultfontfeatures{Scale=MatchLowercase}
  \defaultfontfeatures[\rmfamily]{Ligatures=TeX,Scale=1}
\fi
% Use upquote if available, for straight quotes in verbatim environments
\IfFileExists{upquote.sty}{\usepackage{upquote}}{}
\IfFileExists{microtype.sty}{% use microtype if available
  \usepackage[]{microtype}
  \UseMicrotypeSet[protrusion]{basicmath} % disable protrusion for tt fonts
}{}
\makeatletter
\@ifundefined{KOMAClassName}{% if non-KOMA class
  \IfFileExists{parskip.sty}{%
    \usepackage{parskip}
  }{% else
    \setlength{\parindent}{0pt}
    \setlength{\parskip}{6pt plus 2pt minus 1pt}}
}{% if KOMA class
  \KOMAoptions{parskip=half}}
\makeatother
\usepackage{xcolor}
\IfFileExists{xurl.sty}{\usepackage{xurl}}{} % add URL line breaks if available
\IfFileExists{bookmark.sty}{\usepackage{bookmark}}{\usepackage{hyperref}}
\hypersetup{
  pdftitle={305 Lecture 18 - Building Complicated Truth Tables},
  pdfauthor={Brian Weatherson},
  hidelinks,
  pdfcreator={LaTeX via pandoc}}
\urlstyle{same} % disable monospaced font for URLs
\newif\ifbibliography
\setlength{\emergencystretch}{3em} % prevent overfull lines
\providecommand{\tightlist}{%
  \setlength{\itemsep}{0pt}\setlength{\parskip}{0pt}}
\setcounter{secnumdepth}{-\maxdimen} % remove section numbering
\let\Tiny=\tiny

 \setbeamertemplate{navigation symbols}{} 

% \usetheme{Madrid}
 \usetheme[numbering=none, progressbar=foot]{metropolis}
 \usecolortheme{wolverine}
 \usepackage{color}
 \usepackage{MnSymbol}
% \usepackage{movie15}

\usepackage{amssymb}% http://ctan.org/pkg/amssymb
\usepackage{pifont}% http://ctan.org/pkg/pifont
\newcommand{\cmark}{\ding{51}}%
\newcommand{\xmark}{\ding{55}}%

\DeclareSymbolFont{symbolsC}{U}{txsyc}{m}{n}
\DeclareMathSymbol{\boxright}{\mathrel}{symbolsC}{128}
\DeclareMathAlphabet{\mathpzc}{OT1}{pzc}{m}{it}


% \usepackage{tikz-qtree}
% \usepackage{markdown}
% \usepackage{prooftrees}
% \forestset{not line numbering, close with = x}
% Allow for easy commas inside trees
\renewcommand{\,}{\text{, }}


\usepackage{tabulary}

\usepackage{open-logic-config}

\setlength{\parskip}{1ex plus 0.5ex minus 0.2ex}

\AtBeginSection[]
{
\begin{frame}
	\Huge{\color{darkblue} \insertsection}
\end{frame}
}

\renewenvironment*{quote}	
	{\list{}{\rightmargin   \leftmargin} \item } 	
	{\endlist }

\definecolor{darkgreen}{rgb}{0,0.7,0}
\definecolor{darkblue}{rgb}{0,0,0.8}

\newcommand{\starttab}{\begin{center}
\vspace{6pt}
\begin{tabular}}

\newcommand{\stoptab}{\end{tabular}
\vspace{6pt}
\end{center}
\noindent}


\newcommand{\sif}{\rightarrow}
\newcommand{\siff}{\leftrightarrow}
\newcommand{\EF}{\end{frame}}


\newcommand{\TreeStart}[1]{
%\end{frame}
\begin{frame}
\begin{center}
\begin{tikzpicture}[scale=#1]
\tikzset{every tree node/.style={align=center,anchor=north}}
%\Tree
}

\newcommand{\TreeEnd}{
\end{tikzpicture}
%\end{center}
}

\newcommand{\DisplayArg}[2]{
\begin{enumerate}
{#1}
\end{enumerate}
\vspace{-6pt}
\hrulefill

%\hspace{14pt} #2
%{\addtolength{\leftskip}{14pt} #2}
\begin{quote}
{\normalfont #2}
\end{quote}
\vspace{12pt}
}

\newenvironment{ProofTree}[1][1]{
\begin{center}
\begin{tikzpicture}[scale=#1]
\tikzset{every tree node/.style={align=center,anchor=south}}
}
{
\end{tikzpicture}
\end{center}
}

\newcommand{\TreeFrame}[2]{
\begin{columns}[c]
\column{0.5\textwidth}
\begin{center}
\begin{prooftree}{}
#1
\end{prooftree}
\end{center}
\column{0.45\textwidth}
%\begin{markdown}
#2
%\end{markdown}
\end{columns}
}

\newcommand{\ScaledTreeFrame}[3]{
\begin{columns}[c]
\column{0.5\textwidth}
\begin{center}
\scalebox{#1}{
\begin{prooftree}{}
#2
\end{prooftree}
}
\end{center}
\column{0.45\textwidth}
%\begin{markdown}
#3
%\end{markdown}
\end{columns}
}

\usepackage[bb=boondox]{mathalfa}
\DeclareMathAlphabet{\mathbx}{U}{BOONDOX-ds}{m}{n}
\SetMathAlphabet{\mathbx}{bold}{U}{BOONDOX-ds}{b}{n}
\DeclareMathAlphabet{\mathbbx} {U}{BOONDOX-ds}{b}{n}

\RequirePackage{bussproofs}
\RequirePackage[tableaux]{prooftrees}

\newenvironment{oltableau}{\center\tableau{}} %wff format={anchor = base west}}}
       {\endtableau\endcenter}
       
\newcommand{\formula}[1]{$#1$}

\usepackage{tabulary}
\usepackage{booktabs}

\def\begincols{\begin{columns}}
\def\begincol{\begin{column}}
\def\endcol{\end{column}}
\def\endcols{\end{columns}}

\usepackage[italic]{mathastext}
\usepackage{nicefrac}

\definecolor{mygreen}{RGB}{0, 100, 0}
\definecolor{mypink2}{RGB}{219, 48, 122}
\definecolor{dodgerblue}{RGB}{30,144,255}

\def\True{\textcolor{dodgerblue}{\text{T}}}
\def\False{\textcolor{red}{\text{F}}}

\title{305 Lecture 18 - Building Complicated Truth Tables}
\author{Brian Weatherson}
\date{July 13, 2020}

\begin{document}
\frame{\titlepage}

\begin{frame}{Plan}
\protect\hypertarget{plan}{}

This lecture is about how to build more complicated truth tables than we
have looked at so far.

\end{frame}

\begin{frame}{The Example}
\protect\hypertarget{the-example}{}

We are going to work out the truth table for this sentence:

\begin{quote}
\((A \vee \neg B) \rightarrow (B \rightarrow (A \wedge C))\)
\end{quote}

\end{frame}

\begin{frame}{How Many Rows}
\protect\hypertarget{how-many-rows}{}

\begin{itemize}[<+->]
\tightlist
\item
  How many rows should there be in the truth table?
\item
  There are three (3) atomic sentences, so there should be \(2^3 = 8\)
  rows.
\end{itemize}

\end{frame}

\begin{frame}{Laying Out the Rows}
\protect\hypertarget{laying-out-the-rows}{}

\begin{itemize}
\tightlist
\item
  The convention for these is a bit odd.
\item
  Here's one way to think about it.
\item
  For the left-most column you fill the first half of the rows with
  \(\True\) and then the second half of the rows with \(\False\).
\end{itemize}

\end{frame}

\begin{frame}{First Column}
\protect\hypertarget{first-column}{}

\begin{center}
\begin{tabular}{@{ }c@{ }@{ }c@{ }@{ }c | c@{ }@{}c@{}@{ }c@{ }@{ }c@{ }@{ }c@{ }@{ }c@{ }@{}c@{}@{ }c@{ }@{}c@{}@{ }c@{ }@{ }c@{ }@{}c@{}@{ }c@{ }@{ }c@{ }@{ }c@{ }@{}c@{}@{}c@{}@{ }c}
A & B & C &  & ( & A & $\vee$ & $\neg$ & B & ) & $\rightarrow$ & ( & B & $\rightarrow$ & ( & A & $\wedge$ & C & ) & ) & \\
\hline 
 $\True$ & & \\
 $\True$ & & \\
 $\True$ & & \\
 $\True$ & & \\
 $\False$ & &\\
 $\False$ & &\\
 $\False$ & &\\
 $\False$ & &\\
\end{tabular}
\end{center}

\end{frame}

\begin{frame}{Second Column}
\protect\hypertarget{second-column}{}

\begin{itemize}
\tightlist
\item
  Then the second column has one quarter \(\True\), followed by one
  quarter \(\False\), followed by one quarter \(\True\), followed by one
  quarter \(\False\).
\item
  In this case that means we alternate every two rows.
\end{itemize}

\end{frame}

\begin{frame}{Second Column}
\protect\hypertarget{second-column-1}{}

\begin{center}
\begin{tabular}{@{ }c@{ }@{ }c@{ }@{ }c | c@{ }@{}c@{}@{ }c@{ }@{ }c@{ }@{ }c@{ }@{ }c@{ }@{}c@{}@{ }c@{ }@{}c@{}@{ }c@{ }@{ }c@{ }@{}c@{}@{ }c@{ }@{ }c@{ }@{ }c@{ }@{}c@{}@{}c@{}@{ }c}
A & B & C &  & ( & A & $\lor$ & $\sim$ & B & ) & $\rightarrow$ & ( & B & $\rightarrow$ & ( & A & $\&$ & C & ) & ) & \\
\hline 
 $\True$ & $\True$ &\\
 $\True$ & $\True$ &\\
 $\True$ & $\False$ &\\
 $\True$ & $\False$ & \\
 $\False$ & $\True$ &\\
 $\False$ & $\True$ &\\
 $\False$ & $\False$ &\\
 $\False$ & $\False$ &\\
\end{tabular}
\end{center}

\end{frame}

\begin{frame}{Third Column}
\protect\hypertarget{third-column}{}

\begin{itemize}
\tightlist
\item
  From now on you do half as many rows between changes.
\item
  In this table we did 4 rows with one value then 4 of another for
  column 1, 2 with one value then 2 with another for column 2, and now
  alternate every row for column 3.
\item
  It's helpful to know the full algorithm in case you ever have to do
  this with 5 or more variables.
\item
  But I won't do that in this course.
\end{itemize}

\end{frame}

\begin{frame}{Third Column}
\protect\hypertarget{third-column-1}{}

\begin{center}
\begin{tabular}{@{ }c@{ }@{ }c@{ }@{ }c | c@{ }@{}c@{}@{ }c@{ }@{ }c@{ }@{ }c@{ }@{ }c@{ }@{}c@{}@{ }c@{ }@{}c@{}@{ }c@{ }@{ }c@{ }@{}c@{}@{ }c@{ }@{ }c@{ }@{ }c@{ }@{}c@{}@{}c@{}@{ }c}
A & B & C &  & ( & A & $\lor$ & $\sim$ & B & ) & $\rightarrow$ & ( & B & $\rightarrow$ & ( & A & $\&$ & C & ) & ) & \\
\hline 
 $\True$ & $\True$ & $\True$ &  & \\
 $\True$ & $\True$ & $\False$ &  & \\
 $\True$ & $\False$ & $\True$ &  & \\
 $\True$ & $\False$ & $\False$ &  &  \\
 $\False$ & $\True$ & $\True$ &  & \\
 $\False$ & $\True$ & $\False$ &  & \\
 $\False$ & $\False$ & $\True$ &  & \\
 $\False$ & $\False$ & $\False$ &  & \\
\end{tabular}
\end{center}

\end{frame}

\begin{frame}{Parsing the Sentence}
\protect\hypertarget{parsing-the-sentence}{}

Now we need to go back to our sentence.

\begin{quote}
\((A \vee \neg B) \rightarrow (B \rightarrow (A \wedge C))\)
\end{quote}

What is its \textbf{main connective}? \pause \bigskip

\begin{itemize}
\tightlist
\item
  It's the first \(\rightarrow\). The sentence is of the form
  \(D \rightarrow E\), where \(D\) is \((A \vee \neg B)\) and \(E\) is
  \((B \rightarrow (A \wedge C))\)
\end{itemize}

\end{frame}

\begin{frame}{Building Up}
\protect\hypertarget{building-up}{}

So eventually, we will have the truth value for the whole sentence under
the first \(\rightarrow\).

\begin{itemize}
\tightlist
\item
  But that's some distance away.
\item
  While that's where we want to get to, we have to build from the inside
  out.
\item
  The first thing to do is to repeat the values for the atomic
  sentences.
\end{itemize}

\end{frame}

\begin{frame}{Atomic Replicator}
\protect\hypertarget{atomic-replicator}{}

\begin{center}
\begin{tabular}{@{ }c@{ }@{ }c@{ }@{ }c | c@{ }@{}c@{}@{ }c@{ }@{ }c@{ }@{ }c@{ }@{ }c@{ }@{}c@{}@{ }c@{ }@{}c@{}@{ }c@{ }@{ }c@{ }@{}c@{}@{ }c@{ }@{ }c@{ }@{ }c@{ }@{}c@{}@{}c@{}@{ }c}
A & B & C &  & ( & A & $\vee$ & $\neg$ & B & ) & $\rightarrow$ & ( & B & $\rightarrow$ & ( & A & $\wedge$ & C & ) & ) & \\
\hline 
 $\True$ & $\True$ & $\True$ &  &  & $\True$ &&& $\True$ &  &&  & $\True$ &&  & $\True$ && $\True$ &  &  & \\
 $\True$ & $\True$ & $\False$ &  &  & $\True$ &&& $\True$ &  &&  & $\True$ &&  & $\True$ && $\False$ &  &  & \\
 $\True$ & $\False$ & $\True$ &  &  & $\True$ &&& $\False$ &  &&  & $\False$ &&  & $\True$ && $\True$ &  &  & \\
 $\True$ & $\False$ & $\False$ &  &  & $\True$ &&& $\False$ &  &&  & $\False$ &&  & $\True$ && $\False$ &  &  & \\
 $\False$ & $\True$ & $\True$ &  &  & $\False$ &&& $\True$ &  &&  & $\True$ &&  & $\False$ && $\True$ &  &  & \\
 $\False$ & $\True$ & $\False$ &  &  & $\False$ &&& $\True$ &  &&  & $\True$ &&  & $\False$ && $\False$ &  &  & \\
 $\False$ & $\False$ & $\True$ &  &  & $\False$ &&& $\False$ &  &&  & $\False$ &&  & $\False$ && $\True$ &  &  & \\
 $\False$ & $\False$ & $\False$ &  &  & $\False$ &&& $\False$ &  &&  & $\False$ &&  & $\False$ && $\False$ &  &  & \\
\end{tabular}
\bigskip
\end{center}

What can we fill in immediately?

\end{frame}

\begin{frame}{Next Steps}
\protect\hypertarget{next-steps}{}

\begin{itemize}
\tightlist
\item
  We have enough on the table to include the values for \(\neg B\).
\item
  And we have enough on the table to include the values for the
  \(A \wedge C\) on the far right.
\item
  We'll do these in order.
\end{itemize}

\end{frame}

\begin{frame}{Negation}
\protect\hypertarget{negation}{}

Everywhere \(B\) is \(\True\), \(\neg B\) is \(\False\), so let's
include all of those.

\begin{center}
\bigskip
\begin{tabular}{@{ }c@{ }@{ }c@{ }@{ }c | c@{ }@{}c@{}@{ }c@{ }@{ }c@{ }@{ }c@{ }@{ }c@{ }@{}c@{}@{ }c@{ }@{}c@{}@{ }c@{ }@{ }c@{ }@{}c@{}@{ }c@{ }@{ }c@{ }@{ }c@{ }@{}c@{}@{}c@{}@{ }c}
A & B & C &  & ( & A & $\vee$ & $\neg$ & B & ) & $\rightarrow$ & ( & B & $\rightarrow$ & ( & A & $\wedge$ & C & ) & ) & \\
\hline 
 $\True$ & $\True$ & $\True$ &  &  & $\True$ && $\False$ & $\True$ &  &&  & $\True$ &&  & $\True$ && $\True$ &  &  & \\
 $\True$ & $\True$ & $\False$ &  &  & $\True$ && $\False$ & $\True$ &  &&  & $\True$ &&  & $\True$ && $\False$ &  &  & \\
 $\True$ & $\False$ & $\True$ &  &  & $\True$ &&& $\False$ &  &&  & $\False$ &&  & $\True$ && $\True$ &  &  & \\
 $\True$ & $\False$ & $\False$ &  &  & $\True$ &&& $\False$ &  &&  & $\False$ &&  & $\True$ && $\False$ &  &  & \\
 $\False$ & $\True$ & $\True$ &  &  & $\False$ && $\False$ & $\True$ &  &&  & $\True$ &&  & $\False$ && $\True$ &  &  & \\
 $\False$ & $\True$ & $\False$ &  &  & $\False$ && $\False$ & $\True$ &  &&  & $\True$ &&  & $\False$ && $\False$ &  &  & \\
 $\False$ & $\False$ & $\True$ &  &  & $\False$ &&& $\False$ &  &&  & $\False$ &&  & $\False$ && $\True$ &  &  & \\
 $\False$ & $\False$ & $\False$ &  &  & $\False$ &&& $\False$ &  &&  & $\False$ &&  & $\False$ && $\False$ &  &  & \\
\end{tabular}
\bigskip
\end{center}

\end{frame}

\begin{frame}{Negation (cont)}
\protect\hypertarget{negation-cont}{}

Everywhere \(B\) is \(\False\), \(\neg B\) is \(\True\), so let's
include all of those.

\begin{center}
\bigskip
\begin{tabular}{@{ }c@{ }@{ }c@{ }@{ }c | c@{ }@{}c@{}@{ }c@{ }@{ }c@{ }@{ }c@{ }@{ }c@{ }@{}c@{}@{ }c@{ }@{}c@{}@{ }c@{ }@{ }c@{ }@{}c@{}@{ }c@{ }@{ }c@{ }@{ }c@{ }@{}c@{}@{}c@{}@{ }c}
A & B & C &  & ( & A & $\vee$ & $\neg$ & B & ) & $\rightarrow$ & ( & B & $\rightarrow$ & ( & A & $\wedge$ & C & ) & ) & \\
\hline 
 $\True$ & $\True$ & $\True$ &  &  & $\True$ && $\False$ & $\True$ &  &&  & $\True$ &&  & $\True$ && $\True$ &  &  & \\
 $\True$ & $\True$ & $\False$ &  &  & $\True$ && $\False$ & $\True$ &  &&  & $\True$ &&  & $\True$ && $\False$ &  &  & \\
 $\True$ & $\False$ & $\True$ &  &  & $\True$ && $\True$ & $\False$ &  &&  & $\False$ &&  & $\True$ && $\True$ &  &  & \\
 $\True$ & $\False$ & $\False$ &  &  & $\True$ && $\True$ & $\False$ &  &&  & $\False$ &&  & $\True$ && $\False$ &  &  & \\
 $\False$ & $\True$ & $\True$ &  &  & $\False$ && $\False$ & $\True$ &  &&  & $\True$ &&  & $\False$ && $\True$ &  &  & \\
 $\False$ & $\True$ & $\False$ &  &  & $\False$ && $\False$ & $\True$ &  &&  & $\True$ &&  & $\False$ && $\False$ &  &  & \\
 $\False$ & $\False$ & $\True$ &  &  & $\False$ && $\True$ & $\False$ &  &&  & $\False$ &&  & $\False$ && $\True$ &  &  & \\
 $\False$ & $\False$ & $\False$ &  &  & $\False$ && $\True$ & $\False$ &  &&  & $\False$ &&  & $\False$ && $\False$ &  &  & \\
\end{tabular}
\bigskip
\end{center}

\end{frame}

\begin{frame}{Conjunction}
\protect\hypertarget{conjunction}{}

If \(A, C\) are both \(\True\), so is \(A \wedge C\). So let's include
those.

\begin{center}
\bigskip
\begin{tabular}{@{ }c@{ }@{ }c@{ }@{ }c | c@{ }@{}c@{}@{ }c@{ }@{ }c@{ }@{ }c@{ }@{ }c@{ }@{}c@{}@{ }c@{ }@{}c@{}@{ }c@{ }@{ }c@{ }@{}c@{}@{ }c@{ }@{ }c@{ }@{ }c@{ }@{}c@{}@{}c@{}@{ }c}
A & B & C &  & ( & A & $\vee$ & $\neg$ & B & ) & $\rightarrow$ & ( & B & $\rightarrow$ & ( & A & $\wedge$ & C & ) & ) & \\
\hline 
 $\True$ & $\True$ & $\True$ &  &  & $\True$ && $\False$ & $\True$ &  &&  & $\True$ &&  & $\True$ & $\True$ & $\True$ &  &  & \\
 $\True$ & $\True$ & $\False$ &  &  & $\True$ && $\False$ & $\True$ &  &&  & $\True$ &&  & $\True$ && $\False$ &  &  & \\
 $\True$ & $\False$ & $\True$ &  &  & $\True$ && $\True$ & $\False$ &  &&  & $\False$ &&  & $\True$ & $\True$ & $\True$ &  &  & \\
 $\True$ & $\False$ & $\False$ &  &  & $\True$ && $\True$ & $\False$ &  &&  & $\False$ &&  & $\True$ && $\False$ &  &  & \\
 $\False$ & $\True$ & $\True$ &  &  & $\False$ && $\False$ & $\True$ &  &&  & $\True$ &&  & $\False$ && $\True$ &  &  & \\
 $\False$ & $\True$ & $\False$ &  &  & $\False$ && $\False$ & $\True$ &  &&  & $\True$ &&  & $\False$ && $\False$ &  &  & \\
 $\False$ & $\False$ & $\True$ &  &  & $\False$ && $\True$ & $\False$ &  &&  & $\False$ &&  & $\False$ && $\True$ &  &  & \\
 $\False$ & $\False$ & $\False$ &  &  & $\False$ && $\True$ & $\False$ &  &&  & $\False$ &&  & $\False$ && $\False$ &  &  & \\
\end{tabular}
\bigskip
\end{center}

\end{frame}

\begin{frame}{Conjunction (cont)}
\protect\hypertarget{conjunction-cont}{}

And \(A \wedge C\) is false everywhere else.

\begin{center}
\bigskip
\begin{tabular}{@{ }c@{ }@{ }c@{ }@{ }c | c@{ }@{}c@{}@{ }c@{ }@{ }c@{ }@{ }c@{ }@{ }c@{ }@{}c@{}@{ }c@{ }@{}c@{}@{ }c@{ }@{ }c@{ }@{}c@{}@{ }c@{ }@{ }c@{ }@{ }c@{ }@{}c@{}@{}c@{}@{ }c}
A & B & C &  & ( & A & $\vee$ & $\neg$ & B & ) & $\rightarrow$ & ( & B & $\rightarrow$ & ( & A & $\wedge$ & C & ) & ) & \\
\hline 
 $\True$ & $\True$ & $\True$ &  &  & $\True$ && $\False$ & $\True$ &  &&  & $\True$ &&  & $\True$ & $\True$ & $\True$ &  &  & \\
 $\True$ & $\True$ & $\False$ &  &  & $\True$ && $\False$ & $\True$ &  &&  & $\True$ &&  & $\True$ & $\False$ & $\False$ &  &  & \\
 $\True$ & $\False$ & $\True$ &  &  & $\True$ && $\True$ & $\False$ &  &&  & $\False$ &&  & $\True$ & $\True$ & $\True$ &  &  & \\
 $\True$ & $\False$ & $\False$ &  &  & $\True$ && $\True$ & $\False$ &  &&  & $\False$ &&  & $\True$ & $\False$ & $\False$ &  &  & \\
 $\False$ & $\True$ & $\True$ &  &  & $\False$ && $\False$ & $\True$ &  &&  & $\True$ &&  & $\False$ & $\False$ & $\True$ &  &  & \\
 $\False$ & $\True$ & $\False$ &  &  & $\False$ && $\False$ & $\True$ &  &&  & $\True$ &&  & $\False$ & $\False$ & $\False$ &  &  & \\
 $\False$ & $\False$ & $\True$ &  &  & $\False$ && $\True$ & $\False$ &  &&  & $\False$ &&  & $\False$ & $\False$ & $\True$ &  &  & \\
 $\False$ & $\False$ & $\False$ &  &  & $\False$ && $\True$ & $\False$ &  &&  & $\False$ &&  & $\False$ & $\False$ & $\False$ &  &  & \\
\end{tabular}
\bigskip
\end{center}

\end{frame}

\begin{frame}{Complex Disjunction}
\protect\hypertarget{complex-disjunction}{}

\begin{itemize}
\tightlist
\item
  The next step is combining the values of \(A\) and \(\neg B\) to get
  the value of \(A \vee \neg B\).
\item
  The main thing to remember here is what your inputs are.
\item
  In this case it's not too confusing; it's the values immediately to
  either side of the \(\vee\).
\item
  But that won't be the general case.
\end{itemize}

\end{frame}

\begin{frame}{Disjunction}
\protect\hypertarget{disjunction}{}

When \(A\) is \(\True\), so is \(A \vee \neg B\).

\begin{center}
\bigskip
\begin{tabular}{@{ }c@{ }@{ }c@{ }@{ }c | c@{ }@{}c@{}@{ }c@{ }@{ }c@{ }@{ }c@{ }@{ }c@{ }@{}c@{}@{ }c@{ }@{}c@{}@{ }c@{ }@{ }c@{ }@{}c@{}@{ }c@{ }@{ }c@{ }@{ }c@{ }@{}c@{}@{}c@{}@{ }c}
A & B & C &  & ( & A & $\vee$ & $\neg$ & B & ) & $\rightarrow$ & ( & B & $\rightarrow$ & ( & A & $\wedge$ & C & ) & ) & \\
\hline 
 $\True$ & $\True$ & $\True$ &  &  & $\True$ & $\True$ & $\False$ & $\True$ &  &&  & $\True$ &&  & $\True$ & $\True$ & $\True$ &  &  & \\
 $\True$ & $\True$ & $\False$ &  &  & $\True$ & $\True$ & $\False$ & $\True$ &  &&  & $\True$ &&  & $\True$ & $\False$ & $\False$ &  &  & \\
 $\True$ & $\False$ & $\True$ &  &  & $\True$ & $\True$ & $\True$ & $\False$ &  &&  & $\False$ &&  & $\True$ & $\True$ & $\True$ &  &  & \\
 $\True$ & $\False$ & $\False$ &  &  & $\True$ & $\True$ & $\True$ & $\False$ &  &&  & $\False$ &&  & $\True$ & $\False$ & $\False$ &  &  & \\
 $\False$ & $\True$ & $\True$ &  &  & $\False$ && $\False$ & $\True$ &  &&  & $\True$ &&  & $\False$ & $\False$ & $\True$ &  &  & \\
 $\False$ & $\True$ & $\False$ &  &  & $\False$ && $\False$ & $\True$ &  &&  & $\True$ &&  & $\False$ & $\False$ & $\False$ &  &  & \\
 $\False$ & $\False$ & $\True$ &  &  & $\False$ && $\True$ & $\False$ &  &&  & $\False$ &&  & $\False$ & $\False$ & $\True$ &  &  & \\
 $\False$ & $\False$ & $\False$ &  &  & $\False$ && $\True$ & $\False$ &  &&  & $\False$ &&  & $\False$ & $\False$ & $\False$ &  &  & \\
\end{tabular}
\bigskip
\end{center}

\end{frame}

\begin{frame}{Disjunction (cont)}
\protect\hypertarget{disjunction-cont}{}

And when \(\neg B\) is \(\True\), so is \(A \vee \neg B\).

\begin{center}
\bigskip
\begin{tabular}{@{ }c@{ }@{ }c@{ }@{ }c | c@{ }@{}c@{}@{ }c@{ }@{ }c@{ }@{ }c@{ }@{ }c@{ }@{}c@{}@{ }c@{ }@{}c@{}@{ }c@{ }@{ }c@{ }@{}c@{}@{ }c@{ }@{ }c@{ }@{ }c@{ }@{}c@{}@{}c@{}@{ }c}
A & B & C &  & ( & A & $\vee$ & $\neg$ & B & ) & $\rightarrow$ & ( & B & $\rightarrow$ & ( & A & $\wedge$ & C & ) & ) & \\
\hline 
 $\True$ & $\True$ & $\True$ &  &  & $\True$ & $\True$ & $\False$ & $\True$ &  &&  & $\True$ &&  & $\True$ & $\True$ & $\True$ &  &  & \\
 $\True$ & $\True$ & $\False$ &  &  & $\True$ & $\True$ & $\False$ & $\True$ &  &&  & $\True$ &&  & $\True$ & $\False$ & $\False$ &  &  & \\
 $\True$ & $\False$ & $\True$ &  &  & $\True$ & $\True$ & $\True$ & $\False$ &  &&  & $\False$ &&  & $\True$ & $\True$ & $\True$ &  &  & \\
 $\True$ & $\False$ & $\False$ &  &  & $\True$ & $\True$ & $\True$ & $\False$ &  &&  & $\False$ &&  & $\True$ & $\False$ & $\False$ &  &  & \\
 $\False$ & $\True$ & $\True$ &  &  & $\False$ && $\False$ & $\True$ &  &&  & $\True$ &&  & $\False$ & $\False$ & $\True$ &  &  & \\
 $\False$ & $\True$ & $\False$ &  &  & $\False$ && $\False$ & $\True$ &  &&  & $\True$ &&  & $\False$ & $\False$ & $\False$ &  &  & \\
 $\False$ & $\False$ & $\True$ &  &  & $\False$ & $\True$ & $\True$ & $\False$ &  &&  & $\False$ &&  & $\False$ & $\False$ & $\True$ &  &  & \\
 $\False$ & $\False$ & $\False$ &  &  & $\False$ & $\True$ & $\True$ & $\False$ &  &&  & $\False$ &&  & $\False$ & $\False$ & $\False$ &  &  & \\
\end{tabular}
\bigskip
\end{center}

\end{frame}

\begin{frame}{Disjunction (part III)}
\protect\hypertarget{disjunction-part-iii}{}

Otherwise, \(A \vee \neg B\) is \(\False\).

\begin{center}
\bigskip
\begin{tabular}{@{ }c@{ }@{ }c@{ }@{ }c | c@{ }@{}c@{}@{ }c@{ }@{ }c@{ }@{ }c@{ }@{ }c@{ }@{}c@{}@{ }c@{ }@{}c@{}@{ }c@{ }@{ }c@{ }@{}c@{}@{ }c@{ }@{ }c@{ }@{ }c@{ }@{}c@{}@{}c@{}@{ }c}
A & B & C &  & ( & A & $\vee$ & $\neg$ & B & ) & $\rightarrow$ & ( & B & $\rightarrow$ & ( & A & $\wedge$ & C & ) & ) & \\
\hline 
 $\True$ & $\True$ & $\True$ &  &  & $\True$ & $\True$ & $\False$ & $\True$ &  &&  & $\True$ &&  & $\True$ & $\True$ & $\True$ &  &  & \\
 $\True$ & $\True$ & $\False$ &  &  & $\True$ & $\True$ & $\False$ & $\True$ &  &&  & $\True$ &&  & $\True$ & $\False$ & $\False$ &  &  & \\
 $\True$ & $\False$ & $\True$ &  &  & $\True$ & $\True$ & $\True$ & $\False$ &  &&  & $\False$ &&  & $\True$ & $\True$ & $\True$ &  &  & \\
 $\True$ & $\False$ & $\False$ &  &  & $\True$ & $\True$ & $\True$ & $\False$ &  &&  & $\False$ &&  & $\True$ & $\False$ & $\False$ &  &  & \\
 $\False$ & $\True$ & $\True$ &  &  & $\False$ & $\False$ & $\False$ & $\True$ &  &&  & $\True$ &&  & $\False$ & $\False$ & $\True$ &  &  & \\
 $\False$ & $\True$ & $\False$ &  &  & $\False$ & $\False$ & $\False$ & $\True$ &  &&  & $\True$ &&  & $\False$ & $\False$ & $\False$ &  &  & \\
 $\False$ & $\False$ & $\True$ &  &  & $\False$ & $\True$ & $\True$ & $\False$ &  &&  & $\False$ &&  & $\False$ & $\False$ & $\True$ &  &  & \\
 $\False$ & $\False$ & $\False$ &  &  & $\False$ & $\True$ & $\True$ & $\False$ &  &&  & $\False$ &&  & $\False$ & $\False$ & $\False$ &  &  & \\
\end{tabular}
\bigskip
\end{center}

\end{frame}

\begin{frame}{Conditional}
\protect\hypertarget{conditional}{}

\begin{itemize}
\tightlist
\item
  Now we have to do \(B \rightarrow (A \wedge C)\).
\item
  We have to remember the table for \(\rightarrow\) - TFTT.
\item
  And we have to remember that what's on the right-hand side of this
  conditional is a complex sentence: \(A \wedge C\).
\end{itemize}

\end{frame}

\begin{frame}{Conditional Terminology}
\protect\hypertarget{conditional-terminology}{}

\begin{itemize}
\tightlist
\item
  It's helpful to recall our distinctive terminology for conditionals
\item
  We'll often call the left-hand side of a conditional the
  \textbf{antecedent}. (`Ante' for before, if that helps.) \pause
\item
  And we'll call the right-hand side the \textbf{consequent} (i.e., what
  comes after).
\item
  We won't have fancy distinct terminology for the left-hand and
  right-hand sides of other sentences, because they are symmetric.
\end{itemize}

\end{frame}

\begin{frame}{Row 1}
\protect\hypertarget{row-1}{}

\(B\) is \(\True\), \(A \wedge C\) is \(\True\), so this is
\(\True \rightarrow \True\), i.e., \(\True\).

\begin{center}
\bigskip
\begin{tabular}{@{ }c@{ }@{ }c@{ }@{ }c | c@{ }@{}c@{}@{ }c@{ }@{ }c@{ }@{ }c@{ }@{ }c@{ }@{}c@{}@{ }c@{ }@{}c@{}@{ }c@{ }@{ }c@{ }@{}c@{}@{ }c@{ }@{ }c@{ }@{ }c@{ }@{}c@{}@{}c@{}@{ }c}
A & B & C &  & ( & A & $\vee$ & $\neg$ & B & ) & $\rightarrow$ & ( & B & $\rightarrow$ & ( & A & $\wedge$ & C & ) & ) & \\
\hline 
 $\True$ & $\True$ & $\True$ &  &  & $\True$ & $\True$ & $\False$ & $\True$ &  &&  & $\True$ & $\True$ &  & $\True$ & $\True$ & $\True$ &  &  & \\
 $\True$ & $\True$ & $\False$ &  &  & $\True$ & $\True$ & $\False$ & $\True$ &  &&  & $\True$ &&  & $\True$ & $\False$ & $\False$ &  &  & \\
 $\True$ & $\False$ & $\True$ &  &  & $\True$ & $\True$ & $\True$ & $\False$ &  &&  & $\False$ &&  & $\True$ & $\True$ & $\True$ &  &  & \\
 $\True$ & $\False$ & $\False$ &  &  & $\True$ & $\True$ & $\True$ & $\False$ &  &&  & $\False$ &&  & $\True$ & $\False$ & $\False$ &  &  & \\
 $\False$ & $\True$ & $\True$ &  &  & $\False$ & $\False$ & $\False$ & $\True$ &  &&  & $\True$ &&  & $\False$ & $\False$ & $\True$ &  &  & \\
 $\False$ & $\True$ & $\False$ &  &  & $\False$ & $\False$ & $\False$ & $\True$ &  &&  & $\True$ &&  & $\False$ & $\False$ & $\False$ &  &  & \\
 $\False$ & $\False$ & $\True$ &  &  & $\False$ & $\True$ & $\True$ & $\False$ &  &&  & $\False$ &&  & $\False$ & $\False$ & $\True$ &  &  & \\
 $\False$ & $\False$ & $\False$ &  &  & $\False$ & $\True$ & $\True$ & $\False$ &  &&  & $\False$ &&  & $\False$ & $\False$ & $\False$ &  &  & \\
\end{tabular}
\bigskip
\end{center}

\end{frame}

\begin{frame}{Row 2}
\protect\hypertarget{row-2}{}

\(B\) is \(\True\), \(A \wedge C\) is \(\False\), so this is
\(\True \rightarrow \False\), i.e., \(\False\).

\begin{center}
\bigskip
\begin{tabular}{@{ }c@{ }@{ }c@{ }@{ }c | c@{ }@{}c@{}@{ }c@{ }@{ }c@{ }@{ }c@{ }@{ }c@{ }@{}c@{}@{ }c@{ }@{}c@{}@{ }c@{ }@{ }c@{ }@{}c@{}@{ }c@{ }@{ }c@{ }@{ }c@{ }@{}c@{}@{}c@{}@{ }c}
A & B & C &  & ( & A & $\vee$ & $\neg$ & B & ) & $\rightarrow$ & ( & B & $\rightarrow$ & ( & A & $\wedge$ & C & ) & ) & \\
\hline 
 $\True$ & $\True$ & $\True$ &  &  & $\True$ & $\True$ & $\False$ & $\True$ &  &&  & $\True$ & $\True$ &  & $\True$ & $\True$ & $\True$ &  &  & \\
 $\True$ & $\True$ & $\False$ &  &  & $\True$ & $\True$ & $\False$ & $\True$ &  &&  & $\True$ & $\False$ &  & $\True$ & $\False$ & $\False$ &  &  & \\
 $\True$ & $\False$ & $\True$ &  &  & $\True$ & $\True$ & $\True$ & $\False$ &  &&  & $\False$ &&  & $\True$ & $\True$ & $\True$ &  &  & \\
 $\True$ & $\False$ & $\False$ &  &  & $\True$ & $\True$ & $\True$ & $\False$ &  &&  & $\False$ &&  & $\True$ & $\False$ & $\False$ &  &  & \\
 $\False$ & $\True$ & $\True$ &  &  & $\False$ & $\False$ & $\False$ & $\True$ &  &&  & $\True$ &&  & $\False$ & $\False$ & $\True$ &  &  & \\
 $\False$ & $\True$ & $\False$ &  &  & $\False$ & $\False$ & $\False$ & $\True$ &  &&  & $\True$ &&  & $\False$ & $\False$ & $\False$ &  &  & \\
 $\False$ & $\False$ & $\True$ &  &  & $\False$ & $\True$ & $\True$ & $\False$ &  &&  & $\False$ &&  & $\False$ & $\False$ & $\True$ &  &  & \\
 $\False$ & $\False$ & $\False$ &  &  & $\False$ & $\True$ & $\True$ & $\False$ &  &&  & $\False$ &&  & $\False$ & $\False$ & $\False$ &  &  & \\
\end{tabular}
\bigskip
\end{center}

\end{frame}

\begin{frame}{Row 3}
\protect\hypertarget{row-3}{}

\(B\) is \(\False\), \(A \wedge C\) is \(\True\), so this is
\(\False \rightarrow \True\), i.e., \(\True\).

\begin{center}
\bigskip
\begin{tabular}{@{ }c@{ }@{ }c@{ }@{ }c | c@{ }@{}c@{}@{ }c@{ }@{ }c@{ }@{ }c@{ }@{ }c@{ }@{}c@{}@{ }c@{ }@{}c@{}@{ }c@{ }@{ }c@{ }@{}c@{}@{ }c@{ }@{ }c@{ }@{ }c@{ }@{}c@{}@{}c@{}@{ }c}
A & B & C &  & ( & A & $\vee$ & $\neg$ & B & ) & $\rightarrow$ & ( & B & $\rightarrow$ & ( & A & $\wedge$ & C & ) & ) & \\
\hline 
 $\True$ & $\True$ & $\True$ &  &  & $\True$ & $\True$ & $\False$ & $\True$ &  &&  & $\True$ & $\True$ &  & $\True$ & $\True$ & $\True$ &  &  & \\
 $\True$ & $\True$ & $\False$ &  &  & $\True$ & $\True$ & $\False$ & $\True$ &  &&  & $\True$ & $\False$ &  & $\True$ & $\False$ & $\False$ &  &  & \\
 $\True$ & $\False$ & $\True$ &  &  & $\True$ & $\True$ & $\True$ & $\False$ &  &&  & $\False$ & $\True$ &  & $\True$ & $\True$ & $\True$ &  &  & \\
 $\True$ & $\False$ & $\False$ &  &  & $\True$ & $\True$ & $\True$ & $\False$ &  &&  & $\False$ &&  & $\True$ & $\False$ & $\False$ &  &  & \\
 $\False$ & $\True$ & $\True$ &  &  & $\False$ & $\False$ & $\False$ & $\True$ &  &&  & $\True$ &&  & $\False$ & $\False$ & $\True$ &  &  & \\
 $\False$ & $\True$ & $\False$ &  &  & $\False$ & $\False$ & $\False$ & $\True$ &  &&  & $\True$ &&  & $\False$ & $\False$ & $\False$ &  &  & \\
 $\False$ & $\False$ & $\True$ &  &  & $\False$ & $\True$ & $\True$ & $\False$ &  &&  & $\False$ &&  & $\False$ & $\False$ & $\True$ &  &  & \\
 $\False$ & $\False$ & $\False$ &  &  & $\False$ & $\True$ & $\True$ & $\False$ &  &&  & $\False$ &&  & $\False$ & $\False$ & $\False$ &  &  & \\
\end{tabular}
\bigskip
\end{center}

\end{frame}

\begin{frame}{Row 4}
\protect\hypertarget{row-4}{}

\(B\) is \(\False\), \(A \wedge C\) is \(\False\), so this is
\(\False \rightarrow \False\), i.e., \(\True\).

\begin{center}
\bigskip
\begin{tabular}{@{ }c@{ }@{ }c@{ }@{ }c | c@{ }@{}c@{}@{ }c@{ }@{ }c@{ }@{ }c@{ }@{ }c@{ }@{}c@{}@{ }c@{ }@{}c@{}@{ }c@{ }@{ }c@{ }@{}c@{}@{ }c@{ }@{ }c@{ }@{ }c@{ }@{}c@{}@{}c@{}@{ }c}
A & B & C &  & ( & A & $\vee$ & $\neg$ & B & ) & $\rightarrow$ & ( & B & $\rightarrow$ & ( & A & $\wedge$ & C & ) & ) & \\
\hline 
 $\True$ & $\True$ & $\True$ &  &  & $\True$ & $\True$ & $\False$ & $\True$ &  &&  & $\True$ & $\True$ &  & $\True$ & $\True$ & $\True$ &  &  & \\
 $\True$ & $\True$ & $\False$ &  &  & $\True$ & $\True$ & $\False$ & $\True$ &  &&  & $\True$ & $\False$ &  & $\True$ & $\False$ & $\False$ &  &  & \\
 $\True$ & $\False$ & $\True$ &  &  & $\True$ & $\True$ & $\True$ & $\False$ &  &&  & $\False$ & $\True$ &  & $\True$ & $\True$ & $\True$ &  &  & \\
 $\True$ & $\False$ & $\False$ &  &  & $\True$ & $\True$ & $\True$ & $\False$ &  &&  & $\False$ & $\True$ &  & $\True$ & $\False$ & $\False$ &  &  & \\
 $\False$ & $\True$ & $\True$ &  &  & $\False$ & $\False$ & $\False$ & $\True$ &  &&  & $\True$ &&  & $\False$ & $\False$ & $\True$ &  &  & \\
 $\False$ & $\True$ & $\False$ &  &  & $\False$ & $\False$ & $\False$ & $\True$ &  &&  & $\True$ &&  & $\False$ & $\False$ & $\False$ &  &  & \\
 $\False$ & $\False$ & $\True$ &  &  & $\False$ & $\True$ & $\True$ & $\False$ &  &&  & $\False$ &&  & $\False$ & $\False$ & $\True$ &  &  & \\
 $\False$ & $\False$ & $\False$ &  &  & $\False$ & $\True$ & $\True$ & $\False$ &  &&  & $\False$ &&  & $\False$ & $\False$ & $\False$ &  &  & \\
\end{tabular}
\bigskip
\end{center}

\end{frame}

\begin{frame}{Row 5}
\protect\hypertarget{row-5}{}

\(B\) is \(\True\), \(A \wedge C\) is \(\False\), so this is
\(\True \rightarrow \False\), i.e., \(\False\).

\begin{center}
\bigskip
\begin{tabular}{@{ }c@{ }@{ }c@{ }@{ }c | c@{ }@{}c@{}@{ }c@{ }@{ }c@{ }@{ }c@{ }@{ }c@{ }@{}c@{}@{ }c@{ }@{}c@{}@{ }c@{ }@{ }c@{ }@{}c@{}@{ }c@{ }@{ }c@{ }@{ }c@{ }@{}c@{}@{}c@{}@{ }c}
A & B & C &  & ( & A & $\vee$ & $\neg$ & B & ) & $\rightarrow$ & ( & B & $\rightarrow$ & ( & A & $\wedge$ & C & ) & ) & \\
\hline 
 $\True$ & $\True$ & $\True$ &  &  & $\True$ & $\True$ & $\False$ & $\True$ &  &&  & $\True$ & $\True$ &  & $\True$ & $\True$ & $\True$ &  &  & \\
 $\True$ & $\True$ & $\False$ &  &  & $\True$ & $\True$ & $\False$ & $\True$ &  &&  & $\True$ & $\False$ &  & $\True$ & $\False$ & $\False$ &  &  & \\
 $\True$ & $\False$ & $\True$ &  &  & $\True$ & $\True$ & $\True$ & $\False$ &  &&  & $\False$ & $\True$ &  & $\True$ & $\True$ & $\True$ &  &  & \\
 $\True$ & $\False$ & $\False$ &  &  & $\True$ & $\True$ & $\True$ & $\False$ &  &&  & $\False$ & $\True$ &  & $\True$ & $\False$ & $\False$ &  &  & \\
 $\False$ & $\True$ & $\True$ &  &  & $\False$ & $\False$ & $\False$ & $\True$ &  &&  & $\True$ & $\False$ &  & $\False$ & $\False$ & $\True$ &  &  & \\
 $\False$ & $\True$ & $\False$ &  &  & $\False$ & $\False$ & $\False$ & $\True$ &  &&  & $\True$ &&  & $\False$ & $\False$ & $\False$ &  &  & \\
 $\False$ & $\False$ & $\True$ &  &  & $\False$ & $\True$ & $\True$ & $\False$ &  &&  & $\False$ &&  & $\False$ & $\False$ & $\True$ &  &  & \\
 $\False$ & $\False$ & $\False$ &  &  & $\False$ & $\True$ & $\True$ & $\False$ &  &&  & $\False$ &&  & $\False$ & $\False$ & $\False$ &  &  & \\
\end{tabular}
\bigskip
\end{center}

\end{frame}

\begin{frame}{Row 6}
\protect\hypertarget{row-6}{}

\(B\) is \(\True\), \(A \wedge C\) is \(\False\), so this is
\(\True \rightarrow \False\), i.e., \(\False\).

\begin{center}
\bigskip
\begin{tabular}{@{ }c@{ }@{ }c@{ }@{ }c | c@{ }@{}c@{}@{ }c@{ }@{ }c@{ }@{ }c@{ }@{ }c@{ }@{}c@{}@{ }c@{ }@{}c@{}@{ }c@{ }@{ }c@{ }@{}c@{}@{ }c@{ }@{ }c@{ }@{ }c@{ }@{}c@{}@{}c@{}@{ }c}
A & B & C &  & ( & A & $\vee$ & $\neg$ & B & ) & $\rightarrow$ & ( & B & $\rightarrow$ & ( & A & $\wedge$ & C & ) & ) & \\
\hline 
 $\True$ & $\True$ & $\True$ &  &  & $\True$ & $\True$ & $\False$ & $\True$ &  &&  & $\True$ & $\True$ &  & $\True$ & $\True$ & $\True$ &  &  & \\
 $\True$ & $\True$ & $\False$ &  &  & $\True$ & $\True$ & $\False$ & $\True$ &  &&  & $\True$ & $\False$ &  & $\True$ & $\False$ & $\False$ &  &  & \\
 $\True$ & $\False$ & $\True$ &  &  & $\True$ & $\True$ & $\True$ & $\False$ &  &&  & $\False$ & $\True$ &  & $\True$ & $\True$ & $\True$ &  &  & \\
 $\True$ & $\False$ & $\False$ &  &  & $\True$ & $\True$ & $\True$ & $\False$ &  &&  & $\False$ & $\True$ &  & $\True$ & $\False$ & $\False$ &  &  & \\
 $\False$ & $\True$ & $\True$ &  &  & $\False$ & $\False$ & $\False$ & $\True$ &  &&  & $\True$ & $\False$ &  & $\False$ & $\False$ & $\True$ &  &  & \\
 $\False$ & $\True$ & $\False$ &  &  & $\False$ & $\False$ & $\False$ & $\True$ &  &&  & $\True$ & $\False$ &  & $\False$ & $\False$ & $\False$ &  &  & \\
 $\False$ & $\False$ & $\True$ &  &  & $\False$ & $\True$ & $\True$ & $\False$ &  &&  & $\False$ &&  & $\False$ & $\False$ & $\True$ &  &  & \\
 $\False$ & $\False$ & $\False$ &  &  & $\False$ & $\True$ & $\True$ & $\False$ &  &&  & $\False$ &&  & $\False$ & $\False$ & $\False$ &  &  & \\
\end{tabular}
\bigskip
\end{center}

\end{frame}

\begin{frame}{Rows 7 and 8}
\protect\hypertarget{rows-7-and-8}{}

\(B\) is \(\False\), \(A \wedge C\) is \(\False\), so this is
\(\False \rightarrow \False\), i.e., \(\False\).

\begin{center}
\bigskip
\begin{tabular}{@{ }c@{ }@{ }c@{ }@{ }c | c@{ }@{}c@{}@{ }c@{ }@{ }c@{ }@{ }c@{ }@{ }c@{ }@{}c@{}@{ }c@{ }@{}c@{}@{ }c@{ }@{ }c@{ }@{}c@{}@{ }c@{ }@{ }c@{ }@{ }c@{ }@{}c@{}@{}c@{}@{ }c}
A & B & C &  & ( & A & $\vee$ & $\neg$ & B & ) & $\rightarrow$ & ( & B & $\rightarrow$ & ( & A & $\wedge$ & C & ) & ) & \\
\hline 
 $\True$ & $\True$ & $\True$ &  &  & $\True$ & $\True$ & $\False$ & $\True$ &  &&  & $\True$ & $\True$ &  & $\True$ & $\True$ & $\True$ &  &  & \\
 $\True$ & $\True$ & $\False$ &  &  & $\True$ & $\True$ & $\False$ & $\True$ &  &&  & $\True$ & $\False$ &  & $\True$ & $\False$ & $\False$ &  &  & \\
 $\True$ & $\False$ & $\True$ &  &  & $\True$ & $\True$ & $\True$ & $\False$ &  &&  & $\False$ & $\True$ &  & $\True$ & $\True$ & $\True$ &  &  & \\
 $\True$ & $\False$ & $\False$ &  &  & $\True$ & $\True$ & $\True$ & $\False$ &  &&  & $\False$ & $\True$ &  & $\True$ & $\False$ & $\False$ &  &  & \\
 $\False$ & $\True$ & $\True$ &  &  & $\False$ & $\False$ & $\False$ & $\True$ &  &&  & $\True$ & $\False$ &  & $\False$ & $\False$ & $\True$ &  &  & \\
 $\False$ & $\True$ & $\False$ &  &  & $\False$ & $\False$ & $\False$ & $\True$ &  &&  & $\True$ & $\False$ &  & $\False$ & $\False$ & $\False$ &  &  & \\
 $\False$ & $\False$ & $\True$ &  &  & $\False$ & $\True$ & $\True$ & $\False$ &  &&  & $\False$ & $\True$ &  & $\False$ & $\False$ & $\True$ &  &  & \\
 $\False$ & $\False$ & $\False$ &  &  & $\False$ & $\True$ & $\True$ & $\False$ &  &&  & $\False$ & $\True$ &  & $\False$ & $\False$ & $\False$ &  &  & \\
\end{tabular}
\bigskip
\end{center}

\end{frame}

\begin{frame}{Almost Done}
\protect\hypertarget{almost-done}{}

\begin{itemize}
\tightlist
\item
  Now we just need to put the two parts together.
\item
  We have a conditional whose left-hand side, the antecedent, is
  \(A \vee \neg B\).
\item
  And the right-hand side, the consequent, is
  \(B \rightarrow (A \wedge C)\).
\item
  In each row we've computed the truth values for the antecedent and
  consequent.
\item
  Now it's a matter of just looking up how they combine.
\item
  Remember that the truth table for \(\rightarrow\) is TFTT.
\item
  To help, I've started by bolding the columns for the antecedent and
  consequent. (Not actually bold, but something more visible on
  projector.)
\end{itemize}

\end{frame}

\begin{frame}{Getting There}
\protect\hypertarget{getting-there}{}

Bolding the two relevant columns.
\begin{center} \begin{tabular}{@{ }c@{ }@{ }c@{ }@{ }c | c@{ }@{}c@{}@{ }c@{ }@{ }c@{ }@{ }c@{ }@{ }c@{ }@{}c@{}@{ }c@{ }@{}c@{}@{ }c@{ }@{ }c@{ }@{}c@{}@{ }c@{ }@{ }c@{ }@{ }c@{ }@{}c@{}@{}c@{}@{ }c} A & B & C &  & ( & A & $\vee$ & $\neg$ & B & ) & $\rightarrow$ & ( & B & $\rightarrow$ & ( & A & $\wedge$ & C & ) & ) & \\ \hline   $\True$ & $\True$ & $\True$ &  &  & $\True$ & $\mathbbx{T}$ & $\False$ & $\True$ &  &&  & $\True$ & $\mathbbx{T}$ &  & $\True$ & $\True$ & $\True$ &  &  & \\  $\True$ & $\True$ & $\False$ &  &  & $\True$ & $\mathbbx{T}$ & $\False$ & $\True$ &  &&  & $\True$ & $\mathbbx{F}$ &  & $\True$ & $\False$ & $\False$ &  &  & \\  $\True$ & $\False$ & $\True$ &  &  & $\True$ & $\mathbbx{T}$ & $\True$ & $\False$ &  &&  & $\False$ & $\mathbbx{T}$ &  & $\True$ & $\True$ & $\True$ &  &  & \\  $\True$ & $\False$ & $\False$ &  &  & $\True$ & $\mathbbx{T}$ & $\True$ & $\False$ &  &&  & $\False$ & $\mathbbx{T}$ &  & $\True$ & $\False$ & $\False$ &  &  & \\  $\False$ & $\True$ & $\True$ &  &  & $\False$ & $\mathbbx{F}$ & $\False$ & $\True$ &  &&  & $\True$ & $\mathbbx{F}$ &  & $\False$ & $\False$ & $\True$ &  &  & \\  $\False$ & $\True$ & $\False$ &  &  & $\False$ & $\mathbbx{F}$ & $\False$ & $\True$ &  &&  & $\True$ & $\mathbbx{F}$ &  & $\False$ & $\False$ & $\False$ &  &  & \\  $\False$ & $\False$ & $\True$ &  &  & $\False$ & $\mathbbx{F}$ & $\True$ & $\False$ &  &&  & $\False$ & $\mathbbx{T}$ &  & $\False$ & $\False$ & $\True$ &  &  & \\  $\False$ & $\False$ & $\False$ &  &  & $\False$ & $\mathbbx{F}$ & $\True$ & $\False$ &  &&  & $\False$ & $\mathbbx{T}$ &  & $\False$ & $\False$ & $\False$ &  &  & \\ \end{tabular} \end{center}

\end{frame}

\begin{frame}{Row 1}
\protect\hypertarget{row-1-1}{}

That's \(\True \rightarrow \True\), i.e., \(\True\).

\begin{center}
\bigskip
\begin{tabular}{@{ }c@{ }@{ }c@{ }@{ }c | c@{ }@{}c@{}@{ }c@{ }@{ }c@{ }@{ }c@{ }@{ }c@{ }@{}c@{}@{ }c@{ }@{}c@{}@{ }c@{ }@{ }c@{ }@{}c@{}@{ }c@{ }@{ }c@{ }@{ }c@{ }@{}c@{}@{}c@{}@{ }c}
A & B & C &  & ( & A & $\vee$ & $\neg$ & B & ) & $\rightarrow$ & ( & B & $\rightarrow$ & ( & A & $\wedge$ & C & ) & ) & \\
\hline 
 $\True$ & $\True$ & $\True$ &  &  & $\True$ & $\mathbbx{T}$ & $\False$ & $\True$ &  &\textcolor{red}{$\True$}&  & $\True$ & $\mathbbx{T}$ &  & $\True$ & $\True$ & $\True$ &  &  & \\
 $\True$ & $\True$ & $\False$ &  &  & $\True$ & $\mathbbx{T}$ & $\False$ & $\True$ &  &&  & $\True$ & $\mathbbx{F}$ &  & $\True$ & $\False$ & $\False$ &  &  & \\
 $\True$ & $\False$ & $\True$ &  &  & $\True$ & $\mathbbx{T}$ & $\True$ & $\False$ &  &&  & $\False$ & $\mathbbx{T}$ &  & $\True$ & $\True$ & $\True$ &  &  & \\
 $\True$ & $\False$ & $\False$ &  &  & $\True$ & $\mathbbx{T}$ & $\True$ & $\False$ &  &&  & $\False$ & $\mathbbx{T}$ &  & $\True$ & $\False$ & $\False$ &  &  & \\
 $\False$ & $\True$ & $\True$ &  &  & $\False$ & $\mathbbx{F}$ & $\False$ & $\True$ &  &&  & $\True$ & $\mathbbx{F}$ &  & $\False$ & $\False$ & $\True$ &  &  & \\
 $\False$ & $\True$ & $\False$ &  &  & $\False$ & $\mathbbx{F}$ & $\False$ & $\True$ &  &&  & $\True$ & $\mathbbx{F}$ &  & $\False$ & $\False$ & $\False$ &  &  & \\
 $\False$ & $\False$ & $\True$ &  &  & $\False$ & $\mathbbx{F}$ & $\True$ & $\False$ &  &&  & $\False$ & $\mathbbx{T}$ &  & $\False$ & $\False$ & $\True$ &  &  & \\
 $\False$ & $\False$ & $\False$ &  &  & $\False$ & $\mathbbx{F}$ & $\True$ & $\False$ &  &&  & $\False$ & $\mathbbx{T}$ &  & $\False$ & $\False$ & $\False$ &  &  & \\
\end{tabular}
\bigskip
\end{center}

\end{frame}

\begin{frame}{Row 2}
\protect\hypertarget{row-2-1}{}

That's \(\True \rightarrow \False\), i.e., \(\False\).

\begin{center}
\bigskip
\begin{tabular}{@{ }c@{ }@{ }c@{ }@{ }c | c@{ }@{}c@{}@{ }c@{ }@{ }c@{ }@{ }c@{ }@{ }c@{ }@{}c@{}@{ }c@{ }@{}c@{}@{ }c@{ }@{ }c@{ }@{}c@{}@{ }c@{ }@{ }c@{ }@{ }c@{ }@{}c@{}@{}c@{}@{ }c}
A & B & C &  & ( & A & $\vee$ & $\neg$ & B & ) & $\rightarrow$ & ( & B & $\rightarrow$ & ( & A & $\wedge$ & C & ) & ) & \\
\hline 
 $\True$ & $\True$ & $\True$ &  &  & $\True$ & $\mathbbx{T}$ & $\False$ & $\True$ &  &\textcolor{red}{$\True$}&  & $\True$ & $\mathbbx{T}$ &  & $\True$ & $\True$ & $\True$ &  &  & \\
 $\True$ & $\True$ & $\False$ &  &  & $\True$ & $\mathbbx{T}$ & $\False$ & $\True$ &  &\textcolor{red}{$\False$}&  & $\True$ & $\mathbbx{F}$ &  & $\True$ & $\False$ & $\False$ &  &  & \\
 $\True$ & $\False$ & $\True$ &  &  & $\True$ & $\mathbbx{T}$ & $\True$ & $\False$ &  &&  & $\False$ & $\mathbbx{T}$ &  & $\True$ & $\True$ & $\True$ &  &  & \\
 $\True$ & $\False$ & $\False$ &  &  & $\True$ & $\mathbbx{T}$ & $\True$ & $\False$ &  &&  & $\False$ & $\mathbbx{T}$ &  & $\True$ & $\False$ & $\False$ &  &  & \\
 $\False$ & $\True$ & $\True$ &  &  & $\False$ & $\mathbbx{F}$ & $\False$ & $\True$ &  &&  & $\True$ & $\mathbbx{F}$ &  & $\False$ & $\False$ & $\True$ &  &  & \\
 $\False$ & $\True$ & $\False$ &  &  & $\False$ & $\mathbbx{F}$ & $\False$ & $\True$ &  &&  & $\True$ & $\mathbbx{F}$ &  & $\False$ & $\False$ & $\False$ &  &  & \\
 $\False$ & $\False$ & $\True$ &  &  & $\False$ & $\mathbbx{F}$ & $\True$ & $\False$ &  &&  & $\False$ & $\mathbbx{T}$ &  & $\False$ & $\False$ & $\True$ &  &  & \\
 $\False$ & $\False$ & $\False$ &  &  & $\False$ & $\mathbbx{F}$ & $\True$ & $\False$ &  &&  & $\False$ & $\mathbbx{T}$ &  & $\False$ & $\False$ & $\False$ &  &  & \\
\end{tabular}
\bigskip
\end{center}

\end{frame}

\begin{frame}{Row 3}
\protect\hypertarget{row-3-1}{}

That's \(\True \rightarrow \True\), i.e., \(\True\).

\begin{center}
\bigskip
\begin{tabular}{@{ }c@{ }@{ }c@{ }@{ }c | c@{ }@{}c@{}@{ }c@{ }@{ }c@{ }@{ }c@{ }@{ }c@{ }@{}c@{}@{ }c@{ }@{}c@{}@{ }c@{ }@{ }c@{ }@{}c@{}@{ }c@{ }@{ }c@{ }@{ }c@{ }@{}c@{}@{}c@{}@{ }c}
A & B & C &  & ( & A & $\vee$ & $\neg$ & B & ) & $\rightarrow$ & ( & B & $\rightarrow$ & ( & A & $\wedge$ & C & ) & ) & \\
\hline 
 $\True$ & $\True$ & $\True$ &  &  & $\True$ & $\mathbbx{T}$ & $\False$ & $\True$ &  &\textcolor{red}{$\True$}&  & $\True$ & $\mathbbx{T}$ &  & $\True$ & $\True$ & $\True$ &  &  & \\
 $\True$ & $\True$ & $\False$ &  &  & $\True$ & $\mathbbx{T}$ & $\False$ & $\True$ &  &\textcolor{red}{$\False$}&  & $\True$ & $\mathbbx{F}$ &  & $\True$ & $\False$ & $\False$ &  &  & \\
 $\True$ & $\False$ & $\True$ &  &  & $\True$ & $\mathbbx{T}$ & $\True$ & $\False$ &  &\textcolor{red}{$\True$}&  & $\False$ & $\mathbbx{T}$ &  & $\True$ & $\True$ & $\True$ &  &  & \\
 $\True$ & $\False$ & $\False$ &  &  & $\True$ & $\mathbbx{T}$ & $\True$ & $\False$ &  &&  & $\False$ & $\mathbbx{T}$ &  & $\True$ & $\False$ & $\False$ &  &  & \\
 $\False$ & $\True$ & $\True$ &  &  & $\False$ & $\mathbbx{F}$ & $\False$ & $\True$ &  &&  & $\True$ & $\mathbbx{F}$ &  & $\False$ & $\False$ & $\True$ &  &  & \\
 $\False$ & $\True$ & $\False$ &  &  & $\False$ & $\mathbbx{F}$ & $\False$ & $\True$ &  &&  & $\True$ & $\mathbbx{F}$ &  & $\False$ & $\False$ & $\False$ &  &  & \\
 $\False$ & $\False$ & $\True$ &  &  & $\False$ & $\mathbbx{F}$ & $\True$ & $\False$ &  &&  & $\False$ & $\mathbbx{T}$ &  & $\False$ & $\False$ & $\True$ &  &  & \\
 $\False$ & $\False$ & $\False$ &  &  & $\False$ & $\mathbbx{F}$ & $\True$ & $\False$ &  &&  & $\False$ & $\mathbbx{T}$ &  & $\False$ & $\False$ & $\False$ &  &  & \\
\end{tabular}
\bigskip
\end{center}

\end{frame}

\begin{frame}{Row 4}
\protect\hypertarget{row-4-1}{}

That's also \(\True \rightarrow \True\), i.e., \(\True\).

\begin{center}
\bigskip
\begin{tabular}{@{ }c@{ }@{ }c@{ }@{ }c | c@{ }@{}c@{}@{ }c@{ }@{ }c@{ }@{ }c@{ }@{ }c@{ }@{}c@{}@{ }c@{ }@{}c@{}@{ }c@{ }@{ }c@{ }@{}c@{}@{ }c@{ }@{ }c@{ }@{ }c@{ }@{}c@{}@{}c@{}@{ }c}
A & B & C &  & ( & A & $\vee$ & $\neg$ & B & ) & $\rightarrow$ & ( & B & $\rightarrow$ & ( & A & $\wedge$ & C & ) & ) & \\
\hline 
 $\True$ & $\True$ & $\True$ &  &  & $\True$ & $\mathbbx{T}$ & $\False$ & $\True$ &  &\textcolor{red}{$\True$}&  & $\True$ & $\mathbbx{T}$ &  & $\True$ & $\True$ & $\True$ &  &  & \\
 $\True$ & $\True$ & $\False$ &  &  & $\True$ & $\mathbbx{T}$ & $\False$ & $\True$ &  &\textcolor{red}{$\False$}&  & $\True$ & $\mathbbx{F}$ &  & $\True$ & $\False$ & $\False$ &  &  & \\
 $\True$ & $\False$ & $\True$ &  &  & $\True$ & $\mathbbx{T}$ & $\True$ & $\False$ &  &\textcolor{red}{$\True$}&  & $\False$ & $\mathbbx{T}$ &  & $\True$ & $\True$ & $\True$ &  &  & \\
 $\True$ & $\False$ & $\False$ &  &  & $\True$ & $\mathbbx{T}$ & $\True$ & $\False$ &  &\textcolor{red}{$\True$}&  & $\False$ & $\mathbbx{T}$ &  & $\True$ & $\False$ & $\False$ &  &  & \\
 $\False$ & $\True$ & $\True$ &  &  & $\False$ & $\mathbbx{F}$ & $\False$ & $\True$ &  &&  & $\True$ & $\mathbbx{F}$ &  & $\False$ & $\False$ & $\True$ &  &  & \\
 $\False$ & $\True$ & $\False$ &  &  & $\False$ & $\mathbbx{F}$ & $\False$ & $\True$ &  &&  & $\True$ & $\mathbbx{F}$ &  & $\False$ & $\False$ & $\False$ &  &  & \\
 $\False$ & $\False$ & $\True$ &  &  & $\False$ & $\mathbbx{F}$ & $\True$ & $\False$ &  &&  & $\False$ & $\mathbbx{T}$ &  & $\False$ & $\False$ & $\True$ &  &  & \\
 $\False$ & $\False$ & $\False$ &  &  & $\False$ & $\mathbbx{F}$ & $\True$ & $\False$ &  &&  & $\False$ & $\mathbbx{T}$ &  & $\False$ & $\False$ & $\False$ &  &  & \\
\end{tabular}
\bigskip
\end{center}

\end{frame}

\begin{frame}{Rows 5 and 6}
\protect\hypertarget{rows-5-and-6}{}

That's \(\False \rightarrow \False\), i.e., \(\True\).

\begin{center}
\bigskip
\begin{tabular}{@{ }c@{ }@{ }c@{ }@{ }c | c@{ }@{}c@{}@{ }c@{ }@{ }c@{ }@{ }c@{ }@{ }c@{ }@{}c@{}@{ }c@{ }@{}c@{}@{ }c@{ }@{ }c@{ }@{}c@{}@{ }c@{ }@{ }c@{ }@{ }c@{ }@{}c@{}@{}c@{}@{ }c}
A & B & C &  & ( & A & $\vee$ & $\neg$ & B & ) & $\rightarrow$ & ( & B & $\rightarrow$ & ( & A & $\wedge$ & C & ) & ) & \\
\hline 
 $\True$ & $\True$ & $\True$ &  &  & $\True$ & $\mathbbx{T}$ & $\False$ & $\True$ &  &\textcolor{red}{$\True$}&  & $\True$ & $\mathbbx{T}$ &  & $\True$ & $\True$ & $\True$ &  &  & \\
 $\True$ & $\True$ & $\False$ &  &  & $\True$ & $\mathbbx{T}$ & $\False$ & $\True$ &  &\textcolor{red}{$\False$}&  & $\True$ & $\mathbbx{F}$ &  & $\True$ & $\False$ & $\False$ &  &  & \\
 $\True$ & $\False$ & $\True$ &  &  & $\True$ & $\mathbbx{T}$ & $\True$ & $\False$ &  &\textcolor{red}{$\True$}&  & $\False$ & $\mathbbx{T}$ &  & $\True$ & $\True$ & $\True$ &  &  & \\
 $\True$ & $\False$ & $\False$ &  &  & $\True$ & $\mathbbx{T}$ & $\True$ & $\False$ &  &\textcolor{red}{$\True$}&  & $\False$ & $\mathbbx{T}$ &  & $\True$ & $\False$ & $\False$ &  &  & \\
 $\False$ & $\True$ & $\True$ &  &  & $\False$ & $\mathbbx{F}$ & $\False$ & $\True$ &  &\textcolor{red}{$\True$}&  & $\True$ & $\mathbbx{F}$ &  & $\False$ & $\False$ & $\True$ &  &  & \\
 $\False$ & $\True$ & $\False$ &  &  & $\False$ & $\mathbbx{F}$ & $\False$ & $\True$ &  &\textcolor{red}{$\True$}&  & $\True$ & $\mathbbx{F}$ &  & $\False$ & $\False$ & $\False$ &  &  & \\
 $\False$ & $\False$ & $\True$ &  &  & $\False$ & $\mathbbx{F}$ & $\True$ & $\False$ &  &&  & $\False$ & $\mathbbx{T}$ &  & $\False$ & $\False$ & $\True$ &  &  & \\
 $\False$ & $\False$ & $\False$ &  &  & $\False$ & $\mathbbx{F}$ & $\True$ & $\False$ &  &&  & $\False$ & $\mathbbx{T}$ &  & $\False$ & $\False$ & $\False$ &  &  & \\
\end{tabular}
\bigskip
\end{center}

\end{frame}

\begin{frame}{Rows 7 and 8}
\protect\hypertarget{rows-7-and-8-1}{}

That's \(\False \rightarrow \True\), i.e., \(\True\).

\begin{center}
\bigskip
\begin{tabular}{@{ }c@{ }@{ }c@{ }@{ }c | c@{ }@{}c@{}@{ }c@{ }@{ }c@{ }@{ }c@{ }@{ }c@{ }@{}c@{}@{ }c@{ }@{}c@{}@{ }c@{ }@{ }c@{ }@{}c@{}@{ }c@{ }@{ }c@{ }@{ }c@{ }@{}c@{}@{}c@{}@{ }c}
A & B & C &  & ( & A & $\vee$ & $\neg$ & B & ) & $\rightarrow$ & ( & B & $\rightarrow$ & ( & A & $\wedge$ & C & ) & ) & \\
\hline 
 $\True$ & $\True$ & $\True$ &  &  & $\True$ & $\mathbbx{T}$ & $\False$ & $\True$ &  &\textcolor{red}{$\True$}&  & $\True$ & $\mathbbx{T}$ &  & $\True$ & $\True$ & $\True$ &  &  & \\
 $\True$ & $\True$ & $\False$ &  &  & $\True$ & $\mathbbx{T}$ & $\False$ & $\True$ &  &\textcolor{red}{$\False$}&  & $\True$ & $\mathbbx{F}$ &  & $\True$ & $\False$ & $\False$ &  &  & \\
 $\True$ & $\False$ & $\True$ &  &  & $\True$ & $\mathbbx{T}$ & $\True$ & $\False$ &  &\textcolor{red}{$\True$}&  & $\False$ & $\mathbbx{T}$ &  & $\True$ & $\True$ & $\True$ &  &  & \\
 $\True$ & $\False$ & $\False$ &  &  & $\True$ & $\mathbbx{T}$ & $\True$ & $\False$ &  &\textcolor{red}{$\True$}&  & $\False$ & $\mathbbx{T}$ &  & $\True$ & $\False$ & $\False$ &  &  & \\
 $\False$ & $\True$ & $\True$ &  &  & $\False$ & $\mathbbx{F}$ & $\False$ & $\True$ &  &\textcolor{red}{$\True$}&  & $\True$ & $\mathbbx{F}$ &  & $\False$ & $\False$ & $\True$ &  &  & \\
 $\False$ & $\True$ & $\False$ &  &  & $\False$ & $\mathbbx{F}$ & $\False$ & $\True$ &  &\textcolor{red}{$\True$}&  & $\True$ & $\mathbbx{F}$ &  & $\False$ & $\False$ & $\False$ &  &  & \\
 $\False$ & $\False$ & $\True$ &  &  & $\False$ & $\mathbbx{F}$ & $\True$ & $\False$ &  &\textcolor{red}{$\True$}&  & $\False$ & $\mathbbx{T}$ &  & $\False$ & $\False$ & $\True$ &  &  & \\
 $\False$ & $\False$ & $\False$ &  &  & $\False$ & $\mathbbx{F}$ & $\True$ & $\False$ &  &\textcolor{red}{$\True$}&  & $\False$ & $\mathbbx{T}$ &  & $\False$ & $\False$ & $\False$ &  &  & \\
\end{tabular}
\bigskip
\end{center}

\end{frame}

\begin{frame}{Summing Up}
\protect\hypertarget{summing-up}{}

It's true everywhere except when \(A, B\) are both \(\True\), and \(C\)
is \(\False\).

\begin{center}
\bigskip
\begin{tabular}{@{ }c@{ }@{ }c@{ }@{ }c | c@{ }@{}c@{}@{ }c@{ }@{ }c@{ }@{ }c@{ }@{ }c@{ }@{}c@{}@{ }c@{ }@{}c@{}@{ }c@{ }@{ }c@{ }@{}c@{}@{ }c@{ }@{ }c@{ }@{ }c@{ }@{}c@{}@{}c@{}@{ }c}
A & B & C &  & ( & A & $\vee$ & $\neg$ & B & ) & $\rightarrow$ & ( & B & $\rightarrow$ & ( & A & $\wedge$ & C & ) & ) & \\
\hline 
 $\True$ & $\True$ & $\True$ &  &  & $\True$ & $\mathbbx{T}$ & $\False$ & $\True$ &  &\textcolor{red}{$\True$}&  & $\True$ & $\mathbbx{T}$ &  & $\True$ & $\True$ & $\True$ &  &  & \\
 $\True$ & $\True$ & $\False$ &  &  & $\True$ & $\mathbbx{T}$ & $\False$ & $\True$ &  &\textcolor{red}{$\False$}&  & $\True$ & $\mathbbx{F}$ &  & $\True$ & $\False$ & $\False$ &  &  & \\
 $\True$ & $\False$ & $\True$ &  &  & $\True$ & $\mathbbx{T}$ & $\True$ & $\False$ &  &\textcolor{red}{$\True$}&  & $\False$ & $\mathbbx{T}$ &  & $\True$ & $\True$ & $\True$ &  &  & \\
 $\True$ & $\False$ & $\False$ &  &  & $\True$ & $\mathbbx{T}$ & $\True$ & $\False$ &  &\textcolor{red}{$\True$}&  & $\False$ & $\mathbbx{T}$ &  & $\True$ & $\False$ & $\False$ &  &  & \\
 $\False$ & $\True$ & $\True$ &  &  & $\False$ & $\mathbbx{F}$ & $\False$ & $\True$ &  &\textcolor{red}{$\True$}&  & $\True$ & $\mathbbx{F}$ &  & $\False$ & $\False$ & $\True$ &  &  & \\
 $\False$ & $\True$ & $\False$ &  &  & $\False$ & $\mathbbx{F}$ & $\False$ & $\True$ &  &\textcolor{red}{$\True$}&  & $\True$ & $\mathbbx{F}$ &  & $\False$ & $\False$ & $\False$ &  &  & \\
 $\False$ & $\False$ & $\True$ &  &  & $\False$ & $\mathbbx{F}$ & $\True$ & $\False$ &  &\textcolor{red}{$\True$}&  & $\False$ & $\mathbbx{T}$ &  & $\False$ & $\False$ & $\True$ &  &  & \\
 $\False$ & $\False$ & $\False$ &  &  & $\False$ & $\mathbbx{F}$ & $\True$ & $\False$ &  &\textcolor{red}{$\True$}&  & $\False$ & $\mathbbx{T}$ &  & $\False$ & $\False$ & $\False$ &  &  & \\
\end{tabular}
\bigskip
\end{center}

\end{frame}

\begin{frame}{For Next Time}
\protect\hypertarget{for-next-time}{}

We'll finish our discussion of truth tables with discussion of what we
can do with truth tables.

\end{frame}

\end{document}
