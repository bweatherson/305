% Options for packages loaded elsewhere
\PassOptionsToPackage{unicode}{hyperref}
\PassOptionsToPackage{hyphens}{url}
%
\documentclass[
  ignorenonframetext,
]{beamer}
\usepackage{pgfpages}
\setbeamertemplate{caption}[numbered]
\setbeamertemplate{caption label separator}{: }
\setbeamercolor{caption name}{fg=normal text.fg}
\beamertemplatenavigationsymbolsempty
% Prevent slide breaks in the middle of a paragraph
\widowpenalties 1 10000
\raggedbottom
\setbeamertemplate{part page}{
  \centering
  \begin{beamercolorbox}[sep=16pt,center]{part title}
    \usebeamerfont{part title}\insertpart\par
  \end{beamercolorbox}
}
\setbeamertemplate{section page}{
  \centering
  \begin{beamercolorbox}[sep=12pt,center]{part title}
    \usebeamerfont{section title}\insertsection\par
  \end{beamercolorbox}
}
\setbeamertemplate{subsection page}{
  \centering
  \begin{beamercolorbox}[sep=8pt,center]{part title}
    \usebeamerfont{subsection title}\insertsubsection\par
  \end{beamercolorbox}
}
\AtBeginPart{
  \frame{\partpage}
}
\AtBeginSection{
  \ifbibliography
  \else
    \frame{\sectionpage}
  \fi
}
\AtBeginSubsection{
  \frame{\subsectionpage}
}
\usepackage{lmodern}
\usepackage{amssymb,amsmath}
\usepackage{ifxetex,ifluatex}
\ifnum 0\ifxetex 1\fi\ifluatex 1\fi=0 % if pdftex
  \usepackage[T1]{fontenc}
  \usepackage[utf8]{inputenc}
  \usepackage{textcomp} % provide euro and other symbols
\else % if luatex or xetex
  \usepackage{unicode-math}
  \defaultfontfeatures{Scale=MatchLowercase}
  \defaultfontfeatures[\rmfamily]{Ligatures=TeX,Scale=1}
\fi
% Use upquote if available, for straight quotes in verbatim environments
\IfFileExists{upquote.sty}{\usepackage{upquote}}{}
\IfFileExists{microtype.sty}{% use microtype if available
  \usepackage[]{microtype}
  \UseMicrotypeSet[protrusion]{basicmath} % disable protrusion for tt fonts
}{}
\makeatletter
\@ifundefined{KOMAClassName}{% if non-KOMA class
  \IfFileExists{parskip.sty}{%
    \usepackage{parskip}
  }{% else
    \setlength{\parindent}{0pt}
    \setlength{\parskip}{6pt plus 2pt minus 1pt}}
}{% if KOMA class
  \KOMAoptions{parskip=half}}
\makeatother
\usepackage{xcolor}
\IfFileExists{xurl.sty}{\usepackage{xurl}}{} % add URL line breaks if available
\IfFileExists{bookmark.sty}{\usepackage{bookmark}}{\usepackage{hyperref}}
\hypersetup{
  pdftitle={305 Lecture 37 - States and Choices},
  pdfauthor={Brian Weatherson},
  hidelinks,
  pdfcreator={LaTeX via pandoc}}
\urlstyle{same} % disable monospaced font for URLs
\newif\ifbibliography
\usepackage{longtable,booktabs}
\usepackage{caption}
% Make caption package work with longtable
\makeatletter
\def\fnum@table{\tablename~\thetable}
\makeatother
\setlength{\emergencystretch}{3em} % prevent overfull lines
\providecommand{\tightlist}{%
  \setlength{\itemsep}{0pt}\setlength{\parskip}{0pt}}
\setcounter{secnumdepth}{-\maxdimen} % remove section numbering
\let\Tiny=\tiny

 \setbeamertemplate{navigation symbols}{} 

% \usetheme{Madrid}
 \usetheme[numbering=none, progressbar=foot]{metropolis}
 \usecolortheme{wolverine}
 \usepackage{color}
 \usepackage{MnSymbol}
% \usepackage{movie15}

\usepackage{amssymb}% http://ctan.org/pkg/amssymb
\usepackage{pifont}% http://ctan.org/pkg/pifont
\newcommand{\cmark}{\ding{51}}%
\newcommand{\xmark}{\ding{55}}%

\DeclareSymbolFont{symbolsC}{U}{txsyc}{m}{n}
\DeclareMathSymbol{\boxright}{\mathrel}{symbolsC}{128}
\DeclareMathAlphabet{\mathpzc}{OT1}{pzc}{m}{it}


% \usepackage{tikz-qtree}
% \usepackage{markdown}
% \usepackage{prooftrees}
% \forestset{not line numbering, close with = x}
% Allow for easy commas inside trees
\renewcommand{\,}{\text{, }}


\usepackage{tabulary}

\usepackage{open-logic-config}

\setlength{\parskip}{1ex plus 0.5ex minus 0.2ex}

\AtBeginSection[]
{
\begin{frame}
	\Huge{\color{darkblue} \insertsection}
\end{frame}
}

\renewenvironment*{quote}	
	{\list{}{\rightmargin   \leftmargin} \item } 	
	{\endlist }

\definecolor{darkgreen}{rgb}{0,0.7,0}
\definecolor{darkblue}{rgb}{0,0,0.8}

\newcommand{\starttab}{\begin{center}
\vspace{6pt}
\begin{tabular}}

\newcommand{\stoptab}{\end{tabular}
\vspace{6pt}
\end{center}
\noindent}


\newcommand{\sif}{\rightarrow}
\newcommand{\siff}{\leftrightarrow}
\newcommand{\EF}{\end{frame}}


\newcommand{\TreeStart}[1]{
%\end{frame}
\begin{frame}
\begin{center}
\begin{tikzpicture}[scale=#1]
\tikzset{every tree node/.style={align=center,anchor=north}}
%\Tree
}

\newcommand{\TreeEnd}{
\end{tikzpicture}
%\end{center}
}

\newcommand{\DisplayArg}[2]{
\begin{enumerate}
{#1}
\end{enumerate}
\vspace{-6pt}
\hrulefill

%\hspace{14pt} #2
%{\addtolength{\leftskip}{14pt} #2}
\begin{quote}
{\normalfont #2}
\end{quote}
\vspace{12pt}
}

\newenvironment{ProofTree}[1][1]{
\begin{center}
\begin{tikzpicture}[scale=#1]
\tikzset{every tree node/.style={align=center,anchor=south}}
}
{
\end{tikzpicture}
\end{center}
}

\newcommand{\TreeFrame}[2]{
\begin{columns}[c]
\column{0.5\textwidth}
\begin{center}
\begin{prooftree}{}
#1
\end{prooftree}
\end{center}
\column{0.45\textwidth}
%\begin{markdown}
#2
%\end{markdown}
\end{columns}
}

\newcommand{\ScaledTreeFrame}[3]{
\begin{columns}[c]
\column{0.5\textwidth}
\begin{center}
\scalebox{#1}{
\begin{prooftree}{}
#2
\end{prooftree}
}
\end{center}
\column{0.45\textwidth}
%\begin{markdown}
#3
%\end{markdown}
\end{columns}
}

\usepackage[bb=boondox]{mathalfa}
\DeclareMathAlphabet{\mathbx}{U}{BOONDOX-ds}{m}{n}
\SetMathAlphabet{\mathbx}{bold}{U}{BOONDOX-ds}{b}{n}
\DeclareMathAlphabet{\mathbbx} {U}{BOONDOX-ds}{b}{n}

\RequirePackage{bussproofs}
\RequirePackage[tableaux]{prooftrees}

\newenvironment{oltableau}{\center\tableau{}} %wff format={anchor = base west}}}
       {\endtableau\endcenter}
       
\newcommand{\formula}[1]{$#1$}

\usepackage{tabulary}
\usepackage{booktabs}

\def\begincols{\begin{columns}}
\def\begincol{\begin{column}}
\def\endcol{\end{column}}
\def\endcols{\end{columns}}

\usepackage[italic]{mathastext}
\usepackage{nicefrac}

\definecolor{mygreen}{RGB}{0, 100, 0}
\definecolor{mypink2}{RGB}{219, 48, 122}
\definecolor{dodgerblue}{RGB}{30,144,255}

\def\True{\textcolor{dodgerblue}{\text{T}}}
\def\False{\textcolor{red}{\text{F}}}

\title{305 Lecture 37 - States and Choices}
\author{Brian Weatherson}
\date{July 27, 2020}

\begin{document}
\frame{\titlepage}

\begin{frame}{Plan}
\protect\hypertarget{plan}{}

\begin{itemize}
\tightlist
\item
  In this lecture we'll talk about how to formulate decision problems.
\item
  And we'll talk about one simple way to analyse these problems, using
  so-called dominance reasoning.
\end{itemize}

\end{frame}

\begin{frame}{Associated Reading}
\protect\hypertarget{associated-reading}{}

Odds and Ends, Chapter 12

\end{frame}

\begin{frame}{States and Choices}
\protect\hypertarget{states-and-choices}{}

\begin{itemize}
\tightlist
\item
  We're interested in what to do when the outcomes of your actions
  depend on some external facts about which you are uncertain, e.g.,
\end{itemize}

\begin{longtable}[]{@{}rcc@{}}
\toprule
& State 1 & State 2\tabularnewline
\midrule
\endhead
Choice 1 & \(a\) & \(b\)\tabularnewline
Choice 2 & \(c\) & \(d\)\tabularnewline
\bottomrule
\end{longtable}

\end{frame}

\begin{frame}{States and Choices}
\protect\hypertarget{states-and-choices-1}{}

\begin{itemize}
\tightlist
\item
  The \textbf{choices} are the options you can take.
\item
  The \textbf{states} are the ways the world can be that affect how good
  an outcome you'll get.
\item
  A choice plus a state determines an \textbf{outcome}
\item
  And the variables, \(a\), \(b\), \(c\) and \(d\) are numbers measuring
  how good those outcomes are.
\item
  We'll call this the \textbf{utility} of the outcome.
\item
  The higher the number, the better.
\end{itemize}

\end{frame}

\begin{frame}{What is Utility}
\protect\hypertarget{what-is-utility}{}

\begin{itemize}
\tightlist
\item
  The book spends one chapter (11) on cases where value is easy to
  measure - it's something like dollars won or lost, or time spent.
\item
  Then it spends the next chapter (i.e., 12) on cases where value is
  more abstract.
\item
  I'm not going to carve things up this way.
\item
  But I will start with a case where the values are fairly clear.
\end{itemize}

\end{frame}

\begin{frame}{An Example}
\protect\hypertarget{an-example}{}

\begin{itemize}
\tightlist
\item
  It's a Sunday afternoon in Fall, and your friend, who is a big Packers
  fan, has the choice between watching the Packers game and finishing a
  paper due on Monday.
\item
  It will be a little painful for them to do the paper after the
  football, but not impossible.
\item
  It will be fun to watch football, at least if the Packers wins.
\item
  But if the Packers lose they'll have spent the afternoon watching them
  lose, and still have the paper to write.
\item
  On the other hand, your friend will feel bad if they skip the game and
  the Packers win. So we might have the decision table on the next
  slide.
\end{itemize}

\end{frame}

\begin{frame}{Decision Table for the Example}
\protect\hypertarget{decision-table-for-the-example}{}

\begin{longtable}[]{@{}rcc@{}}
\toprule
& Packers Win & Packers Lose\tabularnewline
\midrule
\endhead
Watch Football & 4 & 1\tabularnewline
Work on Paper & 3 & 2\tabularnewline
\bottomrule
\end{longtable}

\begin{itemize}
\tightlist
\item
  The numbers come from the preferences.
\item
  We're assuming (for now) that what one wants is better for one.
\item
  That assumption could get either conceptual backing (utility is
  defined in terms of preference) or empirical backing.
\end{itemize}

\end{frame}

\begin{frame}{Changing Preferences}
\protect\hypertarget{changing-preferences}{}

\begin{itemize}
\tightlist
\item
  The numbers would be different if your friend had different
  preferences.
\item
  Perhaps their desire to watch football is simply stronger than their
  desire to finish the paper.
\item
  In that case the table might look like this.
\end{itemize}

\begin{longtable}[]{@{}rcc@{}}
\toprule
& Packers Win & Packers Lose\tabularnewline
\midrule
\endhead
Watch Football & 4 & 2\tabularnewline
Work on Paper & 3 & 1\tabularnewline
\bottomrule
\end{longtable}

\end{frame}

\begin{frame}{Dominance Reasoning}
\protect\hypertarget{dominance-reasoning}{}

\begin{itemize}
\tightlist
\item
  The simplest rule we can use for decision making is \textbf{never
  choose dominated options}.
\item
  There is a stronger and a weaker version of this rule.
\end{itemize}

\end{frame}

\begin{frame}{Weak and Strong Dominance}
\protect\hypertarget{weak-and-strong-dominance}{}

\begin{itemize}
\tightlist
\item
  An option \(A\) \textbf{strongly dominates} another option \(B\) no
  matter which state is actual, A leads to better outcomes than B.
  \pause 
\item
  \(A\) \textbf{weakly dominates} B if in every state, A leads to at
  least as good an outcome as B, and in some states it leads to a better
  outcome.
\end{itemize}

\end{frame}

\begin{frame}{Dominance Principles}
\protect\hypertarget{dominance-principles}{}

\begin{itemize}
\tightlist
\item
  Principle 1: If A strongly dominates B, don't choose B.
\item
  Principle 2: If A weakly dominates B, don't choose B.
\item
  The second principle is slightly \textbf{stronger}; it rules out more
  things.
\item
  As such, it is slightly more controversial.
\end{itemize}

\end{frame}

\begin{frame}{Using Dominance Principles}
\protect\hypertarget{using-dominance-principles}{}

\begin{itemize}
\tightlist
\item
  Dominance principles seem very intuitive when applied to everyday
  decision cases.
\item
  Consider, for example, a revised version of our case about choosing
  whether to watch football or work on a term paper.
\item
  Imagine that your friend will do very badly on the term paper if they
  leave it to the last minute.
\item
  And imagine that the term paper is vitally important for something
  that matters to their future.
\item
  Then we might set up the decision table as on the next slide.
\end{itemize}

\end{frame}

\begin{frame}{Football Example with Dominance}
\protect\hypertarget{football-example-with-dominance}{}

\begin{longtable}[]{@{}rcc@{}}
\toprule
& Packers Win & Packers Lose\tabularnewline
\midrule
\endhead
Watch Football & 3 & 1\tabularnewline
Work on Paper & 4 & 2\tabularnewline
\bottomrule
\end{longtable}

\begin{itemize}
\tightlist
\item
  They are better off working on the paper if the Packers win.
\item
  And they are better off working on the paper if the Packers lose.
\item
  So either way, they should work on the paper!
\end{itemize}

\end{frame}

\begin{frame}{For Next Time}
\protect\hypertarget{for-next-time}{}

\begin{itemize}
\tightlist
\item
  We will look at how to think about decisions where dominance reasoning
  doesn't apply.
\end{itemize}

\end{frame}

\end{document}
