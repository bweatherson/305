% Options for packages loaded elsewhere
\PassOptionsToPackage{unicode}{hyperref}
\PassOptionsToPackage{hyphens}{url}
%
\documentclass[
  ignorenonframetext,
]{beamer}
\usepackage{pgfpages}
\setbeamertemplate{caption}[numbered]
\setbeamertemplate{caption label separator}{: }
\setbeamercolor{caption name}{fg=normal text.fg}
\beamertemplatenavigationsymbolsempty
% Prevent slide breaks in the middle of a paragraph
\widowpenalties 1 10000
\raggedbottom
\setbeamertemplate{part page}{
  \centering
  \begin{beamercolorbox}[sep=16pt,center]{part title}
    \usebeamerfont{part title}\insertpart\par
  \end{beamercolorbox}
}
\setbeamertemplate{section page}{
  \centering
  \begin{beamercolorbox}[sep=12pt,center]{part title}
    \usebeamerfont{section title}\insertsection\par
  \end{beamercolorbox}
}
\setbeamertemplate{subsection page}{
  \centering
  \begin{beamercolorbox}[sep=8pt,center]{part title}
    \usebeamerfont{subsection title}\insertsubsection\par
  \end{beamercolorbox}
}
\AtBeginPart{
  \frame{\partpage}
}
\AtBeginSection{
  \ifbibliography
  \else
    \frame{\sectionpage}
  \fi
}
\AtBeginSubsection{
  \frame{\subsectionpage}
}
\usepackage{lmodern}
\usepackage{amssymb,amsmath}
\usepackage{ifxetex,ifluatex}
\ifnum 0\ifxetex 1\fi\ifluatex 1\fi=0 % if pdftex
  \usepackage[T1]{fontenc}
  \usepackage[utf8]{inputenc}
  \usepackage{textcomp} % provide euro and other symbols
\else % if luatex or xetex
  \usepackage{unicode-math}
  \defaultfontfeatures{Scale=MatchLowercase}
  \defaultfontfeatures[\rmfamily]{Ligatures=TeX,Scale=1}
\fi
% Use upquote if available, for straight quotes in verbatim environments
\IfFileExists{upquote.sty}{\usepackage{upquote}}{}
\IfFileExists{microtype.sty}{% use microtype if available
  \usepackage[]{microtype}
  \UseMicrotypeSet[protrusion]{basicmath} % disable protrusion for tt fonts
}{}
\makeatletter
\@ifundefined{KOMAClassName}{% if non-KOMA class
  \IfFileExists{parskip.sty}{%
    \usepackage{parskip}
  }{% else
    \setlength{\parindent}{0pt}
    \setlength{\parskip}{6pt plus 2pt minus 1pt}}
}{% if KOMA class
  \KOMAoptions{parskip=half}}
\makeatother
\usepackage{xcolor}
\IfFileExists{xurl.sty}{\usepackage{xurl}}{} % add URL line breaks if available
\IfFileExists{bookmark.sty}{\usepackage{bookmark}}{\usepackage{hyperref}}
\hypersetup{
  pdftitle={305 Lecture 08 - Indirect Derivations},
  pdfauthor={Brian Weatherson},
  hidelinks,
  pdfcreator={LaTeX via pandoc}}
\urlstyle{same} % disable monospaced font for URLs
\newif\ifbibliography
\setlength{\emergencystretch}{3em} % prevent overfull lines
\providecommand{\tightlist}{%
  \setlength{\itemsep}{0pt}\setlength{\parskip}{0pt}}
\setcounter{secnumdepth}{-\maxdimen} % remove section numbering
\let\Tiny=\tiny

 \setbeamertemplate{navigation symbols}{} 

% \usetheme{Madrid}
 \usetheme[numbering=none, progressbar=foot]{metropolis}
 \usecolortheme{wolverine}
 \usepackage{color}
 \usepackage{MnSymbol}
% \usepackage{movie15}

\usepackage{amssymb}% http://ctan.org/pkg/amssymb
\usepackage{pifont}% http://ctan.org/pkg/pifont
\newcommand{\cmark}{\ding{51}}%
\newcommand{\xmark}{\ding{55}}%

\DeclareSymbolFont{symbolsC}{U}{txsyc}{m}{n}
\DeclareMathSymbol{\boxright}{\mathrel}{symbolsC}{128}
\DeclareMathAlphabet{\mathpzc}{OT1}{pzc}{m}{it}


% \usepackage{tikz-qtree}
% \usepackage{markdown}
% \usepackage{prooftrees}
% \forestset{not line numbering, close with = x}
% Allow for easy commas inside trees
\renewcommand{\,}{\text{, }}


\usepackage{tabulary}

\usepackage{open-logic-config}

\setlength{\parskip}{1ex plus 0.5ex minus 0.2ex}

\AtBeginSection[]
{
\begin{frame}
	\Huge{\color{darkblue} \insertsection}
\end{frame}
}

\renewenvironment*{quote}	
	{\list{}{\rightmargin   \leftmargin} \item } 	
	{\endlist }

\definecolor{darkgreen}{rgb}{0,0.7,0}
\definecolor{darkblue}{rgb}{0,0,0.8}

\newcommand{\starttab}{\begin{center}
\vspace{6pt}
\begin{tabular}}

\newcommand{\stoptab}{\end{tabular}
\vspace{6pt}
\end{center}
\noindent}


\newcommand{\sif}{\rightarrow}
\newcommand{\siff}{\leftrightarrow}
\newcommand{\EF}{\end{frame}}


\newcommand{\TreeStart}[1]{
%\end{frame}
\begin{frame}
\begin{center}
\begin{tikzpicture}[scale=#1]
\tikzset{every tree node/.style={align=center,anchor=north}}
%\Tree
}

\newcommand{\TreeEnd}{
\end{tikzpicture}
%\end{center}
}

\newcommand{\DisplayArg}[2]{
\begin{enumerate}
{#1}
\end{enumerate}
\vspace{-6pt}
\hrulefill

%\hspace{14pt} #2
%{\addtolength{\leftskip}{14pt} #2}
\begin{quote}
{\normalfont #2}
\end{quote}
\vspace{12pt}
}

\newenvironment{ProofTree}[1][1]{
\begin{center}
\begin{tikzpicture}[scale=#1]
\tikzset{every tree node/.style={align=center,anchor=south}}
}
{
\end{tikzpicture}
\end{center}
}

\newcommand{\TreeFrame}[2]{
\begin{columns}[c]
\column{0.5\textwidth}
\begin{center}
\begin{prooftree}{}
#1
\end{prooftree}
\end{center}
\column{0.45\textwidth}
%\begin{markdown}
#2
%\end{markdown}
\end{columns}
}

\newcommand{\ScaledTreeFrame}[3]{
\begin{columns}[c]
\column{0.5\textwidth}
\begin{center}
\scalebox{#1}{
\begin{prooftree}{}
#2
\end{prooftree}
}
\end{center}
\column{0.45\textwidth}
%\begin{markdown}
#3
%\end{markdown}
\end{columns}
}

\usepackage[bb=boondox]{mathalfa}
\DeclareMathAlphabet{\mathbx}{U}{BOONDOX-ds}{m}{n}
\SetMathAlphabet{\mathbx}{bold}{U}{BOONDOX-ds}{b}{n}
\DeclareMathAlphabet{\mathbbx} {U}{BOONDOX-ds}{b}{n}

\RequirePackage{bussproofs}
\RequirePackage[tableaux]{prooftrees}

\newenvironment{oltableau}{\center\tableau{}} %wff format={anchor = base west}}}
       {\endtableau\endcenter}
       
\newcommand{\formula}[1]{$#1$}

\usepackage{tabulary}
\usepackage{booktabs}

\def\begincols{\begin{columns}}
\def\begincol{\begin{column}}
\def\endcol{\end{column}}
\def\endcols{\end{columns}}

\usepackage[italic]{mathastext}
\usepackage{nicefrac}

\definecolor{mygreen}{RGB}{0, 100, 0}
\definecolor{mypink2}{RGB}{219, 48, 122}
\definecolor{dodgerblue}{RGB}{30,144,255}

\def\True{\textcolor{dodgerblue}{\text{T}}}
\def\False{\textcolor{red}{\text{F}}}

\title{305 Lecture 08 - Indirect Derivations}
\author{Brian Weatherson}
\date{July 8, 2020}

\begin{document}
\frame{\titlepage}

\begin{frame}{Plan}
\protect\hypertarget{plan}{}

To introduce the idea of an \textbf{Indirect Derivation}, which is the
next most complicated form of derivation we'll look at.

\end{frame}

\begin{frame}{Associated Reading}
\protect\hypertarget{associated-reading}{}

Carnap book, chapter 4.

\end{frame}

\begin{frame}{Proving a Conditional}
\protect\hypertarget{proving-a-conditional}{}

Intuitively, they way to show \(A \rightarrow B\) is to
imagine/assume/suppose \(A\) is true, and show that then \(B\) will be
true as well.

\begin{itemize}
\tightlist
\item
  That's what we'll do in Carnap.
\end{itemize}

\end{frame}

\begin{frame}{Three New Tricks}
\protect\hypertarget{three-new-tricks}{}

\begin{enumerate}
\tightlist
\item
  Having the `Show' be a conditional.
\item
  Starting with `AS' not `PR'.
\item
  Ending with Conditional Derivation - CD.
\end{enumerate}

\end{frame}

\begin{frame}[fragile]{Example}
\protect\hypertarget{example}{}

To prove: \(P \rightarrow Q \vdash P \rightarrow \neg \neg Q\)

\bigskip

\begincols
\begincol{.48\textwidth}

\begin{verbatim}
1. Show: P -> ~~Q
2.     P          :AS
3.     P -> Q     :PR
4.     Q          :MP 3, 2
5.     ~~Q        :DNI 4
6. :CD 5
\end{verbatim}

\endcol
\begincol{.48\textwidth}

What goes in `Show' is still the conclusion, but it isn't what we end
the proof with.

\endcol
\endcols

\end{frame}

\begin{frame}[fragile]{Example}
\protect\hypertarget{example-1}{}

To prove: \(P \rightarrow Q \vdash P \rightarrow \neg \neg Q\)

\bigskip

\begincols
\begincol{.48\textwidth}

\begin{verbatim}
1. Show: P -> ~~Q
2.     P          :AS
3.     P -> Q     :PR
4.     Q          :MP 3, 2
5.     ~~Q        :DNI 4
6. :CD 5
\end{verbatim}

\endcol
\begincol{.48\textwidth}

We start with two kinds of underived lines.

\begin{enumerate}
\tightlist
\item
  Assumptions
\item
  Premises
\end{enumerate}

\endcol
\endcols

\end{frame}

\begin{frame}[fragile]{Premises}
\protect\hypertarget{premises}{}

To prove: \(P \rightarrow Q \vdash P \rightarrow \neg \neg Q\)

\bigskip
\begincols
\begincol{.48\textwidth}

\begin{verbatim}
1. Show: P -> ~~Q
2.     P          :AS
3.     P -> Q     :PR
4.     Q          :MP 3, 2
5.     ~~Q        :DNI 4
6. :CD 5
\end{verbatim}

\endcol
\begincol{.48\textwidth}

The premises, in this case line 3, you are used to already.

\endcol
\endcols

\end{frame}

\begin{frame}[fragile]{Assumption}
\protect\hypertarget{assumption}{}

To prove: \(P \rightarrow Q \vdash P \rightarrow \neg \neg Q\)

\bigskip

\begincols
\begincol{.48\textwidth}

\begin{verbatim}
1. Show: P -> ~~Q
2.     P          :AS
3.     P -> Q     :PR
4.     Q          :MP 3, 2
5.     ~~Q        :DNI 4
6. :CD 5
\end{verbatim}

\endcol
\begincol{.48\textwidth}

The assumption is the thing on the left of what you're trying to prove -
the \textbf{antecedent} of the conditional.

\endcol
\endcols

\end{frame}

\begin{frame}[fragile]{Conditional Proof}
\protect\hypertarget{conditional-proof}{}

To prove: \(P \rightarrow Q \vdash P \rightarrow \neg \neg Q\)

\bigskip

\begincols
\begincol{.48\textwidth}

\begin{verbatim}
1. Show: P -> ~~Q
2.     P          :AS
3.     P -> Q     :PR
4.     Q          :MP 3, 2
5.     ~~Q        :DNI 4
6. :CD 5
\end{verbatim}

\endcol
\begincol{.48\textwidth}

Then you can use all the regular rules that you've used so far, with the
same constraints.

\endcol
\endcols

\end{frame}

\begin{frame}[fragile]{Conditional Proof}
\protect\hypertarget{conditional-proof-1}{}

To prove: \(P \rightarrow Q \vdash P \rightarrow \neg \neg Q\)

\bigskip

\begincols
\begincol{.48\textwidth}

\begin{verbatim}
1. Show: P -> ~~Q
2.     P          :AS
3.     P -> Q     :PR
4.     Q          :MP 3, 2
5.     ~~Q        :DNI 4
6. :CD 5
\end{verbatim}

\endcol
\begincol{.48\textwidth}

The big difference is that you end with the \textbf{consequent} of what
you're trying to show; the thing to the right of the \(\rightarrow\).

\endcol
\endcols

\end{frame}

\begin{frame}[fragile]{Conditional Proof}
\protect\hypertarget{conditional-proof-2}{}

To prove: \(P \rightarrow Q \vdash P \rightarrow \neg \neg Q\)

\bigskip

\begincols
\begincol{.48\textwidth}

\begin{verbatim}
1. Show: P -> ~~Q
2.     P          :AS
3.     P -> Q     :PR
4.     Q          :MP 3, 2
5.     ~~Q        :DNI 4
6. :CD 5
\end{verbatim}

\endcol
\begincol{.48\textwidth}

And then (and this is a bit distinctive to Carnap), you write `CD' for
Conditional Derivation, not `DD' for Direct Derivation.

\endcol
\endcols

\end{frame}

\begin{frame}{For Next Time}
\protect\hypertarget{for-next-time}{}

\begin{itemize}
\tightlist
\item
  Read chapters 4 and 5.
\item
  We will talk especially about what happens when these conditional
  derivations get \textbf{nested}.
\item
  Finish the first assignment - the exercises from chapters 1 and 3.
\end{itemize}

\end{frame}

\end{document}
